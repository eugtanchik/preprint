%!TeX root = book.tex
%!TEX TS-program = lualatex

\begin{russian}
\setcounter{secnumdepth}{-1}
\chapter{Заключение}
\markright{Заключение}
\setcounter{secnumdepth}{2}

В отчете приведены результаты исследований второго этапа бюджетной науч\-но-ис\-сле\-до\-ва\-тель\-ской работы, посвященной моделированию на\-пря\-же\-н\-но-де\-фор\-ми\-ро\-ва\-н\-но\-го состояния упругого материала, содержащего полости и включения. Рассмотрены глобальные модели, которые описывают поля напряжений и деформаций в реальных упругих пористых и композиционных материалах в областях между конечным числом  концентраторов напряжений. В качестве таковых рассмотрены: цилиндры, шары, вытянутые или сжатые сфероиды. Поля описываются аналитически точно при помощи базисных решений уравнения Ламе в канонических односвязных областях. Для определения параметров моделей при помощи обобщенного метода Фурье получены операторные уравнения с оптимальными свойствами, которые допускают эффективные численные решения. Приведен строгий аналитический анализ предложенных моделей, в результате которого определена область их эффективного применения. Создано программное обеспечение для численной реализации построенных моделей. На его основе проведен численный анализ и дана визуализация распределения напряжений в некоторых телах в зонах их максимальной концентрации. Исследована скорость сходимости приближенных методов решения операторных уравнений для определения параметров моделей. Проведено сравнение полученных результатов с результатами локальных моделей, исследованных на первом этапе работы. По результатам исследований можно сделать следующие выводы:

%\enlargethispage{4\baselineskip}

\begin{enumerate}
\item Особенностями полученных моделей являются:
\begin{itemize}
\item[а)] 	все построенные модели существенно неосесимметричны и неодносвязны;
\item[б)] модели аналитически определяют поля перемещений, напряжений и деформаций в теле;
\item[в)] модели позволяют точно учесть произвольную нагрузку, прикладываемую к телу;
\item[г)] предложенная структура моделей обуславливает оптимальность операторных уравнений для определения параметров моделей;
\item[д)] оптимальность связана с экспоненциальным убыванием матричных коэффициентов этих уравнений;
\item[е)] последнее свойство обеспечивает эффективную численную реализацию моделей, а также приближенные аналитические (в замкнутой форме) описания моделей.
\end{itemize}
\item Для проверки адекватности глобальные модели сравнивались с локальными моделями. Исследования показали, что первые можно заменять вторыми только в определенном диапазоне изменения геометрических и механических параметров.
\end{enumerate}
\end{russian}