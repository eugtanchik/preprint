% !TeX root = report.tex

%\refstepcounter{chapter}
%\addcontentsline{toc}{chapter}{Реферат}
\begin{center}
{\normalsize\textbf{\centering РЕФЕРАТ}}\vspace{14pt} 
\end{center}

%\chapter*{Реферат}

Отчет о НИР \pageref{LastPage}~с., 64~рис., 3~табл., 146~источников.

В отчете приводятся результаты исследований по моделированию на\-пря\-же\-н\-но-де\-фор\-ми\-ро\-ва\-н\-но\-го состояния (НДС) упругого тела с полостями или включениями цилиндрической, сферической, вытянутой или сжатой сфероидальной формы.

Объектом исследования настоящей работы является многосвязный кусочно-однородный упругий материал. Предметом исследования являются модели НДС многосвязного ку\-соч\-но-од\-но\-род\-но\-го упругого материала.

Целью и задачами данного исследования является построение ана\-ли\-ти\-ко-чис\-ле\-н\-ных моделей НДС упругих тел с полостями и включениями, указанной выше формы. Важной составной частью исследования является строгий анализ полученных моделей, определение областей их эффективности, численный анализ полей напряжений и деформаций на основе построенных моделей, проверка адекватности предлагаемых моделей.

Модели НДС указанных тел строятся на основе точных базисных решений уравнения Ламе в канонических пространственных областях. Для определения параметров модели используется обобщенный метод Фурье.

В рамках третьего этапа бюджетной темы были построены и исследованы периодические и стохастические модели НДС в телах с неоднородностями, которые описывают поля напряжений и деформаций реального пористого или композиционного материала в областях с регулярной структурой и различными видами упаковки неоднородностей. Проведен строгий аналитический анализ построенных моделей и определены области их эффективности. На основании построенных моделей предложен метод вычисления эффективных упругих модулей пористых и композиционных материалов. Проведены расчеты эффективных характеристик некоторых материалов.
Созданы пакеты прикладных программ для численной реализации построенных моделей. Проведен численный и качественный анализ распределения напряжений в некоторых рассматриваемых телах. Приведено сравнение изученных периодических моделей с локальными и глобальными моделями, построенными на первом и втором этапах исследования.

КЛЮЧЕВЫЕ СЛОВА: НАПРЯЖЕННО-ДЕФОРМИРОВАННОЕ СОСТОЯНИЕ, КУСОЧНО-ОДНОРОДНОЕ ТЕЛО, ОБОБЩЕННЫЙ МЕТОД ФУРЬЕ, ТЕОРЕМЫ СЛОЖЕНИЯ, ГЛОБАЛЬНАЯ МОДЕЛЬ, КРАЕВАЯ ЗАДАЧА, МЕТОД РЕДУКЦИИ, ПЕРИОДИЧЕСКАЯ МОДЕЛЬ, СТОХАСТИЧЕСКАЯ МОДЕЛЬ, ЭФФЕКТИВНЫЕ УПРУГИЕ МОДУЛИ.

