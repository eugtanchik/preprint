%!TEX TS-program = pdflatex
\documentclass[book,14pt,small,oneside]{ncc}
\usepackage[T2A]{fontenc}
\usepackage[utf8]{inputenc}
\usepackage[ukrainian,russian]{babel}
\usepackage[left=18mm,right=18mm,top=25mm,bottom=20mm]{geometry}
%\usepackage[
%  a4paper, mag=1000,
%  left=2.5cm, right=1cm, top=2cm, bottom=2cm, headsep=0.7cm, footskip=1cm
%]{geometry}
\usepackage{nccmath}
\usepackage{amsfonts,amssymb}
\usepackage{hyperref}
\usepackage{nccfancyhdr}
\usepackage[section,above,below]{placeins}
\usepackage{afterpage,flafter}
\usepackage{lastpage}
\usepackage{morefloats}
\usepackage{cite}
\usepackage{float,nccsect}

\usepackage[square,numbers,sort&compress]{natbib}

%\usepackage{xltxtra,xunicode}
%\defaultfontfeatures{Scale=MatchUppercase,Ligatures=TeX}
%\setromanfont[Numbers=Uppercase,Ligatures=TeX]{Arial}
%\setmainlanguage{english}
%\setotherlanguage{russian}

\renewtheorem{theorem}{Теорема}[theorem]

\pagestyle{plain}
%\renewcommand{\headrulewidth}{0pt}
%\lhead{}
%\rhead{\thepage}
%\chead{}
%\lfoot{}
%\rfoot{}
%\cfoot{}

%\DeclareSection*{0}{chapter}{\bf\normalsize}%
%{3.5ex plus 1ex minus .2ex}%
%{2.3ex plus .2ex}{
%\thispagestyle{fancy}
%\bf\normalsize\MakeUppercase}
%
%\DeclareSection*{1}{section}{}%
%{3.5ex plus 1ex minus .2ex}%
%{2.3ex plus .2ex}{\bf\normalsize}

%\newfloat{program}{tp}{lop}[chapter]
%\floatname{program}{Program}
%\RegisterFloatType{program}
%\DeclareSection{-3}{program}{\bff Табл.}{0pt}{0pt}{}
%\DeclareTOCEntry{-3}{}{}{9.9}{}

%\DeclareSection{-2}{table}{}{0pt}{10pt}{}
%\DeclareSection{-1}{figure}{\bff Рис.}{10pt}{0pt}{}
%\DeclareSection{-1}{theorem}{\bff Теорема}{10pt}{0pt}{}

%\NumberlineSuffix{.}{\enskip}

%\SectionTagSuffix{.\quad}

\setcounter{totalnumber}{10}
\setcounter{topnumber}{10}
%\setcounter{page}{1}
%\linespread{1.3}

\begin{document}

%\mathversion{bold}
\sectionstyle{center}
\indentaftersection
\sectiontagsuffix{.\;}
%\captiontagsuffix{~---~}

%\setyear{2015}
%\titlehead{Национальный аэрокосмический университет им.~Н.~Е.~Жуковского ``ХАИ''}
%\titlefoot{Харьков~--- \theyear}
%\author{Е.~А.~Танчик}
%\title{Упругая механика многокомпонентных тел}
%\maketitle
\begin{titlepage}
\begin{center}
Національний аерокосмічний університет ім.~М.~Є.~Жуковського ``ХАІ''
%Национальный аэрокосмический университет им.~Н.~Е.~Жуковского ``ХАИ''
\end{center}
\vskip1cm
\begin{flushright}
%На правах рукописи \\
УДК 539.3
\end{flushright}
\vskip2cm
\begin{center}
\large Танчік Євген Андрійович
\end{center}
\vskip1cm
\begin{center}
\Large\bf Просторові задачі теорії пружності для деяких класів неосесиметричних багатозв'язних тіл
%Пространственные задачи теории упругости для~некоторых классов неосесимметричных многосвязных тел
\end{center}
\vskip2cm
\begin{center}
01.02.04~--- Механіка деформівного твердого тіла
\end{center}
\vskip4cm
\begin{center}
АВТОРЕФЕРАТ \\
дисертації на здобуття наукового ступеня \\
кандидата фізико-математичних наук
\end{center}
\vskip2cm
%\begin{flushright}
%Научный руководитель \\
%д.~ф.-м.~н., проф. \\
%Николаев Алексей Георгиевич
%\end{flushright}
\vskip3cm
\begin{center}
Харкiв~--- 2015
\end{center}
\end{titlepage}

\begin{titlepage}
\noindent Дисертацією є рукопис
\vskip1cm
\noindent Робота виконана в Національному аерокосмічному університеті ім.~М.~Є.~Жуковського ``ХАІ'' Міністерства освіти і науки України, м.~Харків
\vskip1cm
\begin{desclist}{}{}[Науковий керівник:]
\item[Науковий керівник:] доктор фізико-математичних наук, професор, Ні\-ко\-ла\-єв \\ Оле\-к\-сій Георгійович, Національний аерокосмічний університет ім.~М.~Є.~Жуковського ``ХАІ'', завідувач кафедри вищої математики.
\item[Офіційні опоненти:] 
\item[Провідна установа:] Інститут прикладних проблем математики і механіки \\ ім.~Я.~С.~Підстригача НАН України
\end{desclist}
\vskip2cm
Захист відбудеться ``\underline{\hspace{1cm}}'' \underline{\hspace{3cm}} 2015~р. о \underline{\hspace{1cm}} годині на засіданні спеціалізованої вченої ради Д ХХ.ХХХ.ХХ в Дніпропетровському національному університеті ім. Олеся Гончара за адресою: 49010, м.~Дніпропетровськ, проспект Гагаріна, 72.

З дисертацією можна ознайомитись у бібліотеці Національного аерокосмічного університету ім.~М.~Є.~Жуковського ``ХАІ'', 61070, м.~Харків, вул.~Чкалова, 17.
\vskip2cm
Автореферат розісланий ``\underline{\hspace{1cm}}''\underline{\hspace{3cm}} 2015~р.
\vskip2cm
\noindent Вчений секретар спеціалізованої вченої ради \\
доктор фіз.-мат. наук, професор \hfill Поляков~М.~В.
\end{titlepage}

\begin{center}
ЗАГАЛЬНА ХАРАКТЕРИСТИКА РОБОТИ
\end{center}
{\bf Актуальність теми.} Сучасний рівень розвитку техніки і технології у високотехнологічних областях накладає підвищені вимоги на точність і ефективність моделей матеріалів, які широко використовуються в авіації та ракетобудуванні. Однією з найбільш важливих характеристик матеріалів, які тут застосовуються, є така комплексна характеристика, як мала питома маса і одночасно висока міцність матеріалу. Такою характеристикою володіють матеріали типу композитів, в яких присутні конструктивно закладені неоднорідності. При сучасному рівні моделювання з'являється можливість конструювання матеріалів з наперед заданими властивостями спочатку на рівні моделі, визначаючи оптимальну структуру, геометричні розміри і механічні характеристики неоднорідностей. І тільки після цього отримані в результаті моделювання дані можна втілювати в реальному матеріалі.

По самій назві композиційний матеріал~--- це складений матеріал, що має гетерогенну структуру. Різнорідні компоненти композиту мають різні фізико-механічні властивості. Особливе поєднання цих властивостей і геометрії неоднорідностей призводить до якісно нових характеристик композиту, відмінних від характеристик складових його фаз. Двофазний композит~--- це однорідний матеріал, армований волокнами або зернами з іншого матеріалу. Залежно від технології виготовлення армуючі елементи тим чи іншим способом змочуються сполучною речовиною (матрицею), яка після застигання забезпечує суцільність композиційної середовища і ідеальний механічний і тепловий контакт між різнорідними фазами. В якості армуючих елементів зазвичай застосовують матеріали з кристалічною або аморфною мікроструктурою, такі, як полімери, скла, метали та ін. До матеріалів заповнювача відносяться полімери, метали, кераміка. Властивості композиту істотно залежать від фізико-механічних характеристик арматури і матриці, геометрії арматури, структури армуючих елементів, характеру їх упаковки, об'ємного вмісту елементів, кутів армування та ін. Вкажемо на ті основні фізико-механічні характеристики композиційного матеріалу, які наводяться в нормативних документах і повинні контролюватися в процесі його виготовлення. Це модулі пружності та межі міцності на розтяг, стиск і поперечний зсув, коефіцієнти Пуассона та лінійного температурного розширення вздовж і поперек волокон, питома теплоємність та ін.

У цій роботі будуть розглянуті композиційні та пористі матеріали, що мають регулярну структуру. Під регулярною структурою розуміється періодичне розташування шарів, зерен або волокон в матеріалі. Вважається, що фізико-механічні характеристики включень однакові, але відрізняються від характеристик матриці. Далі будуть розглянуті тільки пружні моделі деформування цих матеріалів.

До теперішнього часу багато важливих задач механіки композиційних матеріалів залишаються невивченими або недостатньо вивченими. До них відносяться задачі визначення напружено-деформованого стану зразка пористого або композиційного матеріалів залежно від прикладеного зовнішнього навантаження, задача виявлення зон концентрації пружних напружень та областей, в першу чергу схильних до руйнування, задача аналізу напружень і деформацій в композиційному матеріалі з відшарованими включеннями, задачі з лінійними і внутріфазними тріщинами та ін. Актуальною залишається задача теоретичного визначення ефективних пружних модулів композиційних і пористих матеріалів. Всі ці проблеми в загальній постановці відносяться до класу дуже складних задач механіки деформованого твердого тіла з багатозв'язною неоднорідною структурою. До останнього часу ефективних методів вирішення подібних задач не існувало.

{\bf Зв'язок роботи з науковими програмами.} Дослідження за темою дисертації проводились в рамках держбюджетної наукової теми ``Нові методи дослідження лінійно і нелінійно деформованих тіл з композиційних матеріалів'', номер державної реєстрації № 0112U002135.

{\bf Мета і задачі дослідження.} Метою дослідження є розвиток узагальненого методу Фур'є на розв'язання неосесиметричних багатозв'язних задач теорії пружності з великим числом компонент зв'язності для канонічних тіл, обмежених поверхнями циліндра, сфери, витягнутого та стисненого сфероїдів.

{\it Об'єктом дослідження} є ізотропні пружні тіла під дією одновісного і двовісного розтягування.

{\it Предметом дослідження} є неосесиметричний пружний стан багатозв'язних або гетерогенних тіл.

{\it Методи дослідження.} Для досягнення поставленої мети в роботі розвинений узагальнений метод Фур'є, отримані теореми додавання (перерозкладання) для модифікованих базисних часткових розв'язків рівняння Ламе для сфери, витягнутого та стисненого сфероїдів. Дані теореми дозволяють представити загальний розв'язок задачі в системі координат, пов'язаній з кожною з граничних поверхонь і забезпечити точне задоволення граничних умов.

{\bf Наукова новизна отриманих результатів} полягає у розвитку узагальненого методу Фур'є для розв'язання неосесиметричних багатозв'язних задач теорії пружності для канонічних тіл, обмежених координатними поверхнями циліндра, сфери, витягнутого та стисненого сфероїдів.

{\bf Практичне значення отриманих результатів} полягає в тому, що розвинений в дисертації метод дозволяє обчислювати напружено-деформований стан в будь-яких неосесиметричних тілах описаної геометрії з великим числом компонент зв'язності під дією одновісного, двовісного або всебічного розтягування або стиснення пружного простору.

Отримані результати також використані для обчислення пружних модулів пористих і зернистих композиційних матеріалів.

{\bf Публікації.} Основні наукові результати опубліковані в 21 статтях у наукових журналах і збірниках праць, які відповідають вимогам ВАК України до опублікування результатів дисертації в профільних виданнях.

{\bf Структура роботи.} Дисертація складається зі вступу, 4 розділів, висновків, списку літератури з 140 найменувань. Загальний обсяг роботи~--- N сторінок.

\begin{center}
ОСНОВНИЙ ЗМІСТ РОБОТИ
\end{center}

У вступі обґрунтовано актуальність теми дисертаційної роботи; сформульовано мету та задачі досліджень; зроблено огляд літератури за даною тематикою; висвітлено новизну отриманих результатів, їх вірогідність та практичну значимість; наведено дані про апробацію результатів і публікації, що відражають основний зміст роботи, та особистий внесок здобувача. 

На підставі огляду літературних джерел висвітлені відомі методи роз\-в'язан\-ня крайових задач теорії пружності для однозв'язних циліндра, кулі та сфероїда. У них крайові задачі для зазначених областей розв'язані методом Фур'є. Відзначимо основоположний вклад, який внесли в ці дослідження Б.~Л.~Абрамян, А.~Я.~Александров, В.~М.~Александров, А.~Є.~Андрейків, Н.~Х.~Арутюнян, В.~А.~Бабешко, А.~А.~Баблоян, Н.~М.~Бородачов, І.~І.~Ворович, Л.~А.~Галін, В.~Т.~Головчан, В.~Т.~Грінченко, А.~Н.~Гузь, Г.~С.~Кіт, С.~О.~Калоєров, А.~С.~Космодаміанский, В.~Д.~Купрадзе, А.~І.~Лурьє, М.~А.~Мартиненко, В.~І.~Моссаковський, Ю.~Н.~Подільчук, Я.~С.~Підстригач, Г.~Я.~Попов, В.~С.~Проценко, В.~Л.~Рвачов, М.~П.~Саврук, А.~Ф.~Улітко, Я.~С.~Уфлянд, Л.~А.~Фільштинський та інші.

Детальніше розглянуто роботи, в яких використовується узагальнений метод Фур'є, висвітлені основні етапи і напрямки його розвитку. Відмічені фундаментальні дослідження Ю.~М.~Подільчука, А.~Ф.~Улітка, В.~Т.~Головчана, В.~С.~Проценка, О.~Г.~Ніколаєва, які присвячені побудові точних розв'язків основних задач теорії пружності в канонічних просторових однозв'язних та багатозв'язних областях і теорем додавання для них. Слід відзначити той факт, що дану роботу цілком присвячено розв'язанню неосесиметричних просторових задач теорії пружності. До цієї роботи в науковій літературі були досліджені практично тільки неосесиметричні задачі.

Перший розділ є теоретичною основою роботи. Він присвячений розвитку узагальненого методу Фур'є на неосесиметричні багатозв'язні задачі теорії пружності для циліндру, кулі та сфероїду з великою кількістю компонент зв'язності. Пункт 1.1 містить стислий виклад математичної техніки узагальненого методу Фур'є стосовно теорем додавання гармонійних функцій в циліндричних, сферичних, витягнутих та стиснених сфероїдальних системах координат із зсунутими початками та являє собою підсумок відповідних робіт О.~Г.~Ніколаєва.

\begin{center}
ОСНОВНІ ВИСНОВКИ І РЕЗУЛЬТАТИ РОБОТИ
\end{center}

У дисертації наведено результати досліджень, присвячені моделюванню на\-пруже\-но-де\-формів\-но\-го стану пружного матеріалу, що містить порожнини і включення. Розглянуто глобальні моделі, які описують поля напружень і деформацій в реальних пружних пористих і композиційних матеріалах в областях між скінченним числом концентраторів напружень. В якості таких розглянуті: циліндри, кулі, витягнуті сфероїди. Поля описуються аналітично точно за допомогою базисних розв'язків рівняння Ламе в канонічних однозв'язних областях. Для визначення параметрів моделей за допомогою узагальненого методу Фур'є отримані операторні рівняння з оптимальними властивостями, які допускають ефективне чисельне розв'язання. Наведено строгий аналітичний аналіз запропонованих моделей, в результаті якого визначено область їх ефективного застосування. Створено програмне забезпечення для чисельної реалізації побудованих моделей. На його основі проведено чисельний аналіз і дана візуалізація розподілу напружень в деяких тілах в зонах їх максимальної концентрації. Досліджено швидкість збіжності наближених методів розв'язання операторних рівнянь для визначення параметрів моделей. Проведено порівняння отриманих результатів з результатами локальних моделей. За результатами досліджень можна зробити наступні висновки:

\begin{enumerate}
\item Особливостями отриманих моделей є:
\begin{itemize}
\item[а)] всі побудовані моделі істотно неосесиметричні і неоднозв'язні;
\item[б)] моделі аналітично визначають поля переміщень, напружень і деформацій у тілі;
\item[в)] моделі дозволяють точно врахувати довільне навантаження, яке прикладено до тіла;
\item[г)] запропонована структура моделей обумовлює оптимальність операторних рівнянь для визначення параметрів моделей;
\item[д)] оптимальність пов'язана з експоненціальним спаданням матричних коефіцієнтів цих рівнянь;
\item[е)] остання властивість забезпечує ефективну чисельну реалізацію моделей, а також наближені аналітичні (в замкнутій формі) описи моделей.
\end{itemize}
\item Для перевірки адекватності глобальні моделі порівнювалися з локальними моделями. Дослідження показали, що перші можна замінювати другими тільки в певному діапазоні зміни геометричних і механічних параметрів.
\end{enumerate}

\begin{center}
СПИСОК ОПУБЛІКОВАНИХ ПРАЦЬ ЗА ТЕМОЮ ДИСЕРТАЦІЇ
\end{center}

\begin{biblist}{99}\setlength\itemsep{-1.9pt}

\bibitem{Nikolaev2015}
Николаев, А.~Г.
Упругая механика многокомпонентных тел [Текст]: монография 
/ А.~Г.~Николаев, Е.~А.~Танчик.~--- Х.: Нац. аэрокосм. ун-т им.~Н.~Е.~Жуковского ``Харьк. авиац. ин-т'', 2014.~--- 272~с.

\bibitem{Nikolaev2009}
Николаев, А.~Г. 
Математическая модель напряженно-деформированного состояния пористого материала [Текст] 
/ А.~Г.~Николаев, Е.~А.~Танчик 
// Вопросы проектирования и производства конструкций летательных аппаратов: сб. науч. тр. Нац. аэрокосм. ун-та им.~Н.~Е.~Жуковского ``ХАИ''.~--- Вып.~2(58).~--- Х., 2009.~--- С.~48--58.

\bibitem{Nikolaev2010}
Николаев, А.~Г. 
Локальная математическая модель зернистого композиционного материала [Текст] 
/ А.~Г.~Николаев, Е.~А.~Танчик 
// Вестн. Харьк. Нац. ун-та им.~В.~Н.~Каразина. Сер. Математика, прикладная математика и механика.~--- 2010.~--- Т.~922.~--- С.~4--19.

\bibitem{Nikolaev2012}
Николаев, А.~Г. 
Распределение напряжений в упругом пространстве с двумя параллельно расположенными сферическими полостями [Текст] 
/ А.~Г.~Николаев, Е.~А.~Танчик 
// Вопросы проектирования и производства конструкций летательных аппаратов: сб. науч. тр. Нац. аэрокосм. ун-та им.~Н.~Е.~Жуковского ``ХАИ''.~--- Вып.~4(72).~--- Х., 2012.~--- С.~92--99.

\bibitem{Nikolaev-Tanchik2012}
Николаев, А.~Г. 
Трехмерная периодическая модель зернистого композиционного материала [Текст] 
/ А.~Г.~Николаев, Е.~А.~Танчик 
// Методы решения прикладных задач механики деформируемого твердого тела: сб. науч. тр. Днепропетр. нац. ун-та им.~О.~Гончара.~--- Дп.: Лира.~--- 2012.~--- Вып.~13.~--- С.~287--293.

\bibitem{Nikolaev2013}
Николаев, А.~Г. 
Развитие локальной модели напряженного состояния пористого материала [Текст] 
/ А.~Г.~Николаев, Е.~А.~Танчик 
// Авиационно-космическая техника и технология.~--- 2013.~--- №1(98).~--- C.~14--18.

\bibitem{Nikolaev2013-1}
Николаев, А.~Г. 
Распределение напряжений в цилиндрическом образце материала с двумя параллельными цилиндрическими полостями [Текст] 
/ А.~Г.~Николаев, Е.~А.~Танчик 
// Вопросы проектирования и производства конструкций летательных аппаратов: сб. науч. тр. Нац. аэрокосм. ун-та им.~Н.~Е.~Жуковского ``ХАИ''.~--- Вып. 4(76).~--- Х., 2013.~--- С.~40--49.

\bibitem{Nikolaev2013-2}
Николаев, А.~Г. 
Напряженное состояние в цилиндрическом образце с двумя параллельными цилиндрическими волокнами [Текст] 
/ А.~Г.~Николаев, Е.~А.~Танчик 
// Авиационно-космическая техника и технология.~--- 2013.~--- №6(103).~--- С.~32--38.

\bibitem{Nikolaev2013-19}
Николаев, А.~Г. 
Хрупкое разрушение цилиндрического стержня с круговой трещиной при кручении [Текст] 
/ А.~Г.~Николаев, Е.~А.~Танчик, И.~С.~Тарасевич 
// Вопросы проектирования и производства конструкций летательных аппаратов: сб. науч. тр. Нац. аэрокосм. ун-та им.~Н.~Е.~Жуковского ``ХАИ''.~--- Вып.~2(74).~--- Х., 2013.~--- С.~64--73.

\bibitem{Nikolaev2013-4}
Николаев, А.~Г. 
Распределение напряжений в ячейке однонаправленного композиционного материала, образованного четырьмя цилиндрическими волокнами [Текст] / А.~Г.~Николаев, Е.~А.~Танчик 
// Вісник Одес. нац. ун-ту. Математика і механіка.~--- 2013.~--- Т.~18.~--- Вип.~4(20).~--- С.~64--73.

\bibitem{Nikolaev2014-1}
Николаев, А.~Г. 
Новые теоремы сложения базисных решений уравнения Ламе для вытянутых сфероидов и их применение к моделированию пористого материала [Текст] 
/ А.~Г.~Николаев, Е.~А.~Танчик 
// Авиационно-космическая техника и технология.~--- 2014.~--- №5(112).~--- С.~46--54.

\bibitem{Nikolaev2014-9}
Николаев, А.~Г. 
Развитие аппарата обобщенного метода Фурье на некоторые многосвязные области и его использование для моделирования пористого материала [Текст] 
/ А.~Г.~Николаев, Е.~А.~Танчик 
// Авиационно-космическая техника и технология.~--- 2014.~--- №6(113).~--- С.~48--56.

\bibitem{Nikolaev2013-3}
Ніколаєв, О.~Г. 
Напруження в нескінченному круговому циліндрі з чотирма циліндричними порожнинами [Текст] 
/ О.~Г.~Ніколаєв, Є.~А.~Танчік 
// Математичні методи та фізико-механічні поля.~--- 2014.~--- Т.~57, №3.~--- С.~51--60.

\bibitem{Nikolaev2014-2}
Николаев, А.~Г. 
Анализ напряженного состояния в окрестности двух сферических включений в упругом пространстве [Текст] 
/ А.~Г.~Николаев, Е.~А.~Танчик 
// Авиационно-космическая техника и технология.~--- 2014.~--- №3(110).~--- С.~26--32.

\bibitem{Nikolaev2014-3}
Николаев, А.~Г. 
Упругое пространство с четырьмя сфероидальными включениями под действием внешней нагрузки [Текст] 
/ А.~Г.~Николаев, Е.~А.~Танчик 
// Авиационно-космическая техника и технология.~--- 2014.~--- №4(111).~--- С.~49--55.

\bibitem{Nikolaev2014-4}
Николаев, А.~Г. 
Напряженное состояние пористого материала в области между четырьмя сфероидальными порами [Текст] 
/ А.~Г.~Николаев, Е.~А.~Танчик 
// Вісник нац. техн. ун-ту ``ХПІ''. Математичне моделювання в техніці і технологіях.~--- 2014.~--- №6(1049).~--- С.~151--160.

\bibitem{Nikolaev2014-5}
Николаев, А.~Г. 
Модель зернистого композита со сферическими зернами [Текст] 
/ А.~Г.~Николаев, Е.~А.~Танчик 
// Вісник нац. техн. ун-ту ``ХПІ''. Математичне моделювання в техніці і технологіях.~--- 2014.~--- №39(1082).~--- С.~141--152.

\bibitem{Nikolaev2014-6}
Николаев, А.~Г. 
Напряженное состояние в окрестности двух сфероидальных зерен в композите [Текст] 
/ А.~Г.~Николаев, Е.~А.~Танчик 
// Вопросы проектирования и производства конструкций летательных аппаратов: сб. науч. тр. Нац. аэрокосм. ун-та им.~Н.~Е.~Жуковского ``ХАИ''.~--- Вып.~1(77).~--- Х., 2014.~--- С.~73--86.

\bibitem{Nikolaev2014-7}
Николаев, А.~Г. 
Напряжения в упругом материале со сферическими порами под действием внешней нагрузки [Текст] 
/ А.~Г.~Николаев, Е.~А.~Танчик 
// Вопросы проектирования и производства конструкций летательных аппаратов: сб. науч. тр. Нац. аэрокосм. ун-та им.~Н.~Е.~Жуковского ``ХАИ''.~--- Вып.~2(78).~--- Х., 2014.~--- С.~99--110.

\bibitem{Nikolaev2015-1}
Николаев, А.~Г. 
Первая краевая задача теории упругости для цилиндра с N цилиндрическими полостями [Текст] 
/ А.~Г.~Николаев, Е.~А.~Танчик 
// Сиб. журн. вычисл. математики, РАН. Сиб. отд-ние.~--- Новосибирск.~--- 2015.~--- Т.~18, №2.~--- С.~177--188.

\bibitem{Nikolaev2014-8}
Nikolaev, A.~G. 
On the distribution of stresses in circular infinite cylinder with cylindrical cavities 
/ A.~G.~Nikolaev, E.~A.~Tanchik 
// Visn. Khark. Nat. Univ., Ser. Mat. Prykl. Mat. Mekh.~--- 2014.~--- V.~1120, Issue~69.~--- P.~4--19.

\bibitem{Nikolaev2015-2}
Nikolaev, A.~G.
The first boundary-value problem of the elasticity theory for a cylinder with N cylindrical cavities
/ A.~G.~Nikolaev, E.~A.~Tanchik
// Numerical Analysis and Applications.~--- V.~8, Issue~2.~--- P.~148--158.

\bibitem{Nikolaev2015-3}
Николаев, А.~Г.
Распределение напряжений в области четырех сжатых сфероидальных включений в упругом пространстве
/ А.~Г.~Николаев, Е.~А.~Танчик
// Вестн. ЗНУ ``Математическое моделирование и прикладная механика'': Сб. наук. статей. Физ.-мат. науки.~--- 2015.~--- №3.~--- С.~189--198.

\bibitem{Nikolaev2015-4}
Николаев, А.~Г.
Модель упругого состояния составного цилиндра, цилиндрические волокна которого образуют гексагональную структуру
/ А.~Г.~Николаев, Е.~А.~Танчик
// Механика композиционных материалов и конструкций.~--- 2015.~--- Т.~21, №1.~--- С.~135--147.

\end{biblist}

\begin{center}
АНОТАЦІЇ
\end{center}

Танчік~Є.~А. Просторові задачі теорії пружності для деяких класів неосесиметричних багатозв'язних тіл.~--- Рукопис.

Дисертація на здобуття наукового ступеня кандидата фізико-математичних наук за спеціальністю 01.02.04~--- механіка деформівного твердого тіла.~--- Національний аерокосмічний університет ім.~М.~Є.~Жуковського ``ХАІ'', Харків, 2015.

У дисертації наведено результати досліджень, присвячені моделюванню на\-пруже\-но-де\-формів\-но\-го стану пружного матеріалу, що містить порожнини і включення. Розглянуто глобальні моделі, які описують поля напружень і деформацій в реальних пружних пористих і композиційних матеріалах в областях між скінченним числом концентраторів напружень. В якості таких розглянуті: циліндри, кулі, витягнуті сфероїди. Поля описуються аналітично точно за допомогою базисних розв'язків рівняння Ламе в канонічних однозв'язних областях. Для визначення параметрів моделей за допомогою узагальненого методу Фур'є отримані операторні рівняння з оптимальними властивостями, які допускають ефективне чисельне розв'язання. Наведено строгий аналітичний аналіз запропонованих моделей, в результаті якого визначено область їх ефективного застосування. Створено програмне забезпечення для чисельної реалізації побудованих моделей. На його основі проведено чисельний аналіз і дана візуалізація розподілу напружень в деяких тілах в зонах їх максимальної концентрації. Досліджено швидкість збіжності наближених методів розв'язання операторних рівнянь для визначення параметрів моделей. Проведено порівняння отриманих результатів з результатами локальних моделей.

Ключові слова: узагальнений метод Фур'є, теореми додавання, неосесиметричні багатозв'язні тіла, композиційний матеріал, напружено-деформівний стан, просторові задачі теорії пружності.

Танчик~Е.~А. Пространственные задачи теории упругости для некоторых классов неосесимметричных многосвязных тел.~--- Рукопись.

Диссертация на соискание ученой степени кандидата фи\-зи\-ко-ма\-те\-ма\-ти\-че\-ских наук по специальности 01.02.04~--- механика деформированного твердого тела.~--- Национальный аэрокосмический университет им.~Н.~Е.~Жуковского ``ХАИ'', Харьков, 2015.

В диссертации приведены результаты исследований, посвященные моделированию на\-пря\-же\-н\-но-де\-фор\-ми\-ро\-ва\-н\-но\-го состояния упругого материала, содержащего полости и включения. Рассмотрены глобальные модели, которые описывают поля напряжений и деформаций в реальных упругих пористых и композиционных материалах в областях между конечным числом  концентраторов напряжений. В качестве таковых рассмотрены: цилиндры, шары, вытянутые сфероиды. Поля описываются аналитически точно при помощи базисных решений уравнения Ламе в канонических односвязных областях. Для определения параметров моделей при помощи обобщенного метода Фурье получены операторные уравнения с оптимальными свойствами, которые допускают эффективные численные решения. Приведен строгий аналитический анализ предложенных моделей, в результате которого определена область их эффективного применения. Создано программное обеспечение для численной реализации построенных моделей. На его основе проведен численный анализ и дана визуализация распределения напряжений в некоторых телах в зонах их максимальной концентрации. Исследована скорость сходимости приближенных методов решения операторных уравнений для определения параметров моделей. Проведено сравнение полученных результатов с результатами локальных моделей.

Ключевые слова: обобщенный метод Фурье, теоремы сложения, неосесимметричные многосвязные тела, композиционный материал, на\-пряже\-н\-но-де\-фор\-ми\-ро\-ван\-ное состояние, пространственные задачи теории упругости.

Tanchik~E.~A. The spatial problems of elasticity theory for certain classes of non-axisymmetric bodies multiply connected.~---Manuscript.

The thesis for the degree of candidate of physical and mathematical sciences on a specialty 01.02.04~--- Mechanics of Deformable Solids.~--- National Aerospace University ``KhAI'', Kharkiv, 2015.

The thesis presents the results of research devoted to modeling of stressed-strained state of an elastic material containing a cavities and inclusions. Reviewed global models that describe the stress and strain fields in real elastic porous and composite materials in the field between a finite number of stress concentrators. As such, we consider: cylinders, spheres, prolate spheroids. The fields are described analytically accurately using basic solutions of Lame canonical simply connected domains. To determine the model parameters using the generalized Fourier method derived operator equation with optimal properties that allow effective numerical solutions. An rigorous analytical analysis of the proposed models, in which defined the scope of their effective application. The software for the numerical im\-ple\-men\-ta\-ti\-on of the models is created. On the basis of numerical analysis and visualization of the distribution of stresses is given in some of the bodies in the areas of maximum concentration. We investigated the rate of convergence of approximate methods for solving operator equations to determine the parameters of the models. The results are compared with the results of the local models.

\end{document}
