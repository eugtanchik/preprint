% !TeX root = report.tex


\setcounter{secnumdepth}{-1}
\chapter{Выводы}
\markright{Выводы}
\setcounter{secnumdepth}{2}

В отчете приведены результаты исследований третьего этапа бюджетной науч\-но-ис\-сле\-до\-ва\-тель\-ской работы, посвященной моделированию на\-пря\-же\-н\-но-де\-фор\-ми\-ро\-ва\-н\-но\-го состояния упругого материала, содержащего полости и включения. Рассмотрены периодические модели, которые описывают поля напряжений и деформаций в реальных упругих пористых и композиционных материалах в областях между бесконечным числом  концентраторов напряжений, расположенных в узлах периодической решетки. В качестве таковых рассмотрены:  шары, вытянутые или сжатые сфероиды. Поля описываются аналитически точно при помощи базисных решений уравнения Ламе в канонических односвязных областях. Для определения параметров моделей при помощи обобщенного метода Фурье получены операторные уравнения с оптимальными свойствами, которые допускают эффективные численные решения. Приведен строгий аналитический анализ предложенных моделей, в результате которого определена область их эффективного применения. На основании полученных моделей предложен метод вычисления эффективных упругих модулей пористых и композиционных материалов.
Также исследованы стохастические модели зернистых и композиционных материалов, в которых радиусы неоднородностей представляют собой  независимые одинаково распределенные случайные величины. Исследована величина разброса нормальных компонент тензора напряжений в зависимости от среднеквадратичного отклонения радиусов неоднородностей.
Создано программное обеспечение для численной реализации построенных моделей. На его основе проведен численный анализ и дана визуализация распределения напряжений в некоторых телах в зонах их максимальной концентрации. Исследована скорость сходимости приближенных методов решения операторных уравнений для определения параметров моделей. Проведено сравнение полученных результатов с результатами локальных и глобальных моделей, исследованных на первом и втором этапах работы. По результатам исследований можно сделать следующие выводы:

\begin{enumerate}
\item Особенностями полученных моделей являются:
\begin{itemize}
\item[а)] впервые построены модели, которые позволяют моделировать напряженно-деформированное состояние в существенно неосесимметричных и неодносвязных телах с большим числом неоднородностей (в пределах от 2 до 125 и более);
\item[б)] модели аналитически определяют поля перемещений, напряжений и деформаций в теле;
\item[в)] модели являются аналитически точными при удовлетворении граничных условий для произвольной нагрузки, прикладываемой к телу;
\item[г)] предложенная структура моделей обуславливает оптимальность разрешающих операторных уравнений для определения параметров моделей;
\item[д)] оптимальность связана с экспоненциальным убыванием матричных коэффициентов этих уравнений;
\item[е)] последнее свойство обеспечивает эффективную численную реализацию моделей, а также приближенные аналитические (в замкнутой форме) описания моделей;
\item[ж)] исследования показали возможность эффективного развития построенных моделей на случай периодической структуры материала и материалов со случайными размерами неоднородностей;
\item[и)] построенные модели позволили с высокой точностью восстановить напряженно-деформированное состояние в представительской ячейке пористого и композиционного материалов;
\item[к)] построенные модели позволили предложить новый метод определения эффективных упругих модулей пористых и композиционных материалов, основанный на информации о полях напряжений и деформаций в неоднородном теле.
\end{itemize}
\item Для проверки адекватности проведено сравнение периодических моделей с локальными и глобальными моделями. Исследования показали, что только в определенном диапазоне изменения геометрических и механических параметров рассмотренные модели имеют близкие распределения напряжений и деформаций. Это указывает на важность всех изученных моделей.
\end{enumerate}
