%!TeX root = book.tex
%!TEX TS-program = lualatex

\begin{russian}
\setcounter{secnumdepth}{-1}
\chapter{Библиографический список}
\markright{Библиографический список}
\setcounter{secnumdepth}{2}

\renewcommand{\bibliststyle}{\normalsize}
\renewcommand{\bibnumfmt}[1]{#1.\hfill}

\begin{biblist}{99}\setlength\itemsep{-1.9pt}

\bibitem{Abramian}
Абрамян, Б.~Л. 
К задаче осесимметричной деформации круглого цилиндра [Текст] 
/ Б.~Л.~Абрамян 
// Докл. АН Арм. ССР.~--- 1954.~--- Т.~19, №1.~--- С.~3--12.

\bibitem{Akbarov}
Акбаров, С.~Д. 
Взаимодействие между двумя соседними круговыми отверстиями при изгибе предварительно растянутой шарнирно опертой ортотропной полосы [Текст] 
/ С.~Д.~Акбаров, Н.~Яхниоглу, У.~Б.~Есил~
// Механика композитных материалов.~--- Latvijas Universitates, Polimeru Mehanikas Instituts.~--- 2008.~--- Т.~44, №6.~--- С.~827--838.

\bibitem{Aleksandrov1978}
Александров, А.~Я. 
Пространственные задачи теории упругости [Текст] 
/ А.~Я.~Александров, Ю.~И.~Соловьёв.~--- М.: Наука, 1978.~--- 464~с.

\bibitem{Aleksandrov1973}
Александров, А.~Я. 
Решение основных трёхмерных задач теории упругости для тел произвольной формы путём численной реализации метода интегральных уравнений [Текст] / А.~Я.~Александров 
// Докл. АН СССР.~--- 1973.~--- Т.~208, №2.~--- С.~291--294.

\bibitem{Ambartsumian}
Амбарцумян, С.~А. 
Разномодульная теория упругости [Текст] 
/ С.~А.~Амбарцумян.~--- М.: Наука, 1967.~--- 266~с.

\bibitem{Andreev}
Определение напряжений в упругом пространстве со сферической полостью с учётом неоднородности [Текст] 
/ В.~И.~Андреев, А.~Б.~Зотов, В.~И.~Прокопьев, В.~Н.~Сидоров 
// Строит. механика и расчёт сооружений.~--- 1980.~--- №6.~--- С.~37--40.

\bibitem{Arutynian}
Арутюнян, Н.~Х. 
Поведение решений задач теории упругости в неограниченных областях с параболоидальными и цилиндрическими включениями или полостями [Текст] 
/ Н.~Х.~Арутюнян, А.~Б.~Мовчан, С.~А.~Назаров 
// Успехи механики.~--- 1987.~--- Т.~10, №4.~--- С.~3--91.

\bibitem{Bahvalov}
Бахвалов, Н.~С. 
Осредненные характеристики тел с периодической структурой  
/ Н.~С.~Бахвалов 
// ДАН СССР.~--- 1974.~--- Т.~218, №5.~--- С.~1046--1048.

\bibitem{Belov}
Белов, П.~А. 
Континуальная модель микрогетерогенных сред [Текст] 
/ П.~А.~Белов, С.~А.~Лурье 
// Прикладная математика и механика.~--- М.: Наука.~--- Т.~73, №5.~--- 2009.~--- С.~833--848.

\bibitem{Bolotin}
Болотин, В.~В. 
Механика многослойных конструкций [Текст] 
/ В.~В.~Болотин, Ю.~Н.~Новичков.~--- М.: Машиностроение, 1980.~--- 375~с.

\bibitem{Burchuladze}
Бурчуладзе, Т.~В. 
Развитие метода потенциала в теории упругости [Текст] 
/ Т.~В.~Бурчуладзе, Т.~Г.~Гегелия.~--- Тбилиси: Мецниереба, 1985.~--- 226~с.

\bibitem{Vavakin}
Вавакин, А.С. 
Эффективные упругие характеристики тел с изолированными трещинами, полостями и жесткими неоднородностями  
/ А.~С.~Вавакин, Р.~Л.~Салганик 
// Изв. АН СССР. МТТ.~--- 1978.~--- №2.~--- С.~95--107.

\bibitem{Valov}
Валов, Г.~М. 
Об осесимметричной деформации сплошного кругового цилиндра конечной длины [Текст] 
/ Г.~М.~Валов 
// Прикладная математика и механика.~--- 1962.~--- Т.~26.~--- Вып.~4.~--- С.~650--667.

\bibitem{VanFoFiKo1971}
Композиционные материалы волокнистого строения [Текст] / Г.~А.~Ван Фо Фы, В.~Н.~Грошева, Е.~Н.~Денбновецкая, Д.~Н.~Карпинос.~--- К.: Наук. думка, 1971.~--- 232~с.

\bibitem{VanFoFi1971}
Ван Фо Фы, Г.~А. 
Теория армированных материалов [Текст] 
/ Г.~А.~Ван Фо Фы.~--- К.: Наук. думка, 1971.~--- 232~с.

\bibitem{Vanin1985}
Ванин, Г.~А.
Микромеханика композиционных материалов [Текст] 
/ Г.~А.~Ванин.~--- К.: Наук. думка.~--- 1985.~--- 304~с.

\bibitem{Vanin1976}
Ванин, Г.~А. 
Новый метод учета взаимодействия в теории композиционных систем [Текст] / Г.~А.~Ванин
// Докл. АН УССР. Сер.~А.~--- 1976.~--- №4.~--- С.~321--324.

\bibitem{Vanin1980}
Ванин, Г.~А. 
Объемное упругое расширение среды с полыми сферическими включениями [Текст] 
/ Г.~А.~Ванин
// Прикл. механика.~--- 1980.~--- Т.~16, №7.~--- С.~127--129.

\bibitem{Vanin1977}
Ванин, Г.~А. 
Продольный сдвиг многокомпонентной упругой среды с дефектами [Текст] 
/ Г.~А.~Ванин
// Прикл. механика.~--- 1977.~--- Т.~13, №8.~--- С.~35--41.

\bibitem{Volpert1977}
Вольперт, В.~С. Осесимметричное напряжённое состояние пространства, содержащего систему сферических полостей или включений [Текст] 
/ В.~С.~Вольперт, И.~П.~Олегин 
// Новосиб. ин-т инж. ж.-д. транспорта.~--- 1977.~--- 19~с.~--- Деп. в ВИНИТИ. №3266--77.

\bibitem{Volpert1967}
Вольперт, В.~С. 
Пространственная задача теории упругости для эллипсоида вращения и эллипсоидальной полости [Текст] 
/ В.~С.~Вольперт 
// Изв. АН СССР. Механика твёрдого тела.~--- 1967.~--- №3.~--- С.~118--124.

\bibitem{Garishin}
Гаришин, О.~К. 
Прогнозирование прочности эластомерных зернистых композитов в зависимости от размеров частиц наполнителя [Текст] 
/ О.~К.~Гаришин, Л.~А.~Комар 
// Механика композиционных материалов и конструкций.~--- 2003.~--- Т.~9, №3.~--- С.~278--286.

\bibitem{Gerega}
Герега, А.~Н. 
Физические аспекты процессов самоорганизации в композитах. 1. Моделирование перколяционных кластеров фаз внутренних границ [Текст] 
/ А.~Н.~Герега 
// Механика композиционных материалов и конструкций.~--- 2013.~--- Т.~19, №3.~--- С.~406--419.

\bibitem{Golovchan1974}
Головчан, В.~Т. 
До розв’язку граничних задач статики пружного тіла, обмеженого сферичними поверхнями [Текст] 
/ В.~Т.~Головчан 
// Доп. АН УРСР. Сер.~А.~--- 1974.~--- №1.~--- С.~61--64.

\bibitem{Golovchan1993}
Механика композитов [Текст] 
/ В.~Т.~Головчан, А.~Н.~Гузь, Ю.~В.~Коханенко, В.~И.~Кущ.~--- К.: Наук. думка, 1993.~--- Т.~1: Статистика материалов.~--- 457~с.

\bibitem{Golovchan1987}
Головчан, В.~Т. 
Анизотропия физико-механических свойств композитных материалов.~--- К.: Наук. думка, 1987.~--- 304~с.

\bibitem{Golotina}
Голотина, Л.~А. 
Численное моделирование реологических свойств зернистого композита с использованием структурного подхода [Текст] 
/ Л.~А.~Голотина, Л.~Л.~Кожевникова, Т.~Б.~Кошкина 
// Механика композитных материалов.~--- 2008.~--- Т.~44, №6.~--- С.~895--906.

\bibitem{Gomilko}
Гомилко, А.~М. 
Однородные решения в задаче о равновесии упругого цилиндра конечной длины [Текст] 
/ А.~М.~Гомилко, В.~Т.~Гринченко, В.~В.~Мелешко 
// Теор. и прикл. механика.~--- 1989.~--- №20.~--- С.~3--9.

\bibitem{Gordeev}
Гордеев, А.~В. 
Моделирование свойств композиционного материала, армированного короткими волокнами [Текст] 
/ А.~В.~Гордеев 
// Механика композиционных материалов и конструкций.~--- М.: ИПМ РАН.~--- 2010.~--- Т.~16, №1.~--- С.~106--116.

\bibitem{Grigoliuk}
Григолюк, Э.~И. 
Перфорированные пластины и оболочки [Текст] 
/ Э.~И.~Григолюк, Л.~А.~Фильштинский.~--- М.: Наука, 1970.~--- 556~с.

\bibitem{Grinchenko1965}
Гринченко, В.~Т. 
Осесимметричная задача теории упругости для полубесконечного кругового цилиндра [Текст] 
/ В.~Т.~Гринченко 
// Прикл. механика.~--- 1965.~--- Т.~1, №1.~--- С.~109--119.

\bibitem{Grinchenko1967}
Гринченко, В.~Т. 
Осесимметричная задача теории упругости для толстостенного цилиндра конечной длины [Текст] 
/ В.~Т.~Гринченко 
// Прикл. механика.~--- 1967.~--- Т.~3, №8.~--- С.~93--103.

\bibitem{Grinchenko1985}
Гринченко, В.~Т. 
Пространственные задачи теории упругости и пластичности [Текст] 
/ В.~Т.~Гринченко, А.~Ф.~Улитко.~---  К.: Наук. думка, 1985.~--- Т.~3: Равновесие упругих тел канонической формы.~--- 280~с.

\bibitem{Grinchenko1978}
Гринченко, В.~Т. 
Равновесие и установившиеся колебания упругих тел конечных размеров [Текст] 
/ В.~Т.~Гринченко.~--- К.: Наук. думка, 1978.~--- 264~с.

\bibitem{Guz1968}
Гузь, А.~Н. 
О решении двумерных и трехмерных задач механики сплошной среды для многосвязных областей [Текст] 
/ А.~Н.~Гузь 
// Концентрация напряжений.~--- К.: Наук. думка.~--- 1968.~--- Вып.~2.~--- С.~54--58.

\bibitem{Guz1984}
Гузь, А.~Н. 
Пространственные задачи теории упругости и пластичности [Текст] 
/ А.~Н.~Гузь, Ю.~Н.~Немиш.~--- К.: Наук. думка, 1984.~--- Т.~2: Статика упругих тел неканонической формы.~--- 280~с.

\bibitem{Dudchenko}
Дудченко, А.~А. 
Структурная модель межфазного слоя для наполненных композиционных материалов [Текст] 
/ А.~А.~Дудченко, С.~А.~Лурье, Н.~П.~Шумова 
// Конструкции из композиционных материалов.~--- 2006.~--- №3.~--- С.~3--11.

\bibitem{Diskin}
Дыскин, А.~В. 
К расчету эффективных деформационных характеристик материала с трещинами  
/ А.~В.~Дыскин 
// Изв. АН СССР. МТТ.~--- 1985.~--- №4.~--- С.~130--135.

\bibitem{Erzhanov}
Ержанов, Ж.~С. Метод конечных элементов в задачах механики горных пород [Текст] 
/ Ж.~С.~Ержанов, Т.~Д.~Каримбаев.~--- Алма-Ата: Наука, 1975.~--- 238~с.

\bibitem{Zharkov}
Исследование напряженно-деформированного состояния дисперсно наполненного полимерного композита с использованием объемных моделей [Текст] 
/ А.~С.~Жарков, И.~И.~Анисимов, А.~В.~Шемелинин и др. 
// Механика композиционных материалов и конструкций.~--- 2012.~--- Т.~18, №1.~--- С.~16--34.

\bibitem{Kanaun}
Канаун, С.~К. 
Пуассоновское множество трещин в упругой сплошной среде  
/ С.~К.~Канаун 
// ПММ.~--- 1980.~--- Т.~44, №6.~--- С.~1129--1139.

\bibitem{Kantorovich}
Канторович, Л.~В. 
Функциональный анализ [Текст] / Л.~В.~Канторович, Г.~П.~Акилов.~--- М.: Наука, 1977.~--- 742~с.

\bibitem{Kapshiviy}
Капшивый, А.~А. 
Осесимметричное напряжённое состояние шара с неконцентрической шаровой полостью [Текст] 
/ А.~А.~Капшивый, Н.~П.~Копыстра, Л.~Н.~Ломонос 
// Докл. АН УССР. Сер.~А.~--- 1980.~--- №9.~--- С.~50--55.

\bibitem{Karimbaev}
Каримбаев, Т.~Д. 
Подходы при моделировании деформаций композиционных материалов [Текст] / Т.~Д.~Каримбаев 
// Космонавтика и ракетостроение.~--- 2009.~--- Т.~54, №1.~--- С.~103--122.

\bibitem{Kaufman}
Кауфман, Р.~Н. 
Сжатие упругого шара с неконцентрической шаровой полостью [Текст] 
/ Р.~Н.~Кауфман 
// Прикл. математика и механика.~--- 1964.~--- Т.~28.~--- Вып.~4.~--- С.~787--790.

\bibitem{Kit}
Кит, Г.~С. 
Метод потенциалов в трёхмерных задачах термоупругости тел с трещинами [Текст] 
/ Г.~С.~Кит, М.~В.~Хай.~--- К.: Наук. думка, 1989.~--- 288~с.

\bibitem{Kolesov}
Концентрация напряжений в упругом шаре с неконцентрической сферической полостью [Текст] 
/ В.~С.~Колесов, Н.~М.~Власов, Л.~О.~Тисовский, И.~П.~Шацкий 
// Мат. методы и физ.-мех. поля.~--- 1989.~--- №30.~--- С.~37--41.

\bibitem{Brautman}
Композиционные материалы 
/ под ред. Л.~Браутмана, Р.~Крока.~--- В 8 т.~--- М.: Мир, 1978.~--- Т.~2: Механика композиционных материалов.~--- 566~с.

\bibitem{Kristensen}
Кристенсен, Р. 
Введение в механику композитов [Текст] 
/ Р.~Кристенсен.~--- М.: Мир, 1982.~--- 334~с.

\bibitem{Kupradze}
Трехмерные задачи математической теории упругости [Текст] 
/ В.~Д.~Купрадзе, Т.~Г.~Гегелия, М.~О.~Башелейшвили, Т.~В.~Бурчуладзе~--- М.: Наука, 1976.~--- 664~с.

\bibitem{Kusch1995}
Кущ, В.~И. 
Напряжённое состояние и эффективные упругие модули среды, нормированной периодически расположенными сфероидальными включениями [Текст] 
/ В.~И.~Кущ 
// Прикл. механика.~--- 1995.~--- Т.~31, №3.~--- С.~32--39.

\bibitem{Lebedev}
Лебедев, Н.~Н. 
Специальные функции и их приложения [Текст] 
/ Н.~Н.~Лебедев
// М.: Физ.-мат. лит., 1963.~--- 358~с.

\bibitem{Levin}
Левин, В.~М. 
К определению эффективных упругих модулей композиционных материалов [Текст] 
/ В.~М.~Левин 
// ДАН СССР. Сер. мат.-физ., 1975.~--- Т.~220, №5.~--- С.~1042--1054.

\bibitem{Lekhnitskiy}
Лехницкий, С.~Г. 
Теория упругости анизотропного тела [Текст] 
/ С.~Г.~Лехницкий.~--- М.: Наука, 1977.~--- 416~с.

\bibitem{Lomonos}
Ломонос, Л.~Н. 
Первая основная задача об осесимметричном напряженном состоянии пространства с двумя сферическими полостями [Текст] 
/ Л.~Н.~Ломонос 
// Мат. физика и нелинейная механика.~--- 1990.~--- №13.~--- С.~51--56.

\bibitem{Lur'e}
Лурье, А.~И. 
Пространственные задачи теории упругости [Текст] 
/ А.~И.~Лурье.~--- М.: Гостехиздат, 1955.~--- 492~с.

\bibitem{Method}
Метод граничных интегральных уравнений. Вычислительные аспекты и приложения в механике [Текст].~--- М.: Мир, 1978.~--- 210~с.

\bibitem{Vanin1994}
Моделирование процессов деформирования и разрушения многоуровневых композитных материалов при высоких градиентных полях 
/ Г.~А.~Ванин, А.~В.~Березин, В.~С.~Добрынин и др. 
// НИР/НИОКР РФФИ: 94-01-00523-а.~--- 1994.

%**************************************************************************************

\bibitem{Nikolaev1984}
Николаев, А.~Г. 
Формулы переразложения векторных решений уравнения Ламе в сферической и сфероидальной системах координат [Текст] 
/ А.~Г.~Николаев 
// Мат. методы анализа динамических систем.~--- Х.: ХАИ.~--- 1984.~--- Вып.~8.~--- С.~100--104.

\bibitem{Nikolaev1993}
Николаев, А.~Г. 
Теоремы сложения решений уравнения Ламе.~--- Х.: Харьк. авиац. ин-т, 1993.~--- 109~с.~--- Деп. в ГНТБ Украины 21.06.93, №1178~--- Ук~93.

\bibitem{Nikolaev1998-1}
Николаев, А.~Г. 
Интегральные представления гармонических функций и теоремы сложения [Текст] 
/ А. Г. Николаев // Доп. НАН України.~--- 1998. №4.~--- С.~36--40.

\bibitem{Nikolaev1998}
Николаев, А.~Г. 
Обоснование метода Фурье в основных краевых задачах теории упругости для некоторых пространственных канонических областей [Текст] 
/ А.~Г.~Николаев 
// Доп. НАН України.~--- 1998.~--- №2.~--- С.~78--83.

\bibitem{Nikolaev2003}
Николаев, А.~Г. 
Температурные напряжения в упругом пространстве, содержащем периодическую систему упругих шаровых включений [Текст] 
/ А.~Г.~Николаев, С.~С.~Куреннов 
// Теор. и прикл. механика.~--- 2003.~--- Вып.~37.~--- С.~37--41.

\bibitem{Nikolaev2004}
Николаев, А.~Г. 
Термоупругие напряжения в пространстве с периодически расположенными упругими шаровыми включениями [Текст] 
/ А.~Г.~Николаев, С.~С.~Куреннов 
// Проблемы машиностроения.~--- 2004.~--- №1.~--- С.~35--48.

\bibitem{Nikolaev2011}
Николаев, А.~Г. 
Обобщенный метод Фурье в пространственных задачах теории упругости [Текст]: монография 
/ А.~Г.~Николаев, В.~С.~Проценко.~--- Х.: Нац. аэрокосм. ун-т им.~Н.~Е.~Жуковского ``Харьк. авиац. ин-т'', 2011.~--- 344~с. 

\bibitem{Nikolaev2009}
Николаев, А.~Г. 
Математическая модель напряженно-деформированного состояния пористого материала [Текст] 
/ А.~Г.~Николаев, Е.~А.~Танчик 
// Вопросы проектирования и производства конструкций летательных аппаратов: сб. науч. тр. Нац. аэрокосм. ун-та им.~Н.~Е.~Жуковского ``ХАИ''.~--- Вып.~2(58).~--- Х., 2009.~--- С.~48--58.

\bibitem{Nikolaev2010}
Николаев, А.~Г. 
Локальная математическая модель зернистого композиционного материала [Текст] 
/ А.~Г.~Николаев, Е.~А.~Танчик 
// Вестн. Харьк. Нац. ун-та им.~В.~Н.~Каразина. Сер. Математика, прикладная математика и механика.~--- 2010.~--- Т.~922.~--- С.~4--19.

\bibitem{Nikolaev2012}
Николаев, А.~Г. 
Распределение напряжений в упругом пространстве с двумя параллельно расположенными сферическими полостями [Текст] 
/ А.~Г.~Николаев, Е.~А.~Танчик 
// Вопросы проектирования и производства конструкций летательных аппаратов: сб. науч. тр. Нац. аэрокосм. ун-та им.~Н.~Е.~Жуковского ``ХАИ''.~--- Вып.~4(72).~--- Х., 2012.~--- С.~92--99.

\bibitem{Nikolaev-Tanchik2012}
Николаев, А.~Г. 
Трехмерная периодическая модель зернистого композиционного материала [Текст] 
/ А.~Г.~Николаев, Е.~А.~Танчик 
// Методы решения прикладных задач механики деформируемого твердого тела: сб. науч. тр. Днепропетр. нац. ун-та им.~О.~Гончара.~--- Дп.: Лира.~--- 2012.~--- Вып.~13.~--- С.~287--293.

\bibitem{Nikolaev2013}
Николаев, А.~Г. 
Развитие локальной модели напряженного состояния пористого материала [Текст] 
/ А.~Г.~Николаев, Е.~А.~Танчик 
// Авиационно-космическая техника и технология.~--- 2013.~--- №1(98).~--- C.~14--18.

\bibitem{Nikolaev2013-1}
Николаев, А.~Г. 
Распределение напряжений в цилиндрическом образце материала с двумя параллельными цилиндрическими полостями [Текст] 
/ А.~Г.~Николаев, Е.~А.~Танчик 
// Вопросы проектирования и производства конструкций летательных аппаратов: сб. науч. тр. Нац. аэрокосм. ун-та им.~Н.~Е.~Жуковского ``ХАИ''.~--- Вып. 4(76).~--- Х., 2013.~--- С.~40--49.

\bibitem{Nikolaev2013-2}
Николаев, А.~Г. 
Напряженное состояние в цилиндрическом образце с двумя параллельными цилиндрическими волокнами [Текст] 
/ А.~Г.~Николаев, Е.~А.~Танчик 
// Авиационно-космическая техника и технология.~--- 2013.~--- №6(103).~--- С.~32--38.

\bibitem{Nikolaev2013-19}
Николаев, А.~Г. 
Хрупкое разрушение цилиндрического стержня с круговой трещиной при кручении [Текст] 
/ А.~Г.~Николаев, Е.~А.~Танчик, И.~С.~Тарасевич 
// Вопросы проектирования и производства конструкций летательных аппаратов: сб. науч. тр. Нац. аэрокосм. ун-та им.~Н.~Е.~Жуковского ``ХАИ''.~--- Вып.~2(74).~--- Х., 2013.~--- С.~64--73.

\bibitem{Nikolaev2013-4}
Николаев, А.~Г. 
Распределение напряжений в ячейке однонаправленного композиционного материала, образованного четырьмя цилиндрическими волокнами [Текст] / А.~Г.~Николаев, Е.~А.~Танчик 
// Вісник Одес. нац. ун-ту. Математика і механіка.~--- 2013.~--- Т.~18.~--- Вип.~4(20).~--- С.~64--73.

\bibitem{Nikolaev2014-1}
Николаев, А.~Г. 
Новые теоремы сложения базисных решений уравнения Ламе для вытянутых сфероидов и их применение к моделированию пористого материала [Текст] 
/ А.~Г.~Николаев, Е.~А.~Танчик 
// Авиационно-космическая техника и технология.~--- 2014.~--- №5(112).~--- С.~46--54.

\bibitem{Nikolaev2014-9}
Николаев, А.~Г. 
Развитие аппарата обобщенного метода Фурье на некоторые многосвязные области и его использование для моделирования пористого материала [Текст] 
/ А.~Г.~Николаев, Е.~А.~Танчик 
// Авиационно-космическая техника и технология.~--- 2014.~--- №6(113).~--- С.~48--56.

\bibitem{Nikolaev2013-3}
Ніколаєв, О.~Г. 
Напруження в нескінченному круговому циліндрі з чотирма циліндричними порожнинами [Текст] 
/ О.~Г.~Ніколаєв, Є.~А.~Танчік 
// Математичні методи та фізико-механічні поля.~--- 2014.~--- Т.~57, №3.~--- С.~51--60.

\bibitem{Nikolaev2014-2}
Николаев, А.~Г. 
Анализ напряженного состояния в окрестности двух сферических включений в упругом пространстве [Текст] 
/ А.~Г.~Николаев, Е.~А.~Танчик 
// Авиационно-космическая техника и технология.~--- 2014.~--- №3(110).~--- С.~26--32.

\bibitem{Nikolaev2014-3}
Николаев, А.~Г. 
Упругое пространство с четырьмя сфероидальными включениями под действием внешней нагрузки [Текст] 
/ А.~Г.~Николаев, Е.~А.~Танчик 
// Авиационно-космическая техника и технология.~--- 2014.~--- №4(111).~--- С.~49--55.

\bibitem{Nikolaev2014-4}
Николаев, А.~Г. 
Напряженное состояние пористого материала в области между четырьмя сфероидальными порами [Текст] 
/ А.~Г.~Николаев, Е.~А.~Танчик 
// Вісник нац. техн. ун-ту ``ХПІ''. Математичне моделювання в техніці і технологіях.~--- 2014.~--- №6(1049).~--- С.~151--160.

\bibitem{Nikolaev2014-5}
Николаев, А.~Г. 
Модель зернистого композита со сферическими зернами [Текст] 
/ А.~Г.~Николаев, Е.~А.~Танчик 
// Вісник нац. техн. ун-ту ``ХПІ''. Математичне моделювання в техніці і технологіях.~--- 2014.~--- №39(1082).~--- С.~141--152.

\bibitem{Nikolaev2014-6}
Николаев, А.~Г. 
Напряженное состояние в окрестности двух сфероидальных зерен в композите [Текст] 
/ А.~Г.~Николаев, Е.~А.~Танчик 
// Вопросы проектирования и производства конструкций летательных аппаратов: сб. науч. тр. Нац. аэрокосм. ун-та им.~Н.~Е.~Жуковского ``ХАИ''.~--- Вып.~1(77).~--- Х., 2014.~--- С.~73--86.

\bibitem{Nikolaev2014-7}
Николаев, А.~Г. 
Напряжения в упругом материале со сферическими порами под действием внешней нагрузки [Текст] 
/ А.~Г.~Николаев, Е.~А.~Танчик 
// Вопросы проектирования и производства конструкций летательных аппаратов: сб. науч. тр. Нац. аэрокосм. ун-та им.~Н.~Е.~Жуковского ``ХАИ''.~--- Вып.~2(78).~--- Х., 2014.~--- С.~99--110.

\bibitem{Nikolaev2015-1}
Николаев, А.~Г. 
Первая краевая задача теории упругости для цилиндра с N цилиндрическими полостями [Текст] 
/ А.~Г.~Николаев, Е.~А.~Танчик 
// Сиб. журн. вычисл. математики, РАН. Сиб. отд-ние.~--- Новосибирск.~--- 2015.~--- Т.~18, №2.~--- С.~177--188.

\bibitem{Nikolaev2007}
Николаев, А.~Г. 
Напряженное состояние  трансверсального изотропного пространства с двумя сфероидальными полостями [Текст] 
/ А.~Г.~Николаев, Ю.~А.~Щербакова 
// Вопросы проектирования и производства конструкций летательных аппаратов: cб. науч. тр. Нац. аэрокосм. ун-та им.~Н.~Е.~Жуковского ``ХАИ''.~--- Вып.~4(51).~--- Х., 2007.~--- С.~49--54.

%\bibitem{Nikolaev2013-6}
%Николаев, А.~Г. 
%Напряжения в упругом цилиндре с шестью цилиндрическими полостями, образующими гексагональную структуру [Текст] 
%/ А.~Г.~Николаев, Е.~А.~Танчик 
%// Прикладная механика и техническая физика (направлено до друку).

%\bibitem{Nikolaev2013-7}
%Николаев, А.~Г. 
%Модель напряженного состояния однонаправленного композита, цилиндрические волокна которого образуют тетрагональную структуру 
%/ А.~Г.~Николаев, Е.~А.~Танчик 
%// Механика композитных материалов (г.~Рига) (направлено до друку).

%\bibitem{Nikolaev2013-8}
%Николаев, А.~Г. 
%Модель упругого состояния составного цилиндра, цилиндрические волокна которого образуют гексагональную структуру 
%/ А.~Г.~Николаев, Е.~А.~Танчик 
%// Механика композиционных материалов и конструкций (г.~Москва) (направлено до друку).

%\bibitem{Nikolaev2013-10}
%Ніколаєв, О.~Г. 
%Пружний стан простору з двома циліндричними включеннями [Текст] 
%/ О.~Г.~Ніколаєв, Є.~А.~Танчік 
%// Сучасні проблеми механіки та математики.~--- Львів, 2013.~--- С.~196--198.

%\bibitem{Nikolaev2013-11}
%Ніколаєв, О.~Г. 
%Напружений стан у пористому матеріалі в області між двома стиснутими сфероїдальними порами [Текст] 
%/ О.~Г.~Ніколаєв, Є.~А.~Танчік 
%// Сучасні науково-методичні проблеми математики у вищій школі.~--- Київ, 2013.~--- С.~26--28.

%\bibitem{Nikolaev2013-12}
%Николаев, А.~Г. 
%Распределение напряжений в ячейке композиционного материала, образованной четырьмя параллельными цилиндрическими включениями [Текст] 
%/ А.~Г.~Николаев, Е.~А.~Танчик 
%// Сучасні проблеми механіки деформівного твердого тіла, диференціальних та інтегральних рівнянь.~--- Одеса, 2013.~--- С.~95--96.

%\bibitem{Nikolaev2013-13}
%Николаев, А.~Г. 
%Модель напряженного состояния однонаправленного волокнистого композита [Текст] 
%/ А.~Г.~Николаев, Е.~А.~Танчик 
%// Тараповские чтения.~--- Харьков, 2013.

%\bibitem{Nikolaev2013-14}
%Николаев, А.~Г. 
%Напряженное состояние в изотропном материале с четырьмя цилиндрическими порами [Текст] 
%/ А.~Г.~Николаев, Е.~А.~Танчик 
%// ИКТМ.~--- Харьков, 2013.~--- С.~173.

%\bibitem{Nikolaev2013-15}
%Николаев, А.~Г. 
%Распределение напряжений в кубической ячейке зернистого композита [Текст] 
%/ А.~Г.~Николаев, Е.~А.~Танчик 
%// ИКТМ.~--- Харьков, 2013.~--- С.~187.

\bibitem{Olegin}
Олегин, И.~П. 
Пространственное напряженное состояние тела, содержащего периодическую систему сферических полостей 
/ И.~П.~Олегин 
// Динамика и прочность авиац. конструкций.~--- Новосибирск.~--- 1989.~--- С.~85--91.

\bibitem{Pobedrya}
Победря, Б.~Е. 
Механика композиционных материалов [Текст] 
/ Б.~Е.~Победря.~--- М.: МГУ, 1984.~--- 336~с.

\bibitem{Podilchuk1967}
Подильчук, Ю.~Н. 
Деформация упругого сфероида 
/ Ю.~Н.~Подильчук 
// Прикл. механика.~--- 1967.~--- Т.~3, №12.~--- С.~34--42.

\bibitem{Podilchuk1984}
Подильчук, Ю.~Н. 
Пространственные задачи теории упругости и пластичности 
/ Ю.~Н.~Подильчук~--- Т.~1. Граничные задачи статики упругих тел.~--- К.: Наук. думка, 1984.~--- 304~с.

\bibitem{Prokopov}
Прокопов, В.~К. 
Осесимметричная задача теории упругости для изотропного цилиндра 
/ В.~К.~Прокопов 
// Тр. Ленингр. политехн. ин-та.~--- 1950.~--- №2.~--- С.~286--304.

\bibitem{Savin}
Савин, Г.~Н. 
Распределение напряжений около отверстий 
/ Г.~Н.~Савин.~--- К.: Наук. думка, 1968.~--- 888~с.

\bibitem{Salganik}
Салганик, Р.~Л. 
Механика тел с большим числом трещин  
/ Р.~Л.~Салганик 
// Изв. АН СССР. МТТ.~--- 1973.~--- №4.~--- С.~149--158.

\bibitem{Sendetski}
Сендецки, Дж. 
Упругие свойства композитов [Текст] 
/ Дж.~Сендецки 
// Механика композиционных материалов: в 8~т.~--- М.: Мир, 1978.~--- Т.~2.~--- С.~61--101.

\bibitem{Skudra}
Скудра, А.~М. 
Структурная теория армированных пластиков [Текст] 
/ А.~М.~Скудра, Ф.~Я.~Булавс.~--- Рига: Зинатне, 1978.~--- 192~с.

\bibitem{Smirnov}
Смирнов, Л.~Г. 
Упругие напряжения в сфере с инородным эксцентрическим сферическим включением 
/ Л.~Г.~Смирнов,  И.~И.~Федик 
// Мат. методы и физ.-мех. поля.~--- 1990.~--- №31.~--- С.~79--83.

%\bibitem{Tanchik}
%Танчик, Е.~А. 
%Распределение напряжений на сфероидальной полости вблизи неосесимметрично расположенных сфероидальных полостей [Текст] 
%/ Е.~А.~Танчик 
%// ИКТМ’2012~--- Тезисы докладов~--- С.~205.

\bibitem{Tokovyy}
Токовий, Ю.~В. 
Осесиметричні напруження в скінченному пружному циліндрі під дією нормального тиску, рівномірно розподіленого по частині бічної поверхні [Текст] 
/ Ю.~В.~Токовий 
// Прикл. проблеми мех. та мат.~--- 2010.~--- Вип.~8.~--- С.~144--151.

\bibitem{Ulitko}
Улитко, А~.Ф. 
Метод собственных векторных функций в пространственных задачах теории упругости 
/ А.~Ф.~Улитко.~--- К.: Наук. думка, 1979.~--- 265~с.

\bibitem{Ustinov}
Устинов, К.~Б. 
Об определении эффективных упругих характеристик двухфазных сред. Случай изолированных неоднородностей в форме эллипсоидов вращения  
/ К.~Б.~Устинов 
// Успехи механики.~--- 2003.~--- №2.~--- С.~126--168.

\bibitem{Fedotov}
Федотов, А.~Ф. 
Приложение модели деформирования пористых материалов к расчёту эффективных упругих модулей зернистых композитов [Текст] 
/ А.~Ф.~Федотов 
// Механика композиционных материалов и конструкций.~--- 2011.~--- Т.~17, №1.~--- С.~3--18.

\bibitem{Fudzii}
Фудзии, Т. 
Механика разрушения композиционных материалов [Текст] 
/ Т.~Фудзии, М.~Дзако.~--- М.: Мир, 1982.~--- 232~с.

\bibitem{Khoroshun}
Хорошун, Л.~П. 
Зернистые материалы [Текст] 
/ Л.~П.~Хорошун, Б.~П.~Маслов.~--- Механика композитных материалов и элементов конструкций: в 3 т.~--- К.: Наук. думка, 1982.~--- С.~191--284.

\bibitem{Chernous}
Методы расчета механических характеристик пороматериалов малой плотности (обзор) 
/ Д.~А.~Черноус, Е.~М.~Петроковец, Д.~А.~Конек, С.~В.~Шилько 
// Механика композиционных материалов и конструкций.~--- М.: ИПМ РАН.~--- 2001.~--- Т.~7, №4.~--- С.~533--545.

\bibitem{Shailiev}
Шайлиев, Р.~Ш. 
Математическая модель расчета эффективных свойств композиционных материалов на примере полиминеральных горных пород [Текст] 
/ Р.~Ш.~Шайлиев 
// Электронный науч. журнал ``Современные проблемы науки и образования''.~--- 2011.~--- №5.

\bibitem{Schermergor}
Шермергор, Т.~Д. 
Теория упругости микронеоднородных сред [Текст] 
/ Т.~Д.~Шермергор.~--- М.: Наука, 1977.~--- 400~с.

\bibitem{Eshelbi}
Эшелби, Дж. 
Континуальная теория дислокаций [Текст] 
/ Дж.~Эшелби.~--- М.: Мир, 1963.~--- 247~с.

\bibitem{Yankovskiy}
Янковский, А.~П. 
Моделирование механического поведения композитов с пространственной структурой армирования из нелинейно-наследственных материалов [Текст] 
/ А.~П.~Янковский 
// Конструкции из композиционных материалов.~--- 2012.~--- №2.~--- С.~12--25.

\bibitem{Atsumi}
Atsumi, A. 
Stresses in a transversely isotropic half space having a spherical cavity 
/ A.~Atsumi, S.~Iton 
// Trans. ASME. E.~--- 1974.~--- V.~41, №3.~--- P.~708--712.

\bibitem{Boucher}
Boucher, S. 
On the effective moduli of isotropic two-phase elastic composites 
/ S.~Boucher 
// J. Comp. Mater.~--- 1974.~--- V.~8.~--- P.~82--90.

\bibitem{Budiansky}
Budiansky, B. 
Elastic moduli of a cracked solid 
/ B.~Budiansky, R.~J.~O'Connell 
// Int. J. Solids Struct.~--- 1976.~--- V.~12.~--- P.~81--97.

\bibitem{Chen1978-1}
Chen, H.-S. 
The effective elastic moduli of composite materials containing spherical inclusions at non-dilute concentrations 
/ H.-S.~Chen, A.~Acrivos 
// Int. J. Solids and Structures.~--- 1978.~--- V.~14.~--- P.~349--360.

\bibitem{Chen1978-2}
Chen, H.-S. 
The solution of the equations of linear elasticity for an infinite region containing two spherical inclusions 
/ H.-S.~Chen, A.~Acrivos 
// Int. J. Solids and Structures.~--- 1978.~--- V.~14.~--- P.~331--348.

\bibitem{Christensen}
Christensen, R.~M. 
Solutions for effective shear properties in three phase sphere and cylinder models 
/ R.~M.~Christensen, R.~H.~Lo 
// J. Mech. and Phys. Solids, 1979.~--- V.~27, №4.

\bibitem{Christensen1990}
Christensen, R.~M. 
A critical evaluation of a class of micro-mechanics models 
/ R.~M.~Christensen 
// J. Mech. and Phys. Solids, 1990.~--- V.~38.~--- P.~379--404.

\bibitem{Edwards}
Edwards, R.~H. 
Stress concentrations around spheroidal inclusions and cavities 
/ R.~H.~Edwards 
// J. Appl. Mech.~--- 1951.~--- V.~18, №1.~--- P.~13--35.

\bibitem{Eshelby}
Eshelby, J.~D. 
The determination of the elastic field of an ellipsoidal inclusion and related problems  
/ J.~D.~Eshelby 
// Proc. R. Soc. London. Ser. A.~--- 1957.~--- V.~241.~--- P.~376--396.

\bibitem{Golovchan2000}
Golovchan, V.~T. 
Double-particle approximation analysis of the residual thermostressed state of granular composites 
/ V.~T.~Golovchan, N.~V.~Litoshenko 
// International Applied Mechanics.~--- 2000.~--- V.~36, No~12.~--- P.~1612--1619.

\bibitem{Hashin}
Hashin, Z. 
The elastic moduli of fiber reinforced materials 
/ Z.~Hashin, W.~Rosen 
// J. Appl. Mech., 1964.~--- V.~31.~--- P.~223--232.

\bibitem{Hashin1983}
Hashin, Z. 
Analysis of composite materials~--- a survey 
/ Z.~Hashin 
// J. Appl. Mech., 1983.~--- V.~50.~--- P.~481--505.

\bibitem{Hashin1988}
Hashin, Z. 
The differential scheme and its application to cracked materials  
/ Z.~Hashin 
// J. Mech. Phys. Solids.~--- 1988.~--- V.~36.~--- P.~719--733.

\bibitem{Hashin1963}
Hashin, Z. 
A variational approach to the theory of the elastic behaviour of multiphase materials    
/ Z.~Hashin, S.~A.~Shtrikman 
// J. Mech. Phys. Solids.~--- 1963.~--- V.~11.~--- P.~128--140.

\bibitem{Kachanov}
Kachanov, M. 
Effective elastic properties of cracked solids: critical review of some basic concepts   
/ M.~Kachanov 
// Appl. Mech. Rev.~--- 1992.~--- V.~45.~--- P.~304--335.

\bibitem{Khoroshun2000-1}
Khoroshun, L.~P. 
Theory of short-term micro damageability of granular composite materials
/ L.~P.~Khoroshun, E.~N.~Shikula 
// International Applied Mechanics.~--- 2000.~--- V.~36, No~8.~--- P.~1060--1066.

\bibitem{Khoroshun2000-2}
Khoroshun, L.~P. 
Mathematical models and methods of the mechanics of stochastic composites 
/ L.~P.~Khoroshun 
// International Applied Mechanics.~--- 2000.~--- V.~36, No~10.~--- P.~1284--1316.

\bibitem{Meleshko}
Meleshko, V.~V. 
Equilibrium of an elastic finite cylinder under axisymmetric discontinuous normal loading 
/ V.~V.~Meleshko, Yu.~V.~Tokovyy 
// J.~Eng.~Math.~--- 2013.~--- V.~78.~--- P.~143--166.  

\bibitem{Miyamoto}
Miyamoto, H. 
On the problem of the theory of elasticity for a region, containing more than two spherical cavities 
/ H.~Miyamoto 
// Bull. JSME.~--- 1958.~--- V.~1, №2.~--- P.~103--115.

\bibitem{Mura}
Mura, T. 
Two-ellipsoidal inhomogeneities by the equivalent inclusion method 
/ T.~Mura, Z.~A.~Moschovidis 
// Trans. ASME. E.~--- 1975.~--- V.~42, №4.~--- P.~847--852.

\bibitem{Nemat-Nasser}
Nemat-Nasser, S. 
On effective moduli of an elastic body containing periodically distributed voids: comments and corrections  
/ S.~Nemat-Nasser, M.~Taya 
// Quart. Appl. Math.~--- 1985.~--- V.~43.~--- P.~187--188.

\bibitem{Nikolaev2014-8}
Nikolaev, A.~G. 
On the distribution of stresses in circular infinite cylinder with cylindrical cavities 
/ A.~G.~Nikolaev, E.~A.~Tanchik 
// Visn. Khark. Nat. Univ., Ser. Mat. Prykl. Mat. Mekh.~--- 2014.~--- V.~1120, Issue~69.~--- P.~4--19.

%\bibitem{Nikolaev2013-9}
%Nikolaev, A.~G. 
%Stress state of infinite elastic cylinder with cylindrical cavities under axisymmetric piecewise constant normal load 
%/ A.~G.~Nikolaev, E.~A.~Tanchik 
%// Journal for Engineering Mathematics (направлено до друку).

%\bibitem{Nikolaev2013-16}
%Nikolaev, O.~G. 
%The stress state in the matrix of granular composite in domain between two oblate-spheroidal grains [Text] 
%/ O.~G.~Nikolaev, E.~A.~Tanchik 
%// DSMSI.~--- Kiev, 2013.~--- P.~300.

%\bibitem{Nikolaev2013-17}
%Nikolaev, A.~G. 
%Stress distribution in the centered cell of unidirectional fiber composite [Text] 
%/ A.~G.~Nikolaev, E.~A.~Tanchik 
%// ICTM.~--- Kharkov, 2013.~--- P.~169.

%\bibitem{Nikolaev2013-18}
%Nikolaev, A.~G. 
%Stress state in a cubic cell of the porous material with prolate-spheroidal cavities [Text] 
%/ A.~G.~Nikolaev, E.~A.~Tanchik 
%// ICTM.~--- Kharkov, 2013.~--- P.~177.

\bibitem{Sangini}
Sangini, A.~S. 
Elastic coefficients of composites containing sperical inclusions in a periodic array  
/ A.~S.~Sangini, W.~Lu 
// J. Mech. Phys. Solids.~--- 1987.~--- V.~35, №1.~--- P.~1--21.

\bibitem{Sheikh}
Sheikh, M.~A. 
Microstructural finite-element modeling of a ceramic matrix composite to predict experimental measurements of its macro thermal properties 
/ M.~A.~Sheikh, S.~C.~Taylor, D.~R.~Hayhurst, R.~Taylor 
// Modeling and Simulation in Materials Science and Engineering.~--- 2001.~--- V.~9, No~1.~--- P.~7--23.

\bibitem{Scalon}
Scalon, J. 
A model-based analysis of particle size distributions in composite materials 
/ J.~Scalon, N.~R.~J.~Fieller, E.~C.~Stillman, H.~V.~Atkinson 
// Acta Materialia.~--- 2003.~--- V.~51, No~4.~--- P.~997--1006.

\bibitem{Strenberg}
Strenberg, E. 
On the axisymmetric problem of elasticity for an infinite region containing two spherical cavities 
/ E.~Strenberg, M.~A.~Sadowsky 
// Trans. ASME. J. Appl. Mech.~--- 1952.~--- V.~74.~--- P.~19--27.

\bibitem{Torquato}
Torquato, S. 
Random heterogeneous media: microstructure and improved bounds on effective properties  
/ S.~Torquato 
// Appl. Mech. Rev.~--- 1991.~--- V.~44.~--- P.~37--76.

\bibitem{Trias}
Trias, D. 
Random models versus periodic models for fibre reinforced composites 
/ D.~Trias, J.~Costa, J.~A.~Mayugo, J.~E.~Hurtado 
// Computational Materials Science.~--- 2006.~--- V.~38, No~2.~--- P.~316--324.

\bibitem{Tsuchida1980}
Tsuchida, E. On the asymmetric problem of elasticity theory for an infinite elastic solid containing two spherical inclusions 
/ E.~Tsuchida, I.~Nakahara, M.~Kodama 
// Bull. JSME.~--- 1980.~--- V.~23, №181.~--- P.~1072--1080.

\bibitem{Tsuchida1979}
Tsuchida, E. 
On the asymmetric problem of elasticity theory for an infinite elastic solid containing some spherical cavities 
/ E.~Tsuchida, N.~Uchiyama, I.~Nakahara, M.~Kodama 
// Bull. JSME.~--- 1979.~--- V.~22, №164.~--- P.~141--147.

\bibitem{Zhong}
Zhong, Z. 
Analysis of a transversely isotropic rod containing a single cylindrical inclusion with axisymmetric eigenstrains 
/ Z.~Zhong, Q.~P.~Sun 
// Int. Journal of Solids and Structures.~--- 2002.~--- V.~39, Issue~23.~--- P.~5753--5765.

\bibitem{Zimmerman}
Zimmerman, R.~W. 
Behaviour of the Poisson ratio of a two-phase conposite materials in the high-concentration limit  
/ R.~W.~Zimmerman 
// Appl. Mech. Rev.~--- 1994.~--- V.~47.~--- P.~38--44.

\bibitem{Zureick1989}
Zureick, A.~H. 

The asymmetric displacement of a rigid spheroidal inclusion embedded in transversely isotropic medium 
/ A.~H.~Zureick 
// Acta. mech.~--- 1989.~--- V.~77, No~1-2.~--- P.~101--110.

\bibitem{Zureick1988}
Zureick, A.~H. 
Transversely isotropic medium with a rigid spheroidal inclusion under an axial pull 
/ A.~H.~Zureick 
// Trans. ASME. J. Appl. Mech.~--- 1988.~--- V.~55, №2.~--- P.~495--497.

\end{biblist}

\end{russian}