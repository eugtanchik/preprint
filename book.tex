%!TEX TS-program = lualatex
\documentclass[book,14pt,small,twoside]{ncc}
\usepackage[T2A]{fontenc}
\usepackage[utf8]{inputenc}
\usepackage{polyglossia}
\usepackage{fontspec}
\usepackage[left=18mm,right=18mm,top=25mm,bottom=20mm]{geometry}
\usepackage{nccmath}
\usepackage{amsfonts,amssymb}
\usepackage{hyperref}
\usepackage{nccfancyhdr}
\usepackage[section,above,below]{placeins}
\usepackage{afterpage,flafter}
\usepackage{lastpage}
\usepackage{morefloats}
\usepackage{cite}
\usepackage{float,nccsect}

\usepackage[square,numbers,sort&compress]{natbib}

\usepackage{xltxtra,xunicode}
\defaultfontfeatures{Scale=MatchUppercase,Ligatures=TeX}
\setromanfont[Numbers=Uppercase,Ligatures=TeX]{Arial}
\setmainlanguage{english}
\setotherlanguage{russian}

\renewtheorem{theorem}{Теорема}[theorem]

%\pagestyle{fancy}
%\renewcommand{\headrulewidth}{0pt}
%\lhead{}
%\rhead{\thepage}
%\chead{}
%\lfoot{}
%\rfoot{}
%\cfoot{}

\DeclareSection*{0}{chapter}{\bf\normalsize}%
{3.5ex plus 1ex minus .2ex}%
{2.3ex plus .2ex}{
\thispagestyle{fancy}
\bf\normalsize\MakeUppercase}

\DeclareSection*{1}{section}{}%
{3.5ex plus 1ex minus .2ex}%
{2.3ex plus .2ex}{\bf\normalsize}

%\newfloat{program}{tp}{lop}[chapter]
%\floatname{program}{Program}
%\RegisterFloatType{program}
%\DeclareSection{-3}{program}{\bff Табл.}{0pt}{0pt}{}
%\DeclareTOCEntry{-3}{}{}{9.9}{}

%\DeclareSection{-2}{table}{}{0pt}{10pt}{}
%\DeclareSection{-1}{figure}{\bff Рис.}{10pt}{0pt}{}
%\DeclareSection{-1}{theorem}{\bff Теорема}{10pt}{0pt}{}

%\NumberlineSuffix{.}{\enskip}

%\SectionTagSuffix{.\quad}

\setcounter{totalnumber}{10}
\setcounter{topnumber}{10}
\setcounter{page}{3}

\begin{document}

\mathversion{bold}
\sectionstyle{center}
\indentaftersection
\sectiontagsuffix{.\;}
%\captiontagsuffix{~---~}

\setyear{2014}
\titlehead{МИНИСТЕРСТВО ОБРАЗОВАНИЯ И НАУКИ УКРАИНЫ \\
Национальный аэрокосмический университет им.~Н.~Е.~Жуковского \\
``Харьковский авиационный институт''}
\titlefoot{Харьков ``ХАИ''\;\theyear}
\author{А.~Г.~Николаев, Е.~А.~Танчик}
\title{УПРУГАЯ МЕХАНИКА МНОГОКОМПОНЕНТНЫХ ТЕЛ}
\maketitle
\titlehead{}
\titlefoot{\theyear}
\author{А.~Г.~Николаев, Е.~А.~Танчик}
\title{УПРУГАЯ МЕХАНИКА МНОГОКОМПОНЕНТНЫХ ТЕЛ}
\maketitle
\abstractstyle{
\small
%\footnotesize
}
\bibindex{УДК 539.3\\ ББК 22.2\par\pbox[lt]{Н 63}}
\copyrighttable{ISBN 978-966-662-369-3}{\item Николаев~А.~Г., Танчик~Е.~А., \theyear \\ \item Национальный аэрокосмический университет \\ им.~Н.~Е.~Жуковского ``Харьковский авиационный институт'', \theyear}
\begin{abstract}
\begin{russian}
Уперше у вітчизняній і зарубіжній монографічній літературі досліджено проблему визначення полів напружень і деформацій у багатозв'язних просторових тілах з великим числом компонент зв'язності. Для її вирішення використано узагальнений метод Фур'є, який отримав подальший розвиток у цій монографії. Розглянуто тіла з неоднорідностями у вигляді однаково орієнтованих циліндричних, сферичних, витягнутих і стислих сфероїдальних порожнин або включень. Основну увагу приділено кількісному і якісному аналізу напружень в зонах їх найбільшої концентрації в зазначених багатокомпонентних тілах залежно від геометричних і механічних характеристик неоднорідностей і типу їхніх упаковок. Розглянуто також задачі з періодичними системами порожнин і включень. Отримані результати використано для визначення ефективних пружних модулів зернистих пористих і композиційних матеріалів.

Для науковців, викладачів, аспірантів, магістрантів, які цікавляться методами розв'язання крайових задач теорії пружності, механікою пористих і композиційних матеріалів.\sloppy
\qef
\begin{desclist}{}{}[Рецензенты:]
\item[Рецензенты:] чл.-корр. НАН Украины, д-р физ.-мат. наук, проф. А.~Н.~Довбня, \\
д-р физ.-мат. наук, проф. В.~А.~Дорошенко
\end{desclist}
\qef
Утверждена на заседании ученого совета Национального аэрокосмического университета им.~Н.~Е.~Жуковского ``ХАИ'' в качестве монографии (протокол №10 от 18.06.2014~г.){\sloppy\par}
\qef
%\begin{minipage}{0.05\linewidth}
%Н 63
%\end{minipage}
%\begin{minipage}{0.9\linewidth}
%\fulltitle{Николаев, А.~Г.}{\\ Упругая механика многокомпонентных тел [Текст]: монография / А.~Г.~Николаев, Е.~А.~Танчик.~--- Х.: Нац. аэрокосм. ун-т им.~Н.~Е.~Жуковского ``Харьк. авиац. ин-т'', 2014.~--- 272~с.}
%\end{minipage}
%
%ISBN 978-966-662-369-3
%\qef
\noindent\begin{tabular}[t]{lll}
 & {\bf\normalsize Николаев,~А.~Г.} & \\
Н 63 & {\normalsize Упругая механика многокомпонентных тел [Текст]: монография / } & \\ & {\normalsize А.~Г.~Николаев, Е.~А.~Танчик.~--- Х.: Нац. аэрокосм. ун-т им.~Н.~Е.~Жуков-} \\ & {\normalsize ского ``Харьк. авиац. ин-т'', 2014.~--- 272~с.} & \\
& & \\
& {\normalsize ISBN 978-966-662-369-3} &
\end{tabular}
\qef
%\begin{desclist}{}{}[Н 63]
%\item[\begin{tabular}{r}[b]{}\\ Н 63\end{tabular}] \fulltitle{Николаев~А.~Г.}{\\ Упругая механика многокомпонентных тел [Текст]: монография / А.~Г.~Николаев, Е.~А.~Танчик.~--- Х.: Нац. аэрокосм. ун-т им.~Н.~Е.~Жуковского ``Харьк. авиац. ин-т'', 2014.~--- 272~с.\\ \\
%ISBN 978-966-662-369-3}
%\end{desclist}
\par Впервые в отечественной и зарубежной монографической литературе исследована проблема определения полей напряжений и деформаций в многосвязных пространственных телах с большим числом компонент связности. Для ее решения использован обобщенный метод Фурье, который получил дальнейшее развитие в настоящей монографии. Рассмотрены тела с неоднородностями в виде одинаково ориентированных цилиндрических, сферических, вытянутых и сжатых сфероидальных полостей или включений. Основное внимание уделено количественному и качественному анализу напряжений в зонах их наибольшей концентрации в указанных многокомпонентных телах в зависимости от геометрических и механических характеристик неоднородностей и типа их упаковок. Рассмотрены также задачи с периодическими системами полостей и включений. Полученные результаты использованы для определения эффективных упругих модулей зернистых пористых и композиционных материалов.

Для научных работников, преподавателей, аспирантов, магистрантов, интересующихся методами решения краевых задач теории упругости, механикой пористых и композиционных материалов.
\qef
\begin{minipage}[t]{0.85\linewidth}
Ил. 392 Табл. 14. Библиогр.: 140 назв.
\end{minipage}
\begin{minipage}[t]{0.15\linewidth}
УДК 539.3\\ ББК 22.2
\end{minipage}
%Ил. 392 Табл. 14. Библиогр.: 140 назв.
\end{russian}
\end{abstract}
\author{Ніколаєв Олексій Георгійович \\ Танчік Євген Андрійович \vskip2cm}
\title{ПРУЖНА МЕХАНІКА БАГАТОКОМПОНЕНТНИХ ТІЛ}
\titlecomment{(Російською мовою)\par\qef Редактор Л.~О.~Кузьменко}
\lastpagehead{Наукове видання}
\lastpageinfo{
\begin{flushleft}
Зв. план 2014 \\
Підписано до друку 30.12.2014 \\
Формат 60х84 1/16. Папір офс. №2. Офс. друк \\
Ум. друк. арк. 14,5. Обл.-вид. арк. 16,36. Наклад 100 пр.
Замовлення 406. Ціна вільна 
\end{flushleft}
%Оформлено в системе \LaTeX, макрос \NCC
}
{Видавець і виготовлювач \\
Національний аерокосмічний університет ім.~М.~Є.~Жуковського \\
``Харківський авіаційний інститут'' \\
61070, Харків-70, вул.~Чкалова, 17 \\
http://www.khai.edu \\
Видавничий центр ``ХАІ'' \\
61070, Харків-70, вул.~Чкалова, 17 \\
izdat@khai.edu}
{Свідоцтво про внесення суб'єкта видавничої справи до Державного реєстру видавців, виготовлювачів і розповсюджувачів видавничої продукції сер.~ДК №391 від 30.03.2001}


%!TeX root = book.tex
%!TEX TS-program = lualatex

\begin{russian}
\setcounter{secnumdepth}{-1}
\chapter{Предисловие}
\markright{Предисловие}
\setcounter{secnumdepth}{2}

В последние десятилетия пористые и композиционные материалы получили широкое распространение не только в высокотехнологичных областях современной техники, но и в разных сферах окружающей нас жизни, таких, как микроэлектроника, транспортное машиностроение, строительство, дизайн, упаковка и др. Такой интерес к использованию этих материалов связан с их уникальными свойствами, в которых сочетаются высокие прочностные качества с малой удельной плотностью. Широта использования многокомпонентных материалов выдвигает на один из передних планов науки проблемы моделирования в них физико-механических полей, анализа их прочностных характеристик, создания оптимальных материалов с заранее заданными свойствами. Решение этих задач невозможно без переосмысления методов и подходов, сложившихся в механике композиционных материалов в последние годы, и создания новых методов исследования пространственных задач теории упругости для многосвязных тел с большим числом компонент связности. Надо заметить, что известные аналитико-численные методы недостаточно эффективны в подобных задачах. В конечном итоге все они не учитывают в полной мере структуру материала, что приводит зачастую к результатам с малой точностью.\sloppy

В данной монографии впервые в мировой научной литературе ставится проблема определения с любой практической степенью точности полей напряжений и деформаций в многосвязных пространственных телах, имеющих большое число неоднородных фаз. Для ее решения используется обобщенный метод Фурье (ОМФ), который был создан в работах одного из авторов монографии. В настоящей книге он получил дальнейшее развитие, которое позволило сделать его более адаптированным к задачам, рассматриваемым в монографии.

В книге исследованы пространственные краевые задачи теории упругости для многокомпонентных тел с неоднородностями в виде одинаково ориентированных цилиндрических, сферических, вытянутых и сжатых сфероидальных полостей или включений. Центры неоднородностей расположены в узлах плоской (цилиндры) или пространственной (сферы, сфероиды) решеток, которые обладают определенной трансляционной симметрией. Обычно в этом случае говорят о некоторой упаковке неоднородностей в материале. Рассматриваются случаи тетрагональной и гексагональной упаковок при наличии или отсутствии объемного центрирования.

Основное внимание в монографии уделено количественному и качественному анализу напряженно-деформированного состояния в зонах наибольшей концентрации напряжений в указанных выше многокомпонентных телах. Изучены локальные модели, в которых распределения напряжений определяются двумя или несколькими ближайшими неоднородностями, образующими ячейку рассматриваемой упаковки. Исследовано влияние дальних неоднородностей на значения напряжений в ячейке. Приводится анализ поведения напряжений в зависимости от изменения геометрических и механических параметров дисперсной фазы материала. Дано сравнение распределений напряжений для разных типов упаковок неоднородностей.

Кроме локальных моделей рассмотрены также модели напряженно-деформированного состояния многокомпонентного материала с периодически расположенными полостями или включениями.

Полученные результаты позволили исследовать в настоящей книге проблему определения эффективных упругих модулей для некоторых типов пористых и композиционных материалов.

Материал монографии полностью основан на исследованиях авторов книги, которые проводились в течение нескольких последних лет.

Авторы считают своим приятным долгом выразить благодарность рецензентам книги члену-корреспонденту НАН Украины, профессору А.~Н.~Довбне и профессору В.~А.~Дорошенко. Особую признательность авторы выражают Л.~А.~Кузьменко за квалифицированное литературное редактирование рукописи книги.

%\begin{flushright}
%\vskip1cm
%А.~Г.~Николаев, \\
%Е.~А.~Танчик. \\
%\vskip1cm
%Харьков, \\
%декабрь 2014~г.
%\end{flushright}

\end{russian}
\include{intro}
\include{chapter-1}
\include{chapter-7}
\include{chapter-8}
\include{chapter-9}
%!TeX root = book.tex
%!TEX TS-program = lualatex

\begin{russian}
\chapter[Механика упругого деформирования зернистых композитов со сжатыми сфероидальными зернами]{Механика упругого деформирования зернистых композитов со сжатыми сфероидальными зернами}\chaptermark{Деформирование композитов со сжатыми сфероидальными зернами}

\section[Упругое состояние пространства с несколькими сжатыми сфероидальными полостями]{Упругое состояние пространства с несколькими сжатыми сфероидальными полостями\sectionmark{Упругое состояние пространства со сжатыми сфероидальными полостями}}\sectionmark{Упругое состояние пространства со сжатыми сфероидальными полостями}

Рассмотрим одно(дву)осное растяжение на бесконечности упругого пространства с несколькими сжатыми сфероидальными полостями, расположенными неосесимметрично. Центры полостей находятся в точках $O_j$, а их границы задаются уравнениями

\begin{equation}
\frac{{\rho _j^2}}{{d_{j1}^2}} + \frac{{z_j^2}}{{d_{j2}^2}} = 1,
\label{eq:10:12}
\end{equation}

\noindent где $d_{ij}>0$~--- полуоси сфероидов, $(\rho_j,\varphi_j,z_j)$~--- одинаково направленные цилиндрические системы координат, начала которых совпадают с точками $O_j$. Считается, что полости свободны от нагрузки.

\begin{figure}[h!]
\centering
\includegraphics[width=8cm]{oblate-spheroids.pdf}
\caption{Схематическое представление задачи}
\label{f:10:1o}
\end{figure}

Введем вытянутые сфероидальные системы координат $\left( {{{\tilde \xi }_j},{{\tilde \eta }_j},{\varphi _j}} \right)$, совмещенные с цилиндрическими системами. В сфероидальных координатах поверхности полостей задаются уравнениями ${\Gamma _j}:\,{\tilde \xi _j} = {\tilde \xi _{j0}}$, где $\tilde\xi_{j0}$ находят из системы уравнений

\begin{equation}
\left\{ {\begin{array}{*{20}{l}}
{{{\tilde c}_j}{\mathop{\rm ch}\nolimits} {{\tilde \xi }_{j0}} = {d_{j1}},}\\
{{{\tilde c}_j}{\mathop{\rm sh}\nolimits} {{\tilde \xi }_{j0}} = {d_{j2}}.}
\end{array}} \right.
\end{equation}

Для определения НДС в рассматриваемом теле необходимо решить краевую задачу относительно вектора перемещения $\mathbf{U}$, удовлетворяющего уравнению Ламе, граничным условиям на поверхностях $\Gamma_j$

\begin{equation}
{\bf{FU}}{|_{{\Gamma _j}}} = 0\qquad {\kern 1pt} \left( {j = \overline {1,N} } \right)
\label{eq:10:3}
\end{equation}
и условиям на бесконечности~\eqref{eq:9:21}, \eqref{eq:9:22}.

Решение задачи будем искать в виде

\begin{equation}
{\bf{U}} = {\bf{\tilde U}} + {{\bf{U}}_0},
\end{equation}

\begin{equation}
{\bf{\tilde U}} = \sum\limits_{j = 1}^N {\sum\limits_{s = 1}^3 {\sum\limits_{n = 0}^\infty  {\sum\limits_{m =  - n - 1}^{n + 1} {a_{s,n,m}^{(j)}} } } } {\bf{U}}_{s,n,m}^{ + (6)}\left( {{{\tilde \xi }_j},{{\tilde \eta }_j},{\varphi _j}} \right).
\end{equation}

Вспомогательный вектор $\mathbf{U}_0$ описан в~\eqref{eq:9:7}, \eqref{eq:9:8}, $a_{s,n,m}^{(j)}$~--- неизвестные коэффициенты.

Рассмотрим базисные частные решения уравнения Ламе для сжатого сфероида $\Omega _6^ \pm \left\{ {(\xi ,\eta ,\varphi ):{\mkern 1mu} {\kern 1pt} \xi  \mathbin{\lower.3ex\hbox{$\buildrel>\over
{\smash{\scriptstyle<}\vphantom{_x}}$}} {\xi _0}} \right\}$:

\begin{multline}
{\bf{U}}_{s,n,m}^{ \pm (6)}\left( {\tilde \xi ,\tilde \eta ,\varphi } \right) = \\
= \frac{{ - i\tilde c}}{{2n + 1}}{{\bf{D}}_s}\left[ {u_{n - 1,m}^{ \pm (6)}\left( {\tilde \xi ,\tilde \eta ,\varphi } \right) - u_{n + 1,m}^{ \pm (6)}\left( {\tilde \xi ,\tilde \eta ,\varphi } \right)} \right];\qquad {\kern 1pt} s = 1,3;
\label{eq:10:1}
\end{multline}

\begin{equation}
{\bf{U}}_{2,n,m}^{ \pm (6)}\left( {\tilde \xi ,\tilde \eta ,\varphi } \right) = {{\bf{D}}_2}u_{n,m}^{ \pm (6)}\left( {\tilde \xi ,\tilde \eta ,\varphi } \right) - i\tilde c{{\mathop{\rm sh}\nolimits} ^2}{\tilde \xi _0}{{\bf{D}}_1}u_{n \pm 1,m}^{ \pm (6)}\left( {\tilde \xi ,\tilde \eta ,\varphi } \right),
\label{eq:10:2}
\end{equation}
где $n=0,1,\dots$; $|m|\le n+1$; $m,n\in\mathbb{Z}$; $\mathbf{D}_s$ определены в~\eqref{eq:1:41};

\begin{multline}
u_{n,m}^{ \pm (6)}\left( {\tilde \xi ,\tilde \eta ,\varphi } \right) = \\
= \left\{ \begin{array}{l}
Q_n^{ - m}(i{\mathop{\rm sh}\nolimits} \tilde \xi )\\
P_n^{ - m}(i{\mathop{\rm sh}\nolimits} \tilde \xi )
\end{array} \right\}P_n^m(\cos \tilde \eta ){e^{im\varphi }},\qquad {\kern 1pt} n,m \in\mathbb{Z} ,\quad |m| \le n,
\end{multline}
где $P_n^m(x)$, $Q_n^m(x)$~--- функции Лежандра первого и второго родов соответственно.

Приведем координатную форму перемещений~\eqref{eq:10:1}, \eqref{eq:10:2}:

\begin{multline}
{\bf{U}}_{1,n,m}^{ \pm (6)}\left( {\tilde \xi ,\tilde \eta ,\varphi } \right) = \\
= u_{n,m - 1}^{ \pm (6)}\left( {\tilde \xi ,\tilde \eta ,\varphi } \right){{\bf{e}}_{ - 1}} - u_{n,m + 1}^{ \pm (6)}\left( {\tilde \xi ,\tilde \eta ,\varphi } \right){{\bf{e}}_1} - u_{n,m}^{ \pm (6)}\left( {\tilde \xi ,\tilde \eta ,\varphi } \right){{\bf{e}}_0};
\end{multline}

\begin{multline}
{\bf{U}}_{2,n,m}^{ \pm (6)}\left( {\tilde \xi ,\tilde \eta ,\varphi } \right) = i\tilde qu_{1,n,m - 1}^{ \pm (6)}\left( {\tilde \xi ,\tilde \eta ,\varphi } \right){{\bf{e}}_{ - 1}} - i\tilde qu_{1,n,m + 1}^{ \pm (6)}\left( {\tilde \xi ,\tilde \eta ,\varphi } \right){{\bf{e}}_1} - \\
- \left[ {i\tilde qu_{1,n,m}^{ \pm (6)}\left( {\tilde \xi ,\tilde \eta ,\varphi } \right) + \chi u_{n,m}^{ \pm (6)}\left( {\tilde \xi ,\tilde \eta ,\varphi } \right)} \right]{{\bf{e}}_0} + \\
+ i\tilde c\left( {{{\tilde q}^2} - \tilde q_0^2} \right)\nabla u_{n \pm 1,m}^{ \pm (6)}\left( {\tilde \xi ,\tilde \eta ,\varphi } \right);
\end{multline}

\begin{equation}
{\bf{U}}_{3,n,m}^{ \pm (6)}\left( {\tilde \xi ,\tilde \eta ,\varphi } \right) =  - u_{n,m - 1}^{ \pm (6)}\left( {\tilde \xi ,\tilde \eta ,\varphi } \right){{\bf{e}}_{ - 1}} - u_{n,m + 1}^{ \pm (6)}\left( {\tilde \xi ,\tilde \eta ,\varphi } \right){{\bf{e}}_1},
\end{equation}
где $\tilde q = {\mathop{\rm sh}\nolimits} \tilde \xi $, ${\tilde q_0} = {\mathop{\rm sh}\nolimits} {\tilde \xi _0}$;

\begin{equation}
u_{1,n,m}^{ \pm (6)}\left( {\tilde \xi ,\tilde \eta ,\varphi } \right) = \tilde u_{1,n,m}^{ \pm (6)}(\tilde \xi )P_n^m(\cos \tilde \eta ){e^{im\varphi }};
\end{equation}

\begin{equation}
\tilde u_{1,n,m}^{ \pm (6)}(\tilde \xi ) = \left\{ \begin{array}{l}
(n + m + 1)Q_{n + 1}^{ - m}(i\tilde q)\\
 - (n - m)P_{n - 1}^{ - m}(i\tilde q)
\end{array} \right\}.
\end{equation}

В работе~\cite{Nikolaev1998} установлена базисность решений~\eqref{eq:10:1}, \eqref{eq:10:2} в областях $\Omega_6^\pm$.

Справедливы следующие теоремы сложения базисных решений уравнения Ламе для сжатого сфероида:

\begin{multline}
{\bf{U}}_{s,n,m}^{ + (6)}\left( {{{\tilde \xi }_1},{{\tilde \eta }_1},{\varphi _1}} \right) = \\
= \sum\limits_{k = 0}^\infty  {\sum\limits_{l =  - \infty }^\infty  {f_{n,m}^{ + (66)k,l}} } {\bf{U}}_{s,k,l}^{ - (6)}\left( {{{\tilde \xi }_2},{{\tilde \eta }_2},{\varphi _2}} \right),\,s = 1,{\mkern 1mu} {\kern 1pt} 3;
\label{eq:10:4}
\end{multline}

\begin{multline}
{\bf{U}}_{2,n,m}^{ + (6)}\left( {{{\tilde \xi }_1},{{\tilde \eta }_1},{\varphi _1}} \right) = \sum\limits_{k = 0}^\infty  {\sum\limits_{l =  - \infty }^\infty  {\left[ {f_{n,m}^{ + (66)k,l}} \right.} } {\bf{U}}_{2,k,l}^{ - (6)}\left( {{{\tilde \xi }_2},{{\tilde \eta }_2},{\varphi _2}} \right) + \\
\left. { + \tilde f_{n,m}^{ + (66)k,l}{\bf{U}}_{1,k,l}^{ - (6)}\left( {{{\tilde \xi }_2},{{\tilde \eta }_2},{\varphi _2}} \right)} \right];
\label{eq:10:5}
\end{multline}

\begin{equation}
f_{n,m}^{ + (66)k,l} = \sum\limits_{j = n}^\infty  {g_{n,m}^{ + (64)j}} ({\tilde c_1})f_{j,m}^{(46)k,l}({\tilde c_2});
\end{equation}

\begin{multline}
g_{n,m}^{ + (64)j}({\tilde c_1}) = \\
= {( - 1)^m}\sqrt \pi  {( - i)^{j + 1}}{\left( {\frac{{{{\tilde c}_1}}}{2}} \right)^{j + 1}}\frac{{{\varepsilon _{jn}}}}{{\Gamma \left( {\frac{{j - n}}{2} + 1} \right)\Gamma \left( {\frac{{j + n}}{2} + \frac{3}{2}} \right)}};
\end{multline}

\begin{multline}
f_{j,m}^{(46)k,l}({\tilde c_2}) = \\
= \sum\limits_{p = 0}^\infty  {{{( - 1)}^{p + l}}} \sqrt \pi  {( - i)^p}{\left( {\frac{{{{\tilde c}_2}}}{2}} \right)^p}\frac{{{\varepsilon _{pk}}\left( {k + \frac{1}{2}} \right)u_{j + p,m - l}^{ + (4)}\left( {{r_{12}},{\theta _{12}},{\varphi _{12}}} \right)}}{{\Gamma \left( {\frac{{p - k}}{2} + 1} \right)\Gamma \left( {\frac{{p + k}}{2} + \frac{3}{2}} \right)}};
\end{multline}

\begin{multline}
\tilde f_{n,m}^{ + (66)k,l} = \sum\limits_{j = n}^\infty  {\left[ {i{{\tilde c}_2}\tilde q_{20}^2\frac{{2k + 1}}{{2k + 3}}} \right.} g_{n,m}^{ + (64)j}f_{j + 1,m}^{(46)k + 1,l} + {z_{12}}g_{n,m}^{ + (64)j}f_{j + 1,m}^{(46)k,l} - \\
\left. {\frac{{}}{{}}i{{\tilde c}_1}\tilde q_{10}^2g_{n + 1,m}^{ + (64)j - 1}f_{j,m}^{(46)k,l}} \right];
\end{multline}

\begin{multline}
{\bf{U}}_{s,n,m}^{ + (6)}\left( {{{\tilde \xi }_2},{{\tilde \eta }_2},{\varphi _2}} \right) = \\
= \sum\limits_{k = 0}^\infty  {\sum\limits_{l =  - \infty }^\infty  {f_{n,m}^{ - (66)k,l}} } {\bf{U}}_{s,k,l}^{ - (6)}\left( {{{\tilde \xi }_1},{{\tilde \eta }_1},{\varphi _1}} \right),\,s = 1,{\mkern 1mu} {\kern 1pt} 3;
\label{eq:10:6}
\end{multline}

\begin{multline}
{\bf{U}}_{2,n,m}^{ + (6)}\left( {{{\tilde \xi }_2},{{\tilde \eta }_2},{\varphi _2}} \right) = \sum\limits_{k = 0}^\infty  {\sum\limits_{l =  - \infty }^\infty  {\left[ {f_{n,m}^{ - (66)k,l}} \right.} } {\bf{U}}_{2,k,l}^{ - (6)}\left( {{{\tilde \xi }_1},{{\tilde \eta }_1},{\varphi _1}} \right) + \\
\left. { + \tilde f_{n,m}^{ - (66)k,l}{\bf{U}}_{1,k,l}^{ - (6)}\left( {{{\tilde \xi }_1},{{\tilde \eta }_1},{\varphi _1}} \right)} \right];
\label{eq:10:7}
\end{multline}

\begin{equation}
f_{n,m}^{ - (66)k,l} = \sum\limits_{j = k}^\infty  {f_{n,m}^{(64)j,l}({{\tilde c}_2})g_{j,l}^{ - (46)k}} ({\tilde c_1});
\end{equation}

\begin{multline}
f_{n,m}^{(64)j,l}({\tilde c_2}) = \\
= \sum\limits_{p = 0}^\infty  {{{( - 1)}^{p + l + m}}} \sqrt \pi  {( - i)^{p + 1}}{\left( {\frac{{{{\tilde c}_2}}}{2}} \right)^{p + 1}}\frac{{{\varepsilon _{pn}}u_{p + j,m - l}^{ + (4)}\left( {{r_{12}},{\theta _{12}},{\varphi _{12}}} \right)}}{{\Gamma \left( {\frac{{p - n}}{2} + 1} \right)\Gamma \left( {\frac{{p + n}}{2} + \frac{3}{2}} \right)}};
\end{multline}

\begin{multline}
\tilde f_{n,m}^{ - (66)k,l} = \sum\limits_{j = k}^\infty\bigg[i{{\tilde c}_2}\tilde q_{20}^2f_{n + 1,m}^{(64)j + 1,l}({{\tilde c}_2})g_{j,l}^{ - (46)k}({{\tilde c}_1}) + \\
+ {z_{12}}f_{n,m}^{(64)j + 1,l}({{\tilde c}_2})g_{j,l}^{ - (46)k}({{\tilde c}_1}) - \frac{{}}{{}}i{{\tilde c}_1}\tilde q_{10}^2\frac{{2k + 1}}{{2k + 3}}f_{n,m}^{(64)j,l}({{\tilde c}_2})g_{j - 1,l}^{ - (46)k + 1}({{\tilde c}_1}) \bigg].
\end{multline}

Приведем формулы для напряжений, отвечающих базисным перемещениям ${\bf{U}}_{s,n,m}^{ \pm (6)}\left( {{{\tilde \xi }_j},{{\tilde \eta }_j},{\varphi _j}} \right)$ на поверхностях $\Gamma_j$ ($\mathbf{n}_j=\mathbf{e}_{\tilde\xi_j}$~--- нормаль к поверхности $\Gamma_j$):

\begin{multline}
{\bf{FU}}_{1,n,m}^{ \pm (6)}\left( {{{\tilde \xi }_j},{{\tilde \eta }_j},{\varphi _j}} \right) = 2G\frac{{{{\tilde H}_j}}}{{{{\tilde c}_j}}}\left[ {\frac{\partial }{{\partial {{\tilde \xi }_j}}}u_{n,m - 1}^{ \pm (6)}\left( {{{\tilde \xi }_j},{{\tilde \eta }_j},{\varphi _j}} \right){{\bf{e}}_{ - 1}} - } \right.\\
\left. { - \frac{\partial }{{\partial {{\tilde \xi }_j}}}u_{n,m + 1}^{ \pm (6)}\left( {{{\tilde \xi }_j},{{\tilde \eta }_j},{\varphi _j}} \right){{\bf{e}}_1} - \frac{\partial }{{\partial {{\tilde \xi }_j}}}u_{n,m}^{ \pm (6)}\left( {{{\tilde \xi }_j},{{\tilde \eta }_j},{\varphi _j}} \right){{\bf{e}}_0}} \right];
\label{eq:10:25o}
\end{multline}

\begin{multline}
{\bf{FU}}_{2,n,m}^{ \pm (6)}\left( {{{\tilde \xi }_j},{{\tilde \eta }_j},{\varphi _j}} \right) = 2G\frac{{{{\tilde H}_j}}}{{{{\tilde c}_j}}}\left\{ {\left[ { - \bar q_j^2\frac{\partial }{{\partial {{\tilde \xi }_j}}}\left( {\frac{{u_{1,n,m - 1}^{ \pm (6)}}}{{{{\bar q}_j}}}} \right) - 2\sigma u_{2,n,m}^{ \pm (6)}} \right]{{\bf{e}}_{ - 1}} - } \right.\\
- \left[ { - \bar q_j^2\frac{\partial }{{\partial {{\tilde \xi }_j}}}\left( {\frac{{u_{1,n,m + 1}^{ \pm (6)}}}{{{{\bar q}_j}}}} \right) - 2\sigma u_{3,n,m}^{ \pm (6)}} \right]{{\bf{e}}_1} - \left[ { - \tilde q_j^2\frac{\partial }{{\partial {{\tilde \xi }_j}}}\left( {\frac{{u_{1,n,m}^{ \pm (6)}}}{{{{\bar q}_j}}}} \right) + } \right.\\
\left. { + \left. {(1 - 2\sigma )\frac{\partial }{{\partial {{\tilde \xi }_j}}}u_{n,m}^{ \pm (6)}} \right]{{\bf{e}}_0}} \right\};
\label{eq:10:26o}
\end{multline}

\begin{multline}
{\bf{FU}}_{3,n,m}^{ \pm (6)}\left( {{\tilde \xi _j},{\eta _j},{\varphi _j}} \right) = 2G\frac{{{{\tilde H}_j}}}{{{{\tilde c}_j}}}\left\{ { - \left[ {\frac{\partial }{{\partial {{\tilde \xi }_j}}}u_{n,m - 1}^{ \pm (6)} - \frac{1}{2}u_{2,n,m}^{ \pm (6)}} \right]{{\bf{e}}_{ - 1}} - } \right.\\
\left. { - \left[ {\frac{\partial }{{\partial {{\tilde \xi }_j}}}u_{n,m + 1}^{ \pm (6)} - \frac{1}{2}u_{3,n,m}^{ \pm (6)}} \right]{{\bf{e}}_1} + \frac{m}{2}\frac{{{{\bar q}_j}}}{{{q_j}}}u_{n,m}^{ \pm (6)}{{\bf{e}}_0}} \right\},
\label{eq:10:27o}
\end{multline}

где ${\tilde H_j} = {\left( {\bar q_j^2 + {{\cos }^2}{{\tilde \eta }_j}} \right)^{ - \frac{1}{2}}}$;
$$
u_{1,n,m}^{ \pm (6)}\left( {{{\tilde \xi }_j},{{\tilde \eta }_j},{\varphi _j}} \right) = \tilde u_{1,n,m}^{ \pm (6)}({\tilde \xi _j})P_n^m(\cos {\tilde \eta _j}){e^{im{\varphi _j}}};
$$
$$
u_{2,n,m}^{ \pm (6)}\left( {{{\tilde \xi }_j},{{\tilde \eta }_j},{\varphi _j}} \right) = \tilde u_{2,n,m}^{ \pm (6)}({\tilde \xi _j})P_n^{m - 1}(\cos {\tilde \eta _j}){e^{i(m - 1){\varphi _j}}};
$$
$$
u_{3,n,m}^{ \pm (6)}\left( {{{\tilde \xi }_j},{{\tilde \eta }_j},{\varphi _j}} \right) = \tilde u_{n,m}^{ \pm (6)}({\tilde \xi _j})P_n^{m + 1}(\cos {\tilde \eta _j}){e^{i(m + 1){\varphi _j}}};
$$
$$
\tilde u_{n,m}^{ \pm (6)}(\xi ) = \left\{ \begin{array}{l}
Q_n^{ - m}(i\bar q)\\
P_n^{ - m}(i\bar q)
\end{array} \right\};\qquad {\kern 1pt} \tilde u_{1,n,m}^{ \pm (6)}(\tilde \xi ) = \left\{ \begin{array}{l}
(n + m + 1)Q_{n + 1}^{ - m}(i\bar q)\\
 - (n - m)P_{n - 1}^{ - m}(i\bar q)
\end{array} \right\};
$$
$$
\tilde u_{2,n,m}^{ \pm (6)}(\tilde \xi ) = (n + m)(n - m + 1)\tilde u_{n,m}^{ \pm (6)}(\tilde \xi ),\qquad {\kern 1pt} q = {\mathop{\rm ch}\nolimits} \tilde \xi ,\quad \bar q = {\mathop{\rm sh}\nolimits} \tilde\xi.
$$

Относительно перемещения $\mathbf{\tilde U}$ граничные условия~\eqref{eq:10:3} можно записать так:

\begin{equation}
{\bf{F\tilde U}}{|_{{\Gamma _j}}} =  - {\bf{F}}{{\bf{U}}_0}{|_{{\Gamma _j}}}.
\end{equation}

На бесконечности вектор $\mathbf{F\tilde U}$ подчиняется условиям~\eqref{eq:9:21}, \eqref{eq:9:22}. Заметим, что вектор усилий ${\bf{F}}{{\bf{U}}_0}{|_{{\Gamma _j}}}$ для каждого типа растяжений упругого пространства находят по формулам $(\mathbf{n}_j=\mathbf{e}_{\tilde\xi_j})$

\begin{equation}
{\bf{F}}{{\bf{U}}_0}{|_{{\Gamma _j}}} = T{\tilde H_j}{\mathop{\rm ch}\nolimits} {\tilde \xi _{j0}}{P_1}(\cos {\tilde \eta _j}){{\bf{e}}_z}\quad\text{(одноосное растяжение)},
\end{equation}

\begin{equation}
{\bf{F}}{{\bf{U}}_0}{|_{{\Gamma _j}}} =  - T{\tilde H_j}{\mathop{\rm sh}\nolimits} {\tilde \xi _{j0}}P_1^{(1)}(\cos {\tilde \eta _j}){{\bf{e}}_{{\rho _j}}}\quad\text{(двуосное растяжение)}.
\end{equation}

Используя теоремы сложения~\eqref{eq:10:4}, \eqref{eq:10:5} и \eqref{eq:10:6}, \eqref{eq:10:7}, перемещение $\mathbf{\tilde U}$ можно записать в системе координат с началом в точке $O_j$:

\begin{multline}
{\bf{\tilde U}} = \sum\limits_{s = 1}^3 {\sum\limits_{n = 0}^\infty  {\sum\limits_{m =  - n - 1}^{n + 1} {a_{s,n,m}^{(j)}} } } {\bf{U}}_{s,n,m}^{ + (6)}\left( {{{\tilde \xi }_j},{{\tilde \eta }_j},{\varphi _j}} \right) + \\
+ \sum\limits_{n = 0}^\infty  {\sum\limits_{m =  - n - 1}^{n + 1} {\left\{ {{\bf{U}}_{1,n,m}^{ - (6)}\left( {{{\tilde \xi }_j},{{\tilde \eta }_j},{\varphi _j}} \right)\sum\limits_{\alpha  \ne j} {\sum\limits_{k = 0}^\infty  {\sum\limits_{l =  - k - 1}^{k + 1} {\left[ {a_{1,k,l}^{(\alpha )}f_{k,l,j,\alpha }^{ - (66)n,m} + } \right.} } } } \right.} } \\
\left. { + a_{2k,l}^{(\alpha )}\tilde f_{k,l,j,\alpha }^{ - (66)n,m}} \right] + {\bf{U}}_{2,n,m}^{ - (6)}\left( {{{\tilde \xi }_j},{{\tilde \eta }_j},{\varphi _j}} \right)\sum\limits_{\alpha  \ne j} {\sum\limits_{k = 0}^\infty  {\sum\limits_{l =  - k - 1}^{k + 1} {a_{2,k,l}^{(\alpha )}} } \tilde f_{k,l,j,\alpha }^{ - (66)n,m} + } \\
\left. { + {\bf{U}}_{3,n,m}^{ - (6)}\left( {{{\tilde \xi }_j},{{\tilde \eta }_j},{\varphi _j}} \right)\sum\limits_{\alpha  \ne j} {\sum\limits_{k = 0}^\infty  {\sum\limits_{l =  - k - 1}^{k + 1} {a_{3,k,l}^{(\alpha )}} } f_{k,l,j,\alpha }^{ - (66)n,m}} } \right\}.
\label{eq:10:8}
\end{multline}

После перехода в формулах~\eqref{eq:10:8} к напряжениям и удовлетворения граничных условий относительно неизвестных $a_{s,n,m}^{(j)}$ получаем бесконечную систему линейных алгебраических уравнений:

\begin{multline}
\sum\limits_{s = 1}^3 {a_{s,n,m}^{(1)}} \tilde W_{s,n,m}^{ + (k)}({\tilde \xi _{10}}) + \tilde W_{1,n,m}^{ - (k)}({\tilde \xi _{10}})\sum\limits_{k = 0}^\infty  {\sum\limits_{l =  - k - 1}^{k + 1} {\left[ {a_{1,k,l}^{(2)}f_{k,l}^{ - (66)n,m} + } \right.} } \\
\left. { + a_{2,k,l}^{(2)}\tilde f_{k,l}^{ - (66)n,m}} \right] + \tilde W_{2,n,m}^{ - (k)}({\tilde \xi _{10}})\sum\limits_{k = 0}^\infty  {\sum\limits_{l =  - k - 1}^{k + 1} {a_{2,k,l}^{(2)}} } f_{k,l}^{ - (66)n,m} + \\
+ \tilde W_{3,n,m}^{ - (k)}({\tilde \xi _{10}})\sum\limits_{k = 0}^\infty  {\sum\limits_{l =  - k - 1}^{k + 1} {a_{3,k,l}^{(2)}} } f_{k,l}^{ - (66)n,m} = \tilde W_{n,m}^{(k)j};
\end{multline}
$$
n,m \in\mathbb{Z}:\quad n \ge 0,\quad |m| \le n + 1,\quad k =  - 1,{\mkern 1mu} {\kern 1pt} 0,{\mkern 1mu} {\kern 1pt} 1;
$$

\begin{equation}
\tilde W_{1,n,m}^{ \pm ( - 1)}(\tilde \xi ) = \frac{\partial }{{\partial \tilde \xi }}\tilde u_{n,m - 1}^{ \pm (6)}(\tilde \xi ),\quad \tilde W_{1,n,m}^{ \pm (1)} =  - \frac{\partial }{{\partial \tilde \xi }}\tilde u_{n,m + 1}^{ \pm (6)}(\tilde \xi );
\end{equation}

\begin{equation}
\tilde W_{1,n,m}^{ \pm (0)}(\tilde \xi ) =  - \frac{\partial }{{\partial \tilde \xi }}\tilde u_{n,m}^{ \pm (6)}(\tilde \xi );
\end{equation}
\begin{equation}
\tilde W_{2,n,m}^{ \pm ( - 1)}(\tilde \xi ) = i{\bar q^2}\frac{\partial }{{\partial \tilde \xi }}\left[ {\frac{{\tilde u_{1,n,m - 1}^{ \pm (6)}(\tilde \xi )}}{{\bar q}}} \right] - 2\sigma \tilde u_{2,n,m}^{ \pm (6)}(\tilde \xi );
\end{equation}

\begin{equation}
\tilde W_{2,n,m}^{ \pm (1)}(\tilde \xi ) =  - i{\bar q^2}\frac{\partial }{{\partial \tilde \xi }}\left[ {\frac{{u_{1,n,m + 1}^{ \pm (6)}(\tilde \xi )}}{{\bar q}}} \right] + 2\sigma \tilde u_{n,m}^{ \pm (6)}(\tilde \xi );
\end{equation}

\begin{equation}
\tilde W_{2,n,m}^{ \pm (0)}(\tilde \xi ) =  - i{\bar q^2}\frac{\partial }{{\partial \tilde \xi }}\left[ {\frac{{\tilde u_{1,n,m}^{ \pm (6)}(\tilde \xi )}}{{\bar q}}} \right] - (1 - 2\sigma )\frac{\partial }{{\partial \tilde \xi }}\tilde u_{n,m}^{ \pm (6)}(\tilde \xi );
\end{equation}

\begin{equation}
\tilde W_{3,n,m}^{ \pm ( - 1)}(\tilde \xi ) =  - \frac{\partial }{{\partial \tilde \xi }}\tilde u_{n,m - 1}^{ \pm (6)}(\tilde \xi ) + \frac{1}{2}\tilde u_{2,n,m}^{ \pm (6)}(\tilde \xi );
\end{equation}

\begin{equation}
\tilde W_{3,n,m}^{ \pm (1)}(\tilde \xi ) =  - \frac{\partial }{{\partial \tilde \xi }}\tilde u_{n,m + 1}^{ \pm (6)}(\tilde \xi ) + \frac{1}{2}\tilde u_{n,m}^{ \pm (6)}(\tilde \xi );
\end{equation}
\begin{equation}
\tilde W_{3,n,m}^{ \pm (0)}(\tilde \xi ) = \frac{m}{2}\frac{{\bar q}}{q}\tilde u_{n,m}^{ \pm (6)}(\tilde \xi );
\end{equation}

\begin{equation}
\tilde W_{n,m}^{(k)j} = \left\{ \begin{array}{l}
 - \dfrac{T}{{2G}}{{\tilde c}_j}{\mathop{\rm ch}\nolimits} {{\tilde \xi }_{j0}}{\delta _{n1}}{\delta _{m0}}\quad\text{(одноосное растяжение)},\\ \\
\dfrac{T}{{2G}}{{\tilde c}_j}{\mathop{\rm sh}\nolimits} {{\tilde \xi }_{j0}}{\delta _{n1}}{\delta _{m0}}\quad\text{(двуосное растяжение)}.
\end{array} \right.
\end{equation}

\subsection{Анализ численных результатов для пространства со сжатыми сфероидальными полостями при одноосном и двуосном растяжениях}

\sidefig[t](99mm){
\includegraphics[width=10cm]{cartesian-oblate-spheroids-4.pdf}
\caption{Четыре сжатые сфероидальные полости}
\label{f:10:1}
}{Рассмотрим одноосное и двуосное растяжения упругого пространства с четырьмя сжатыми сфероидальными полостями. Полости расположены в вершинах квадрата (рис.~\ref{f:10:1}). Считается, что полости свободны от нагрузки.

При численном анализе полагаем коэффициент Пуассона материала упругого} пространства $\sigma=0.38$, полости считаем одного размера, отношение полуосей сфероидов~--- $d_2/d_1=0.5$. Разрешающую систему уравнений численно решаем методом редукции. На основании полученных решений находим нормальные напряжения на площадках, параллельных координатным плоскостям.

На рис.~\ref{f:10:2}~--- \ref{f:10:4} приведены напряжения $\sigma_x/T$, $\sigma_y/T$ и $\sigma_z/T$ на линии $O_1O_4$ (см.~рис.~\ref{f:10:1}) вне полостей при одноосном растяжении в зависимости от относительного расстояния $a/d_1$ между полостями.

Областью концентрации напряжений $\sigma_y/T$ и $\sigma_z/T$ является граница полостей, в то время как напряжения $\sigma_x/T$ достигают максимальных значений в окрестности середины отрезка $O_1O_4$. Для всех случаев характерен рост напряжений с приближением полостей друг к другу.

На рис.~\ref{f:10:5}~--- \ref{f:10:7} приведены напряжения $\sigma_x/T$, $\sigma_y/T$ и $\sigma_z/T$ на линии $O_1O_4$ (рис.~\ref{f:10:1}) вне полостей при двуосном растяжении в зависимости от относительного расстояния $a/d_1$ между полостями.

\begin{figure}[h!]
\centering\footnotesize
\parbox[b]{7.5cm}{\centering\includegraphics[width=7.6cm]{cav4-oblate-a-d50-t1-sig_x.pdf}
\caption{Напряжения $\sigma_x/T$ на линии $O_1O_4$ в зависимости от относительного расстояния между полостями при одноосном растяжении
\label{f:10:2}}}\hfil\hfil
\parbox[b]{7.5cm}{\centering\includegraphics[width=7.6cm]{cav4-oblate-a-d50-t1-sig_y.pdf}
\caption{Напряжения $\sigma_y/T$ на линии $O_1O_4$ в зависимости от относительного расстояния между полостями при одноосном растяжении
\label{f:10:3}}}
\end{figure}

\begin{figure}[h!]
\centering\footnotesize
\parbox[b]{7.5cm}{\centering\includegraphics[width=7.6cm]{cav4-oblate-a-d50-t1-sig_z.pdf}
\caption{Напряжения $\sigma_z/T$ на линии $O_1O_4$ в зависимости от относительного расстояния между полостями при одноосном растяжении
\label{f:10:4}}}\hfil\hfil
\parbox[b]{7.5cm}{\centering\includegraphics[width=7.6cm]{cav4-oblate-a-d50-t2-sig_x.pdf}
\caption{Напряжения $\sigma_x/T$ на линии $O_1O_4$ в зависимости от относительного расстояния между полостями при двуосном растяжении
\label{f:10:5}}}
\end{figure}

Напряжения $\sigma_x/T$ убывают с приближением полостей друг к другу. Напряжения $\sigma_y/T$ растут с приближением полостей и для случая $a/d_1=2.5$ практически постоянны на всем рассматриваемом отрезке. Областью концентрации напряжений $\sigma_z/T$ являются границы полостей, причем с удалением полостей друг от друга напряжения $\sigma_z/T$ растут по модулю, оставаясь вблизи полостей сжимающими.

На рис.~\ref{f:10:8}~--- \ref{f:10:10} представлены напряжения $\sigma_x/T$, $\sigma_y/T$ и $\sigma_z/T$ на линии $O_1O_4$ вне полостей при одноосном растяжении в зависимости от отношения вертикальной полуоси сфероида к его горизонтальной полуоси при $a/d_1=3.5$.

\begin{figure}[h!]
\centering\footnotesize
\parbox[b]{7.5cm}{\centering\includegraphics[width=7.6cm]{cav4-oblate-a-d50-t2-sig_y.pdf}
\caption{Напряжения $\sigma_y/T$ на линии $O_1O_4$ в зависимости от относительного расстояния между полостями при двуосном растяжении
\label{f:10:6}}}\hfil\hfil
\parbox[b]{7.5cm}{\centering\includegraphics[width=7.6cm]{cav4-oblate-a-d50-t2-sig_z.pdf}
\caption{Напряжения $\sigma_z/T$ на линии $O_1O_4$ в зависимости от относительного расстояния между полостями при двуосном растяжении
\label{f:10:7}}}
\end{figure}

\begin{figure}[h!]
\centering\footnotesize
\parbox[b]{7.5cm}{\centering\includegraphics[width=7.6cm]{cav4-prolate-oblate-a35-t1-sig_x.pdf}
\caption{Напряжения $\sigma_x/T$ на линии $O_1O_4$ в зависимости от отношения полуосей сфероида при одноосном растяжении
\label{f:10:8}}}\hfil\hfil
\parbox[b]{7.5cm}{\centering\includegraphics[width=7.6cm]{cav4-prolate-oblate-a35-t1-sig_y.pdf}
\caption{Напряжения $\sigma_y/T$ на линии $O_1O_4$ в зависимости от отношения полуосей сфероида при одноосном растяжении
\label{f:10:9}}}
\end{figure}

\begin{figure}[h!]
\centering\footnotesize
\parbox[b]{7.5cm}{\centering\includegraphics[width=7.6cm]{cav4-prolate-oblate-a35-t1-sig_z.pdf}
\caption{Напряжения $\sigma_z/T$ на линии $O_1O_4$ в зависимости от отношения полуосей сфероида при одноосном растяжении
\label{f:10:10}}}\hfil\hfil
\parbox[b]{7.5cm}{\centering\includegraphics[width=7.6cm]{cav4-prolate-oblate-a35-t2-sig_x.pdf}
\caption{Напряжения $\sigma_x/T$ на линии $O_1O_4$ в зависимости от отношения полуосей сфероида при двуосном растяжении
\label{f:10:11}}}
\end{figure}

Наибольшие значения напряжений $\sigma_x/T$ наблюдаются при наименьшем отношении $d_2/d_1$. Для случая сжатых сфероидальных полостей присутствует ярко выраженная двухмодальность кривых в распределении напряжений.

Для напряжений $\sigma_y/T$ в середине рассматриваемого отрезка имеется область, в которой напряжения практически не зависят от отношения $d_2/d_1$, в то время как вблизи границ полостей есть область концентрации напряжений, в которой напряжения растут при уменьшении $d_2/d_1$. Похожая ситуация наблюдается и для напряжений $\sigma_z/T$.

Напряжения $\sigma_x/T$ концентрируются в середине отрезка и возрастают с уменьшением $d_2/d_1$. Для напряжений $\sigma_y/T$ характерно изменение направления выпуклости кривых при переходе от вытянутых сфероидальных полостей к сжатым сфероидальным. Для напряжений $\sigma_z/T$ наблюдается концентрация у границ полостей, при этом они растут по модулю с уменьшением отношения $d_2/d_1$.

На рис.~\ref{f:10:11}~--- \ref{f:10:13} приведены напряжения $\sigma_x/T$, $\sigma_y/T$ и $\sigma_z/T$ на линии $O_1O_4$ вне полостей при двуосном растяжении в зависимости от отношения вертикальной полуоси сфероида к его горизонтальной полуоси при $a/d_1=\\=3.5$.

По результатам проведенных вычислений можно сделать следующие выводы:
\begin{enumerate}
\item В случае одноосного растяжения наибольшей концентрацией (в 3.5 раза) обладают напряжения $\sigma_z/T$ на границах полостей, при этом коэффициент концентрации практически не зависит от взаимной близости полостей.
\item С уменьшением вертикальной полуоси сфероидов $d_2$ (переход от вытянутых сфероидов к сжатым) при фиксированном расстоянии между полостями все напряжения в области их концентрации растут при одноосном растяжении.
\item В случае двуосного растяжения наибольшей концентрацией (в 1.5 раза) обладают напряжения $\sigma_y/T$ на границах полостей, при этом коэффициент концентрации растет с ростом $d_2$ при фиксированном расстоянии между полостями.
\end{enumerate}

\begin{figure}[h!]
\centering\footnotesize
\parbox[b]{7.5cm}{\centering\includegraphics[width=7.6cm]{cav4-prolate-oblate-a35-t2-sig_y.pdf}
\caption{Напряжения $\sigma_y/T$ на линии $O_1O_4$ в зависимости от отношения полуосей сфероида при двуосном растяжении
\label{f:10:12}}}\hfil\hfil
\parbox[b]{7.5cm}{\centering\includegraphics[width=7.6cm]{cav4-prolate-oblate-a35-t2-sig_z.pdf}
\caption{Напряжения $\sigma_z/T$ на линии $O_1O_4$ в зависимости от отношения полуосей сфероида при двуосном растяжении
\label{f:10:13}}}
\end{figure}

\section[Упругое состояние пространства с несколькими сжатыми сфероидальными включениями]{Упругое состояние пространства с несколькими сжатыми сфероидальными включениями\sectionmark{Упругое состояние пространства со сжатыми сфероидальными включениями}}\sectionmark{Упругое состояние пространства со сжатыми сфероидальными включениями}

Рассмотрим постановку задачи предыдущего параграфа в случае, когда сферические полости заполнены упругими материалами с механическими характеристиками ($\sigma_j$, $G_j$). Упругие постоянные матрицы будем считать равными ($\sigma$, $G$).

Граничные условия~\eqref{eq:10:3} нужно заменить условиями сопряжения полей перемещений и напряжений на поверхностях $\Gamma_j$. Для того, чтобы их записать, представим вектор перемещений в упругом пространстве в виде

\begin{equation}
{\bf{U}} = \left\{ {\begin{array}{*{20}{l}}
{{\bf{\tilde U}}_j^ - ,\quad \left( {x,y,z} \right) \in {\Omega _j},}\\
{{{{\bf{\tilde U}}}^ + } + {{\bf{U}}_0},\quad \left( {x,y,z} \right) \in\mathbb{R}^3\backslash {\Omega _j},}
\end{array}} \right.
\label{eq:10:11}
\end{equation}
где ${\Omega _j} = \left\{ {\left( {{{\tilde \xi }_j},{{\tilde \eta }_j},{\varphi _j}} \right):\,{{\tilde \xi }_j} < {{\tilde \xi }_{j0}}} \right\}$. Тогда условия сопряжения принимают следующий вид:

\begin{equation}
\left( {{{{\bf{\tilde U}}}^ + } + {{\bf{U}}_0}} \right){|_{{\Gamma _j}}} = {\bf{\tilde U}}_j^ - {|_{{\Gamma _j}}};
\label{eq:10:9}
\end{equation}

\begin{equation}
\left( {{\bf{F\tilde U}} + {\bf{F}}{{\bf{U}}_0}} \right){|_{{\Gamma _j}}} = {\bf{F}}{{\bf{\tilde U}}_j}{|_{{\Gamma _j}}},\qquad {\kern 1pt} j = \overline {1,N}.
\label{eq:10:10}
\end{equation}

Вектор-функции $\mathbf{\tilde U}^+$, $\mathbf{\tilde U}_j^-$ будем искать в виде

\begin{equation}
{{\bf{\tilde U}}^ + } = \mathop \sum \limits_{j = 1}^N \sum\limits_{s = 1}^3 {\sum\limits_{n = 0}^\infty  {\sum\limits_{m =  - n - 1}^{n + 1} {a_{s,n,m}^{(j)}} } } {\bf{U}}_{s,n,m}^{ + (6)}\left( {{{\tilde \xi }_j},{{\tilde \eta }_j},{\varphi _j}} \right),
\end{equation}

\begin{equation}
{\bf{\tilde U}}_j^ -  = \sum\limits_{s = 1}^3 {\sum\limits_{n = 0}^\infty  {\sum\limits_{m =  - n - 1}^{n + 1} {b_{s,n,m}^{(j)}} } } {\bf{U}}_{s,n,m}^{ - (6)}\left( {{{\tilde \xi }_j},{{\tilde \eta }_j},{\varphi _j}} \right),
\end{equation}
где $a_{s,n,m}^{(j)}$, $b_{s,n,m}^{(j)}$~--- неизвестные коэффициенты.

Для представления вектора перемещений $\mathbf{\tilde U}^+$ в системах координат с началами $O_j$ можем использовать формулы~\eqref{eq:10:8}.

После удовлетворения условиям~\eqref{eq:10:9}, \eqref{eq:10:10} получаем бесконечную систему линейных алгебраических уравнений относительно неизвестных $a_{s,n,m}^{(j)}$, $b_{s,n,m}^{(j)}$:

\begin{multline}
\sum\limits_{s = 1}^3 {a_{s,n,m}^{(j)}} \tilde W_{s,n,m}^{ + (k)0}({\tilde \xi _{j0}}) + \tilde W_{1,n,m}^{ - (k)0}({\tilde \xi _{j0}})\sum\limits_{\alpha  \ne j} {\sum\limits_{k = 0}^\infty  {\sum\limits_{l =  - k - 1}^{k + 1} {\left[ {a_{1,k,l}^{(\alpha )}f_{k,l,j,\alpha }^{ - (66)n,m} + } \right.} } } \\
\left. { + a_{2,k,l}^{(\alpha )}\tilde f_{k,l,j,\alpha }^{ - (66)n,m}} \right] + \tilde W_{2,n,m}^{ - (k)0}({\tilde \xi _{j0}})\sum\limits_{\alpha  \ne j} {\sum\limits_{k = 0}^\infty  {\sum\limits_{l =  - k - 1}^{k + 1} {a_{2,k,l}^{(\alpha )}} } f_{k,l,j,\alpha }^{ - (66)n,m} + } \\
+ \tilde W_{3,n,m}^{ - (k)0}({\tilde \xi _{j0}})\sum\limits_{\alpha  \ne j} {\sum\limits_{k = 0}^\infty  {\sum\limits_{l =  - k - 1}^{k + 1} {a_{3,k,l}^{(\alpha )}} } f_{k,l,j,\alpha }^{ - (66)n,m}}  = \\
= \tilde W_{n,m}^{(k)j} + \sum\limits_{s = 1}^3 {b_{s,n,m}^{(j)}} \tilde W_{s,n,m}^{ - (k)j}({\tilde \xi _{j0}}),
\end{multline}
$$
n,m \in\mathbb{Z}:\quad n \ge 0,\quad |m| \le n + 1,\quad k =  - 1,{\mkern 1mu} {\kern 1pt} 0,{\mkern 1mu} {\kern 1pt} 1;
$$
\begin{multline}
\sum\limits_{s = 1}^3 {a_{s,n,m}^{(j)}} \tilde V_{s,n,m}^{ + (k)0}({\tilde \xi _{j0}}) + \tilde V_{1,n,m}^{ - (k)0}({\tilde \xi _{j0}})\sum\limits_{\alpha  \ne j} {\sum\limits_{k = 0}^\infty  {\sum\limits_{l =  - k - 1}^{k + 1} {\left[ {a_{1,k,l}^{(\alpha )}f_{k,l,j,\alpha }^{ - (66)n,m} + } \right.} } } \\
\left. { + a_{2,k,l}^{(\alpha )}\tilde f_{k,l,j,\alpha }^{ - (66)n,m}} \right] + \tilde V_{2,n,m}^{ - (k)0}({\tilde \xi _{j0}})\sum\limits_{\alpha  \ne j} {\sum\limits_{k = 0}^\infty  {\sum\limits_{l =  - k - 1}^{k + 1} {a_{2,k,l}^{(\alpha )}} } f_{k,l,j,\alpha }^{ - (66)n,m} + } \\
+ \tilde V_{3,n,m}^{ - (k)0}({\tilde \xi _{j0}})\sum\limits_{\alpha  \ne j} {\sum\limits_{k = 0}^\infty  {\sum\limits_{l =  - k - 1}^{k + 1} {a_{3,k,l}^{(\alpha )}} } f_{k,l,j,\alpha }^{ - (66)n,m} + } \\
+ \frac{T}{{2G}}\gamma {\rho _{j\alpha }}\left[ {{e^{ - i{\varphi _{j\alpha }}}}{\delta _{n0}}{\delta _{m1}}{\delta _{k, - 1}} + {e^{i{\varphi _{j\alpha }}}}{\delta _{n0}}{\delta _{m, - 1}}{\delta _{k1}}} \right] = \\
= \sum\limits_{s = 1}^3 {b_{s,n,m}^{(j)}} \tilde V_{s,n,m}^{ - (k)j}({\tilde \xi _{j0}});
\end{multline}
$$
n,m \in\mathbb{Z}:\quad n \ge 0,\quad |m| \le n + 1,\quad k =  - 1,{\mkern 1mu} {\kern 1pt} 0,{\mkern 1mu} {\kern 1pt} 1;
$$
$$
\tilde V_{1,n,m}^{ \pm ( - 1)r}(\tilde \xi ) = \tilde u_{n,m - 1}^{ \pm (6)}(\tilde \xi ),\quad \tilde V_{1,n,m}^{ \pm (1)r} =  - \tilde u_{n,m + 1}^{ \pm (6)}(\tilde \xi );
$$
$$
\tilde V_{1,n,m}^{ \pm (0)r}(\tilde \xi ) =  - \tilde u_{n,m}^{ \pm (6)}(\tilde \xi ),\quad \tilde V_{2,n,m}^{ \pm ( - 1)r}(\tilde \xi ) = i\bar q\tilde u_{1,n,m - 1}^{ \pm (6)}(\tilde \xi );
$$
$$
\tilde V_{2,n,m}^{ \pm (1)r}(\tilde \xi ) =  - i\bar q\tilde u_{1,n,m + 1}^{ \pm (6)}(\tilde \xi ),\quad \tilde V_{2,n,m}^{ \pm (0)r}(\tilde \xi ) =  - i\bar q\tilde u_{1,n,m}^{ \pm (6)}(\tilde \xi ) - {\chi _r}\tilde u_{n,m}^{ \pm (6)}(\tilde \xi );
$$
$$
\tilde V_{3,n,m}^{ \pm ( - 1)r}(\tilde \xi ) =  - \tilde u_{n,m - 1}^{ \pm (6)}(\tilde \xi ),\quad \tilde V_{3,n,m}^{ \pm (1)r}(\tilde \xi ) =  - \tilde u_{n,m + 1}^{ \pm (6)}(\tilde \xi ),\quad \tilde V_{3,n,m}^{ \pm (0)r}(\tilde \xi ) = 0;
$$
$$
{\chi _r} = 3 - 4{\sigma _r}.
$$

Коэффициенты $\tilde W_{s,n,m}^{ \pm (k)r}$ получают из коэффициентов $\tilde W_{s,n,m}^{ \pm (k)}$, определенных в формулах~\eqref{eq:9:15}~--- \eqref{eq:9:20b}, путем подстановки в последние вместо параметров $\sigma$ и $G$ упругих постоянных $\sigma_r$ и $G_r$ $(r=0,j)$ соответственно. При этом считается, что $\sigma_0=\sigma$, $G_0=G$.

\subsection{Анализ численных результатов для пространства со сжатыми сфероидальными включениями при одноосном и двуосном растяжениях}

Рассмотрим одноосное и двуосное растяжения упругого пространства с восьмью сжатыми сфероидальными включениями. Включения расположены в вершинах куба со стороной $a$ (см.~рис.~\ref{f:10:1o}). Считается, что включения находятся в условиях идеального контакта с упругим пространством.

\begin{figure}[h!]
\centering\footnotesize
\parbox[b]{7.5cm}{\centering\includegraphics[width=7.8cm]{oblate-inc8-a-d50-g25-t1-sig_x.pdf}
\caption{Напряжения $\sigma_x/T$ на линии $AB$ в зависимости от расстояния между включениями при одноосном растяжении
\label{f:10:14}}}\hfil\hfil
\parbox[b]{7.5cm}{\centering\includegraphics[width=7.5cm]{oblate-inc8-a-d50-g25-t2-sig_x.pdf}
\caption{Напряжения $\sigma_x/T$ на линии $AB$ в зависимости от расстояния между включениями при двуосном растяжении
\label{f:10:15}}}
\end{figure}

\begin{figure}[h!]
\centering\footnotesize
\parbox[b]{7.5cm}{\centering\includegraphics[width=7.6cm]{oblate-inc8-a-d50-g25-t1-sig_y.pdf}
\caption{Напряжения $\sigma_y/T$ на линии $AB$ в зависимости от расстояния между включениями при одноосном растяжении
\label{f:10:16}}}\hfil\hfil
\parbox[b]{7.5cm}{\centering\includegraphics[width=7.6cm]{oblate-inc8-a-d50-g25-t2-sig_y.pdf}
\caption{Напряжения $\sigma_y/T$ на линии $AB$ в зависимости от расстояния между включениями при двуосном растяжении
\label{f:10:17}}}
\end{figure}

При численном анализе полагаем коэффициент Пуассона материала упругого пространства $\sigma=0.38$, а включения~--- $\sigma_j=0.21$, отношение модулей сдвига $G_j/G=25$. Включения считаем одного размера, отношение полуосей сфероидов $d_1/d_2=0.5$. Разрешающую систему уравнений численно решают методом редукции. На основании полученных решений находят нормальные напряжения на площадках, параллельных координатным плоскостям.

На рис.~\ref{f:10:14}~--- \ref{f:10:19} приведены напряжения $\sigma_x/T$, $\sigma_y/T$ и $\sigma_z/T$ на линии $AB$ (см.~рис.~\ref{f:10:1o}) при одноосном и двуосном растяжениях упругого пространства в зависимости от относительного расстояния $a/d_1$ между полостями.

\begin{figure}[h!]
\centering\footnotesize
\parbox[b]{7.5cm}{\centering\includegraphics[width=7.6cm]{oblate-inc8-a-d50-g25-t1-sig_z.pdf}
\caption{Напряжения $\sigma_z/T$ на линии $AB$ в зависимости от расстояния между включениями при одноосном растяжении
\label{f:10:18}}}\hfil\hfil
\parbox[b]{7.5cm}{\centering\includegraphics[width=7.6cm]{oblate-inc8-a-d50-g25-t2-sig_z.pdf}
\caption{Напряжения $\sigma_z/T$ на линии $AB$ в зависимости от расстояния между включениями при двуосном растяжении
\label{f:10:19}}}
\end{figure}


\end{russian}
%!TeX root = book.tex
%!TEX TS-program = lualatex

\begin{russian}
\chapter[Механика упругого деформирования пространства с периодической системой полостей или включений]{Механика упругого деформирования пространства с~периодической системой полостей или включений}\chaptermark{Пространство с периодической системой полостей или включений}

\section[Упругое пространство с периодической системой сферических полостей]{Упругое пространство с периодической системой сферических полостей\sectionmark{Упругое пространство с периодической системой сферических полостей}}\sectionmark{Упругое пространство с периодической системой сферических полостей}

Рассмотрим упругое пространство $\Omega$ и бесконечную систему сферических полостей $\{\omega_{\alpha\beta\gamma}\}_{\alpha,\beta,\gamma=-\infty}^\infty$, центры которых расположены в узлах $\{O_{\alpha\beta\gamma}\}_{\alpha,\beta,\gamma=-\infty}^\infty$ кубической периодической решетки со стороной $a$. Декартовыми координатами узлов решетки будут упорядоченные наборы чисел $\{(\alpha a,\beta a,\gamma a);\,\alpha,\beta,\gamma\in\mathbb{Z}\}$. Радиусы полостей обозначим через $R$. Применим линейное упорядочение к узлам решетки, перенумеровав их согласно правилу\sloppy
$$
n(O_{\alpha_1,\beta_1,\gamma_1})>n(O_{\alpha_2,\beta_2,\gamma_2}),
$$
если 
\begin{multline}
\bigg\{|\alpha_1|+|\beta_1|+|\gamma_1|>|\alpha_2|+|\beta_2|+|\gamma_2|\bigg\}\bigvee \\
\bigvee\bigg\{\Big(|\alpha_1|+|\beta_1|+|\gamma_1|=|\alpha_2|+|\beta_2|+|\gamma_2|\Big)\bigwedge\Big((\alpha_1>\alpha_2)\bigvee \\
\bigvee(\alpha_1=\alpha_2,\beta_1>\beta_2)\bigvee(\alpha_1=\alpha_2,\beta_1=\beta_2,\gamma_1>\gamma_2)\Big)\bigg\}.
\end{multline}

\begin{figure}[h!]
\centering
\includegraphics[width=7cm]{cav-125.jpg}
\caption{Периодическая система сферических полостей в упругом пространстве}
\label{f:11:1f}
\end{figure}

Выше обозначен через $n(O_{\alpha_1,\beta_1,\gamma_1})$ номер узла $O_{\alpha_1,\beta_1,\gamma_1}$. В новой нумерации точка $O_{\alpha,\beta,\gamma}$ обозначается через $O_j$ (рис.~\ref{f:11:1}).

\begin{figure}[h!]
\centering
\includegraphics[width=12cm]{cartesian-spheres-periodic.pdf}
\caption{Схематическое представление задачи}
\label{f:11:1}
\end{figure}

С каждой точкой $O_j$ свяжем локальные декартовую $(x_j,y_j,z_j)$ и сонаправленную с ней сферическую системы координат $(r_j,\theta_j,\varphi_j)$. Считается, что декартовые системы координат с началами в точках $O_j$ сонаправлены.

Будем рассматривать задачу упругого деформирования пространства со сферическими полостями $\Omega\backslash\bigg\{\bigcup\limits_{\alpha,\beta,\gamma}\omega_{\alpha\beta\gamma}\bigg\}$ под действием нагрузки, приложенной на бесконечности (одноосное, двуосное или всестороннее растяжение упругого пространства). Сферические полости считаются свободными от нагрузки.

Соотношения между координатами можно описать формулами

\begin{equation*}
{x_i} = {r_i}\sin {\theta _i}\cos {\varphi _i},
\end{equation*}

\begin{equation}
{y_i} = {r_i}\sin {\theta _i}\sin {\varphi _i},
\label{eq:11:1}
\end{equation}

\begin{equation*}
{z_i} = {r_i}\cos {\theta _i},
\end{equation*}

\begin{equation}
\left\{ {\begin{array}{*{20}{l}}
{{x_j} = {x_\alpha } + {x_{j\alpha }},}\\
{{y_j} = {y_\alpha } + {y_{j\alpha }},}\\
{{z_j} = {z_\alpha } + {z_{j\alpha }},}
\end{array}} \right.\qquad {\kern 1pt} j \ne \alpha ,\quad j,\alpha  = \overline {1,N},
\label{eq:11:2}
\end{equation}

\noindent где $\overrightarrow {{O_j}{O_\alpha }}  = \left( {{x_{j\alpha }},{y_{j\alpha }},{z_{j\alpha }}} \right) = \left( {{r_{j\alpha }},{\theta _{j\alpha }},{\varphi _{j\alpha }}} \right)$.

Для определения НДС в рассматриваемом теле необходимо решить краевую задачу для уравнения Ламе относительно неизвестного вектора перемещения   $\mathbf{U}$ с граничными условиями

\begin{equation}
{\bf{FU}}{|_{{\Gamma _j}}} = 0
\end{equation}

\noindent и условиями на бесконечности одного из трех типов

\begin{equation}
{\bf{FU}}{|_{z =  \pm \infty }} =  \pm T{{\bf{e}}_z}\quad\text{(одноосное растяжение)},
\label{eq:11:1a}
\end{equation}

\begin{equation}
{\bf{FU}}{|_{\rho  = \infty }} = T{{\bf{e}}_\rho }\quad\text{(двуосное растяжение)},
\label{eq:11:2a}
\end{equation}

\begin{equation}
{\bf{FU}}{|_{r  = \infty }} = T{{\bf{e}}_r }\quad\text{(всестороннее растяжение)},
\label{eq:11:3a}
\end{equation}

\noindent где $\mathbf{FU}$~--- отвечающий перемещению $\mathbf{U}$ вектор усилий на соответствующей граничной поверхности, ${\Gamma _j} = \left\{ {\left( {{r_j},{\theta _j},{\varphi _j}} \right):\,{r_j} = {R_j}} \right\}$.

Решение задачи будем искать в виде

\begin{equation}
{\bf{U}} = {\bf{\tilde U}} + {{\bf{U}}_0},
\label{eq:11:8s}
\end{equation}

\begin{equation}
{\bf{\tilde U}} = \sum\limits_{j = 1}^\infty {\sum\limits_{s = 1}^3 {\sum\limits_{n = 0}^\infty  {\sum\limits_{m=-n}^{n} {a_{s,n,m}^{(j)}} } } } {\bf{\tilde U}}_{s,n,m}^{ + (4)}\left( {{r_j},{\theta _j},{\varphi _j}} \right),
\label{eq:11:8a}
\end{equation}

\begin{equation}
{{\bf{U}}_0} = \frac{T}{{2G}}\left( { - \frac{\sigma }{{1 + \sigma }}{\rho}{{\bf{e}}_{{\rho}}} + \frac{1}{{1 + \sigma }}{z}{{\bf{e}}_z}} \right)\,\text{(одноосное растяжение)},
\label{eq:11:8}
\end{equation}

\begin{equation}
{{\bf{U}}_0} = \frac{T}{{2G}}\left( {\frac{{1 - \sigma }}{{1 + \sigma }}{\rho}{{\bf{e}}_{{\rho}}} - \frac{{2\sigma }}{{1 + \sigma }}{z}{{\bf{e}}_z}} \right)\,\text{(двуосное растяжение)},
\label{eq:11:9}
\end{equation}

\begin{equation}
{{\bf{U}}_0} = \frac{T}{2G}\frac{1-2\sigma}{1+\sigma}\left(\rho\mathbf{e}_\rho+z\mathbf{e}_z\right)\,\text{(всестороннее растяжение)},
\label{eq:11:9a}
\end{equation}

\noindent где $G$, $\sigma$~--- модуль сдвига и коэффициент Пуассона упругого пространства; $a_{s,n,m}^{(j)}$~--- неизвестные коэффициенты; перемещения $\mathbf{\tilde U}_{s,n,m}^{+(4)}$ приведены в формулах~\eqref{eq:1:89b}~--- \eqref{eq:1:94b}.

Относительно перемещения $\mathbf{\tilde U}$ граничные условия записываем следующим образом:

\begin{equation}
{\bf{F\tilde U}}{|_{{\Gamma _j}}} =  - {\bf{F}}{{\bf{U}}_0}{|_{{\Gamma _j}}};
\label{eq:11:5}
\end{equation}

\begin{equation}
{\bf{F\tilde U}}{|_{z =  \pm \infty }} = 0\quad\text{(одноосное растяжение)},
\label{eq:11:3}
\end{equation}

\begin{equation}
{\bf{F\tilde U}}{|_{\rho  = \infty }} = 0\quad\text{(двуосное растяжение)},
\label{eq:11:4}
\end{equation}

\begin{equation}
{\bf{F\tilde U}}{|_{r  = \infty }} = 0\quad\text{(всестороннее растяжение)}.
\label{eq:11:4a}
\end{equation}

Вспомогательным перемещениям $\mathbf{U}_0$ отвечают следующие напряжения на поверхностях $\Gamma_j$ ($\mathbf{n}_j=\mathbf{e}_{r_j}$~--- вектор нормали на поверхности $\Gamma_j$):

\begin{equation}
{\bf{F}}{{\bf{U}}_0} = T{P_1}(\cos {\theta _j}){{\bf{e}}_z}\quad\text{(одноосное растяжение)},
\end{equation}

\begin{equation}
{\bf{F}}{{\bf{U}}_0} =  - TP_1^{(1)}(\cos {\theta _j}){{\bf{e}}_{{\rho _j}}}\quad\text{(двуосное растяжение)},
\end{equation}

\begin{multline}
\mathbf{FU}_0=T\bigg[-P_1^{(1)}(\cos\theta_j)\mathbf{e}_{\rho_j}+ \\
+P_1(\cos\theta_j)\mathbf{e}_z\bigg]\,\text{(всестороннее растяжение)}.
\end{multline}

Используя теоремы сложения~\eqref{eq:1:99t}, перемещение $\mathbf{\tilde U}$ можно записать полностью в системе координат с началом в точке $O_j$:

\begin{multline}
{\bf{\tilde U}} = \sum\limits_{s = 1}^3 {\sum\limits_{n = 0}^\infty  {\sum\limits_{m =  - n }^{n } {a_{s,n,m}^{(j)}} } } {\bf{\tilde U}}_{s,n,m}^{ +(4)}\left( {{r_j},{\theta _j},{\varphi _j}} \right) + \\
+ \sum\limits_{s=1}^3\sum\limits_{n = 0}^\infty\sum\limits_{m =  - n }^{n }{\bf{\tilde U}}_{s,n,m}^{ - (4)}\left( {{r_j},{\theta _j},{\varphi _j}} \right)\sum\limits_{\alpha  \ne j}\sum\limits_{t=1}^3\sum\limits_{k = 0}^\infty\mathop \sum \limits_{l =  - k}^{k}\tilde T_{t,k,l,\alpha}^{s,n,m,j}a_{t,k,l}^{(\alpha)}.
\label{eq:11:10}
\end{multline}

%Формулы для напряжений, отвечающих базисным функциям перемещений $\mathbf{U}_{s,n,m}^{\pm(4)}(r_j,\theta_j,\varphi_j)$ на сферических поверхностях $\Gamma_j$ ($\mathbf{n}_j=\mathbf{e}_{\rho_j}$, $\mathbf{n}_j$~--- нормаль к поверхности $\Gamma_j$):
%
%\begin{multline}
%{\bf{FU}}_{1,n,m}^{ + (4)}\left( {{r_j},{\theta _j},{\varphi _j}} \right) =  - \frac{{2G}}{{{r_j}}}(n + 1)\left[ { - u_{n,m - 1}^{ + (4)}\left( {{r_j},{\theta _j},{\varphi _j}} \right){{\bf{e}}_{ - 1}} + } \right.\\
%\left. { + u_{n,m + 1}^{ + (4)}\left( {{r_j},{\theta _j},{\varphi _j}} \right){{\bf{e}}_1} - u_{n,m}^{ + (4)}\left( {{r_j},{\theta _j},{\varphi _j}} \right){{\bf{e}}_0}} \right],
%\end{multline}
%
%\begin{multline}
%{\bf{FU}}_{2,n,m}^{ + (4)}\left( {{r_j},{\theta _j},{\varphi _j}} \right) = \frac{{2G}}{{{r_j}}}\left\{ {(n + m)\left[ {\frac{{(n + 3)(n - m + 2)}}{{2n + 3}} - 2\sigma } \right] \times } \right.\\
%\times u_{n,m - 1}^{ + (4)}\left( {{r_j},{\theta _j},{\varphi _j}} \right){{\bf{e}}_{ - 1}} - \\
%- (n - m)\left[ {\frac{{(n + 3)(n + m + 2)}}{{2n + 3}} - 2\sigma } \right]u_{n,m + 1}^{ + (4)}\left( {{r_j},{\theta _j},{\varphi _j}} \right){{\bf{e}}_1} + \\
%+ \left[ {\frac{{(n + 3)(n - m + 1)(n + m + 1)}}{{2n + 3}} - (n + 1)(2\sigma  - 1)} \right] \times \\
%\times u_{n,m}^{ + (4)}\left( {{r_j},{\theta _j},{\varphi _j}} \right){{\bf{e}}_0} \bigg\},
%\end{multline}
%
%\begin{multline}
%{\bf{FU}}_{3,n,m}^{ + (4)}\left( {{r_j},{\theta _j},{\varphi _j}} \right) =  - \frac{{2G}}{{{r_j}}}\left[ {(n - m + 2)u_{n,m - 1}^{ + (4)}\left( {{r_j},{\theta _j},{\varphi _j}} \right){{\bf{e}}_{ - 1}} + } \right.\\
%\left. { + (n + m + 2)u_{n,m + 1}^{ + (4)}\left( {{r_j},{\theta _j},{\varphi _j}} \right){{\bf{e}}_1} - mu_{n,m}^{ + (4)}\left( {{r_j},{\theta _j},{\varphi _j}} \right){{\bf{e}}_0}} \right],
%\end{multline}
%
%\begin{multline}
%{\bf{FU}}_{1,n,m}^{ - (4)}\left( {{r_j},{\theta _j},{\varphi _j}} \right) = \frac{{2G}}{{{r_j}}}n\bigg[ - u_{n,m - 1}^{ - (4)}\left( {{r_j},{\theta _j},{\varphi _j}} \right){{\bf{e}}_{ - 1}} + \\
%+ u_{n,m + 1}^{ - (4)}\left( {{r_j},{\theta _j},{\varphi _j}} \right){{\bf{e}}_1} + u_{n,m}^{ - (4)}\left( {{r_j},{\theta _j},{\varphi _j}} \right){{\bf{e}}_0} \bigg],
%\end{multline}
%
%\begin{multline}
%{\bf{FU}}_{2,n,m}^{ - (4)}\left( {{r_j},{\theta _j},{\varphi _j}} \right) = \\
%= \frac{{2G}}{{{r_j}}}\left\{ { - (n - m + 1)\left[ {\frac{{(n - 2)(n + m - 1)}}{{2n - 1}} + 2\sigma } \right] \times } \right.\\
%\times u_{n,m - 1}^{ - (4)}\left( {{r_j},{\theta _j},{\varphi _j}} \right){{\bf{e}}_{ - 1}} + (n + m + 1)\left[ {\frac{{(n - 2)(n - m - 1)}}{{2n - 1}} + 2\sigma } \right] \times \\
%\times u_{n,m + 1}^{ + (4)}\left( {{r_j},{\theta _j},{\varphi _j}} \right){{\bf{e}}_1} + \\
%\left. { + \left[ {\frac{{(n - 2)(n - m)(n + m)}}{{2n - 1}} + n(2\sigma  - 1)} \right]u_{n,m}^{ - (4)}\left( {{r_j},{\theta _j},{\varphi _j}} \right){{\bf{e}}_0}} \right\},
%\end{multline}
%
%\begin{multline}
%{\bf{FU}}_{3,n,m}^{ - (4)}\left( {{r_j},{\theta _j},{\varphi _j}} \right) = \frac{{2G}}{{{r_j}}}\left[ {(n + m - 1)u_{n,m - 1}^{ - (4)}\left( {{r_j},{\theta _j},{\varphi _j}} \right){{\bf{e}}_{ - 1}} + } \right.\\
%\left. { + (n - m - 1)u_{n,m + 1}^{ - (4)}\left( {{r_j},{\theta _j},{\varphi _j}} \right){{\bf{e}}_1} - mu_{n,m}^{ - (4)}\left( {{r_j},{\theta _j},{\varphi _j}} \right){{\bf{e}}_0}} \right].
%\end{multline}

В силу периодичности задачи вклад каждого слагаемого в формуле~\eqref{eq:11:8a} будет одинаковым, поэтому можно считать, что $a_{s,n,m}^{(j)}=a_{s,n,m}$.

После перехода в формуле~\eqref{eq:11:10} к напряжениям и удовлетворения граничным условиям относительно неизвестных $a_{s,n,m}$ получаем бесконечную систему линейных алгебраических уравнений:

\begin{equation}
\sum\limits_{s=1}^3 a_{s,n,m}\tilde F_{s,n,m}^{+(k)}+\tilde F_{s,n,m}^{-(k)}\sum\limits_{t=1}^3\sum\limits_{k=0}^\infty\sum\limits_{l=-k}^k a_{t,k,l}\sum\limits_{\alpha\neq j}\tilde T_{t,k,l,\alpha}^{s,n,m,j}=F_{n,m}^{(k)};
\label{eq:11:sys}
\end{equation}
$$
n,m \in\mathbb{Z}:\quad n \ge 0,{\mkern 1mu} \quad {\kern 1pt} |m| \le n,\quad {\mkern 1mu} k =  - 1,{\mkern 1mu} {\kern 1pt} 0,{\mkern 1mu} {\kern 1pt} 1;\;\;\;{\mkern 1mu} {\kern 1pt} j = \overline {1,\infty},
$$

\noindent где $\tilde F_{s,n,m}^{\pm(k)}$~--- компоненты вектора напряжений на поверхности $r_j=R$, отвечающего вектору перемещения $\tilde U_{s,n,m}^{\pm(4)}$:

$$
\mathbf{F\tilde U}_{s,n,m}^{\pm(4)}=\tilde F_{s,n,m}^{\pm(-1)}S_n^{m-1}\mathbf{e}_{-1}+\tilde F_{s,n,m}^{\pm(1)}S_n^{m+1}\mathbf{e}_1+\tilde F_{s,n,m}^{\pm(0)}S_n^m\mathbf{e}_0;
$$

\begin{equation*}
F_{n,m}^{(k)} =  -\frac{T}{2G}{\delta _{n1}}{\delta _{m0}}{\delta _{k0}}\quad\text{(одноосное растяжение)},
\label{eq:11:19}
\end{equation*}

\begin{equation*}
F_{n,m}^{(k)} =  -\frac{T}{2G}{\delta _{n1}}{\delta _{m0\,}}(2{\delta _{k, - 1}} - {\delta _{k1}})\quad\text{(двуосное растяжение)},
\label{eq:11:20}
\end{equation*}

\begin{equation*}
F_{n,m}^{(k)} =  -\frac{T}{2G}{\delta _{n1}}{\delta _{m0\,}}(\delta_{k0}+2{\delta _{k, - 1}} - {\delta _{k1}})\quad\text{(всестороннее растяжение)}.
\label{eq:11:21}
\end{equation*}

Явный вид компонент $\tilde F_{s,n,m}^{\pm(k)}$ не приводится ввиду их громоздкости. Они получаются из формул~\eqref{eq:1:89b}~--- \eqref{eq:1:99b}, \eqref{eq:8:f1}~--- \eqref{eq:8:f6}.

На рис.~\ref{f:11:4}~--- \ref{f:11:9} приведены нормальные напряжения на линии $AB$ в зависимости от расстояния между полостями при одноосном и двуосном растяжениях упругого пространства с периодической системой сферических полостей.

При одноосном растяжении основной вклад в тензор напряжений вносят напряжения $\sigma_z/T$. Областью их концентрации являются границы полостей. С приближением полостей друг к другу напряжения $\sigma_z/T$ возрастают.

При двуосном растяжении основной вклад в тензор напряжений вносят напряжения $\sigma_y/T$. Областью их концентрации являются границы полостей. С приближением полостей друг к другу напряжения $\sigma_y/T$ возрастают.

\begin{figure}[h!]
\centering\footnotesize
\parbox[b]{7.5cm}{\centering\includegraphics[width=7.6cm]{periodic-spheres-cav27-a-t1-sig_x.pdf}
\caption{Напряжения $\sigma_x/T$ на линии $AB$ в зависимости от расстояния между полостями при одноосном растяжении
\label{f:11:4}}}\hfil\hfil
\parbox[b]{7.5cm}{\centering\includegraphics[width=7.6cm]{periodic-spheres-cav27-a-t2-sig_x.pdf}
\caption{Напряжения $\sigma_x/T$ на линии $AB$ в зависимости от расстояния между полостями при двуосном растяжении
\label{f:11:5}}}
\end{figure}

\begin{figure}[h!]
\centering\footnotesize
\parbox[b]{7.5cm}{\centering\includegraphics[width=7.6cm]{periodic-spheres-cav27-a-t1-sig_y.pdf}
\caption{Напряжения $\sigma_y/T$ на линии $AB$ в зависимости от расстояния между полостями при одноосном растяжении
\label{f:11:6}}}\hfil\hfil
\parbox[b]{7.5cm}{\centering\includegraphics[width=7.6cm]{periodic-spheres-cav27-a-t2-sig_y.pdf}
\caption{Напряжения $\sigma_y/T$ на линии $AB$ в зависимости от расстояния между полостями при двуосном растяжении
\label{f:11:7}}}
\end{figure}

\begin{figure}[h!]
\centering\footnotesize
\parbox[b]{7.5cm}{\centering\includegraphics[width=7.6cm]{periodic-spheres-cav27-a-t1-sig_z.pdf}
\caption{Напряжения $\sigma_z/T$ на линии $AB$ в зависимости от расстояния между полостями при одноосном растяжении
\label{f:11:8}}}\hfil\hfil
\parbox[b]{7.5cm}{\centering\includegraphics[width=7.8cm]{periodic-spheres-cav27-a-t2-sig_z.pdf}
\caption{Напряжения $\sigma_z/T$ на линии $AB$ в зависимости от расстояния между полостями при двуосном растяжении
\label{f:11:9}}}
\end{figure}

\begin{table}[h!]
\centering
\caption{\centering Сравнение напряжений для разного количества полостей периодической~структуры}
$
\begin{array}{|c|c|c|c|}
\hline
\text{Кол-во полостей} & \sigma_x/T & \sigma_y/T & \sigma_z/T \\
\hline
27 & 0.27904 & 0.10871 & 1.27536 \\
\hline
125 & 0.27934 & 0.11061 & 1.27775 \\
\hline
\end{array}
$
\label{t:11:1}
\end{table}

В табл.~\ref{t:11:1} приведено сравнение нормальных напряжений в средней точке отрезка $AB$ для разного количества (27 и 125) полостей периодической структуры. Из таблицы видно, что увеличение числа полостей практически не меняет значения напряжений.

\begin{figure}[h!]
\centering\footnotesize
\parbox[b]{7.5cm}{\centering\includegraphics[width=7.8cm]{spheres-cav27-8-a30-t1.pdf}
\caption{Сравнение напряжения на линии $AB$ для периодической и тетрагональной структур при одноосном растяжении
\label{f:11:10}}}\hfil\hfil
\parbox[b]{7.5cm}{\centering\includegraphics[width=7.8cm]{spheres-cav27-8-a30-t2.pdf}
\caption{Сравнение напряжения на линии $AB$ для периодической и тетрагональной структур при двуосном растяжении
\label{f:11:11}}}
\end{figure}

На рис.~\ref{f:11:10}, \ref{f:11:11} представлено сравнение нормальных напряжений на линии $AB$ для периодической (27 полостей) и тетрагональной (8 полостей) структур при одноосном и двуосном растяжениях упругого пространства. Графики показывают, что все рассмотренные пары напряжений практически не отличаются друг от друга.

\section[Упругое пространство с периодической системой сферических включений]{Упругое пространство с периодической системой сферических включений\sectionmark{Упругое пространство с периодической системой сферических включений}}\sectionmark{Упругое пространство с периодической системой сферических включений}

Рассмотрим упругое пространство $\Omega$ и бесконечную систему сферических включений $\{\omega_{\alpha\beta\gamma}\}_{\alpha,\beta,\gamma=-\infty}^\infty$, центры которых расположены в узлах $\{O_{\alpha\beta\gamma}\}_{\alpha,\beta,\gamma=-\infty}^\infty$ кубической периодической решетки со стороной $a$. Декартовыми координатами узлов решетки будут упорядоченные наборы чисел $\{(\alpha a,\beta a,\gamma a);\,\alpha,\beta,\gamma\in\mathbb{Z}\}$. Радиусы полостей обозначим через $R$. Выполним линейное упорядочение узлов таким же образом, как в предыдущем параграфе.\sloppy

В новой нумерации точка $O_{\alpha,\beta,\gamma}$ обозначается через $O_j$ (см.~рис.~\ref{f:11:1}). С каждой точкой $O_j$ свяжем локальные декартовую $(x_j,y_j,z_j)$ и сонаправленную с ней сферическую системы координат $(r_j,\theta_j,\varphi_j)$. Считается, что декартовые системы координат с началами в точках $O_j$ сонаправлены.

Будем рассматривать задачу упругого деформирования пространства со сферическими включениями под действием нагрузки, приложенной на бесконечности (одноосное, двуосное или всестороннее растяжения упругого пространства).

Соотношения между координатами можно описать формулами~\eqref{eq:11:1}, \eqref{eq:11:2}.

Предполагается, что упругие постоянные включений равны $(\sigma_j,G_j)$. Упругие постоянные матрицы будем считать равными $(\sigma,G)$.

Граничные условия в рассматриваемой задаче представляют собой условия сопряжения полей перемещений и напряжений на поверхностях $\Gamma_j$. Для того, чтобы их записать, представим вектор перемещений в упругом пространстве в виде

\begin{equation}
{\bf{U}} = \left\{ {\begin{array}{*{20}{l}}
{{\bf{\tilde U}}_j^ - ,\quad \left( {x,y,z} \right) \in \omega_j,}\\
{{{{\bf{\tilde U}}}^ + } + {{\bf{U}}_0},\quad \left( {x,y,z} \right) \in\Omega\backslash {\bigcup\limits_j\omega_j},}
\end{array}} \right.
\label{eq:11:23}
\end{equation}

\noindent где $\omega_j = \left\{ {\left( {{r_j},{\theta _j},{\varphi _j}} \right):\, {r_j} < {R_j}} \right\}$. Тогда условия сопряжения принимают следующий вид:

\begin{equation}
\left( {{{{\bf{\tilde U}}}^ + } + {{\bf{U}}_0}} \right){|_{{\Gamma _j}}} = {\bf{\tilde U}}_j^ - {|_{{\Gamma _j}}};
\label{eq:11:24}
\end{equation}

\begin{equation}
\left( {{\bf{F\tilde U^+}} + {\bf{F}}{{\bf{U}}_0}} \right){|_{{\Gamma _j}}} = {\bf{F}}{{\bf{\tilde U^-}}_j}{|_{{\Gamma _j}}},\qquad {\kern 1pt} j = 1,{\mkern 1mu} {\kern 1pt} 2,\dots.
\label{eq:11:25}
\end{equation}

\noindent Условия на бесконечности задаются формулами~\eqref{eq:11:1a}~--- \eqref{eq:11:3a}.

Решение задачи будем искать в виде

\begin{equation}
{\bf{\tilde U^+}} = \sum\limits_{j = 1}^\infty {\sum\limits_{s = 1}^3 {\sum\limits_{n = 0}^\infty  {\sum\limits_{m=-n}^{n} {a_{s,n,m}^{(j)}} } } } {\bf{\tilde U}}_{s,n,m}^{ + (4)}\left( {{r_j},{\theta _j},{\varphi _j}} \right),
\label{eq:11:26}
\end{equation}

\begin{equation}
\mathbf{\tilde U}_j^- = {\sum\limits_{s = 1}^3 {\sum\limits_{n = 0}^\infty  {\sum\limits_{m=-n}^{n} {b_{s,n,m}^{(j)}} } } } {\bf{\tilde U}}_{s,n,m}^{ - (4)}\left( {{r_j},{\theta _j},{\varphi _j}} \right),
\label{eq:11:27}
\end{equation}

\noindent где $G$, $\sigma$~--- модуль сдвига и коэффициент Пуассона упругого пространства; $a_{s,n,m}^{(j)}$, $b_{s,n,m}^{(j)}$~--- неизвестные коэффициенты; перемещения $\mathbf{\tilde U}_{s,n,m}^{\pm(4)}$ приведены в формулах~\eqref{eq:1:89b}~--- \eqref{eq:1:99b}.

Для представления вектора перемещений $\mathbf{\tilde U}^+$ в системах координат с началами $O_j$ можем использовать формулу~\eqref{eq:11:10}.

В силу периодичности задачи вклад каждого слагаемого в формулах~\eqref{eq:11:26}, \eqref{eq:11:27} будет одинаковым, поэтому можно считать, что $a_{s,n,m}^{(j)}=\\=a_{s,n,m}$, $b_{s,n,m}^{(j)}=b_{s,n,m}$.

После удовлетворения условиям~\eqref{eq:11:24}, \eqref{eq:11:25} получаем бесконечную систему линейных алгебраических уравнений относительно неизвестных $a_{s,n,m}$, $b_{s,n,m}$:

\begin{multline}
\sum\limits_{s=1}^3 a_{s,n,m}\tilde E_{s,n,m}^{+(k)}(\sigma)+\tilde E_{s,n,m}^{-(k)}(\sigma)\sum\limits_{t=1}^3\sum\limits_{k=0}^\infty\sum\limits_{l=-k}^k a_{t,k,l}\sum\limits_{\alpha\neq j}\tilde T_{t,k,l,\alpha}^{s,n,m,j}= \\
=-E_{n,m}^{(k)}+\sum\limits_{s=1}^3 b_{s,n,m}\tilde E_{s,n,m}^{-(k)}(\sigma_j);
\label{eq:11:28}
\end{multline}

\begin{multline}
\sum\limits_{s=1}^3 a_{s,n,m}\tilde F_{s,n,m}^{+(k)}(\sigma)+\tilde F_{s,n,m}^{-(k)}(\sigma)\sum\limits_{t=1}^3\sum\limits_{k=0}^\infty\sum\limits_{l=-k}^k a_{t,k,l}\sum\limits_{\alpha\neq j}\tilde T_{t,k,l,\alpha}^{s,n,m,j}= \\
=F_{n,m}^{(k)}+\frac{G_j}{G}\sum\limits_{s=1}^3 b_{s,n,m}\tilde F_{s,n,m}^{-(k)}(\sigma_j);
\label{eq:11:29}
\end{multline}
$$
n,m \in\mathbb{Z} :\quad n \ge 0,\quad |m| \le n,\quad k =  - 1,{\mkern 1mu} {\kern 1pt} 0,{\mkern 1mu} {\kern 1pt} 1,
$$
где $\tilde E_{s,n,m}^{\pm(k)}$~--- компоненты вектора перемещений на поверхности $r_j=R$; $\tilde F_{s,n,m}^{\pm(k)}$~--- компоненты вектора напряжений на поверхности $r_j=R$, отвечающего вектору перемещения $\tilde U_{s,n,m}^{\pm(4)}$:

$$
\mathbf{\tilde U}_{s,n,m}^{\pm(4)}=\tilde E_{s,n,m}^{\pm(-1)}S_n^{m-1}\mathbf{e}_{-1}+\tilde E_{s,n,m}^{\pm(1)}S_n^{m+1}\mathbf{e}_1+\tilde E_{s,n,m}^{\pm(0)}S_n^m\mathbf{e}_0;
$$

$$
\mathbf{F\tilde U}_{s,n,m}^{\pm(4)}=\tilde F_{s,n,m}^{\pm(-1)}S_n^{m-1}\mathbf{e}_{-1}+\tilde F_{s,n,m}^{\pm(1)}S_n^{m+1}\mathbf{e}_1+\tilde F_{s,n,m}^{\pm(0)}S_n^m\mathbf{e}_0;
$$

\begin{equation*}
E_{n,m}^{(k)} =\frac{TR}{2G}\delta_{n1}\delta_{m0}\bigg[-\frac{2\sigma}{1+\sigma}\delta_{k,-1}+\frac{\sigma}{1+\sigma}\delta_{k1}+\frac{1}{1+\sigma}\delta_{k0}\bigg],
\label{eq:11:19a}
\end{equation*}

\begin{equation*}
F_{n,m}^{(k)} =  -\frac{T}{2G}{\delta _{n1}}{\delta _{m0}}{\delta _{k0}}\quad\text{(одноосное растяжение)};
\label{eq:11:19}
\end{equation*}

\begin{equation*}
E_{n,m}^{(k)} =\frac{TR}{2G}\delta_{n1}\delta_{m0}\bigg[\frac{2-2\sigma}{1+\sigma}\delta_{k,-1}-\frac{1-\sigma}{1+\sigma}\delta_{k1}-\frac{2\sigma}{1+\sigma}\delta_{k0}\bigg],
\label{eq:11:20a}
\end{equation*}

\begin{equation*}
F_{n,m}^{(k)} =  -\frac{T}{2G}{\delta _{n1}}{\delta _{m0\,}}(2{\delta _{k, - 1}} - {\delta _{k1}})\quad\text{(двуосное растяжение)};
\label{eq:11:20}
\end{equation*}

\begin{equation*}
E_{n,m}^{(k)} =\frac{TR}{2G}\delta_{n1}\delta_{m0}\frac{1-2\sigma}{1+\sigma}\bigg[2\delta_{k,-1}-\delta_{k1}+\delta_{k0}\bigg],
\label{eq:11:21a}
\end{equation*}

\begin{equation*}
F_{n,m}^{(k)} =  -\frac{T}{2G}{\delta _{n1}}{\delta _{m0\,}}(\delta_{k0}+2{\delta _{k, - 1}} - {\delta _{k1}})\quad\text{(всестороннее растяжение)}.
\label{eq:11:21}
\end{equation*}

Явный вид компонент $\tilde E_{s,n,m}^{\pm(k)}$ и $\tilde F_{s,n,m}^{\pm(k)}$ не приводится ввиду их громоздкости. Они получаются из формул~\eqref{eq:1:91}~--- \eqref{eq:1:94}, \eqref{eq:1:89b}~--- \eqref{eq:1:99b}, \eqref{eq:8:f1}~--- \eqref{eq:8:f6}.

\begin{figure}[h!]
\centering\footnotesize
\parbox[b]{7.5cm}{\centering\includegraphics[width=8cm]{periodic-spheres-inc27-a-g25-t1-sig_x.pdf}
\caption{Напряжения $\sigma_x/T$ на линии $AB$ в зависимости от расстояния между включениями при одноосном растяжении
\label{f:11:12}}}\hfil\hfil
\parbox[b]{7.5cm}{\centering\includegraphics[width=7.6cm]{periodic-spheres-inc27-a-g25-t2-sig_x.pdf}
\caption{Напряжения $\sigma_x/T$ на линии $AB$ в зависимости от расстояния между включениями при двуосном растяжении
\label{f:11:13}}}
\end{figure}

\begin{figure}[h!]
\centering\footnotesize
\parbox[b]{7.5cm}{\centering\includegraphics[width=7.6cm]{periodic-spheres-inc27-a-g25-t1-sig_y.pdf}
\caption{Напряжения $\sigma_y/T$ на линии $AB$ в зависимости от расстояния между включениями при одноосном растяжении
\label{f:11:14}}}\hfil\hfil
\parbox[b]{7.5cm}{\centering\includegraphics[width=7.6cm]{periodic-spheres-inc27-a-g25-t2-sig_y.pdf}
\caption{Напряжения $\sigma_y/T$ на линии $AB$ в зависимости от расстояния между включениями при двуосном растяжении
\label{f:11:15}}}
\end{figure}

\begin{figure}[h!]
\centering\footnotesize
\parbox[b]{7.5cm}{\centering\includegraphics[width=7.5cm]{periodic-spheres-inc27-a-g25-t1-sig_z.pdf}
\caption{Напряжения $\sigma_z/T$ на линии $AB$ в зависимости от расстояния между включениями при одноосном растяжении
\label{f:11:16}}}\hfil\hfil
\parbox[b]{7.5cm}{\centering\includegraphics[width=7.5cm]{periodic-spheres-inc27-a-g25-t2-sig_z.pdf}
\caption{Напряжения $\sigma_z/T$ на линии $AB$ в зависимости от расстояния между включениями при двуосном растяжении
\label{f:11:17}}}
\end{figure}

На рис.~\ref{f:11:12}~--- \ref{f:11:17} представлены нормальные напряжения на линии $AB$ в зависимости от расстояния между включениями при одноосном и двуосном растяжениях упругого пространства с периодической системой сферических включений.

При одноосном растяжении основной вклад в тензор напряжений вносят напряжения $\sigma_x/T$ и $\sigma_z/T$. Областью их концентрации является середина отрезка $AB$, где напряжения $\sigma_z/T$~--- растягивающие, а $\sigma_x/T$~--- сжимающие. С приближением включений друг к другу напряжения $\sigma_x/T$ и $\sigma_z/T$ убывают.

При двуосном растяжении основной вклад в тензор напряжений вносят напряжения $\sigma_x/T$, при этом напряжения $\sigma_y/T$ и $\sigma_z/T$ тоже значимы. С приближением включений друг к другу все нормальные напряжения возрастают.

\begin{figure}[h!]
\centering\footnotesize
\parbox[b]{7.5cm}{\centering\includegraphics[width=8cm]{spheres-inc27-8-a25-g25-t1.pdf}
\caption{Сравнение напряжений на линии $AB$ для периодической и тетрагональной структур при одноосном растяжении
\label{f:11:18}}}\hfil\hfil
\parbox[b]{7.5cm}{\centering\includegraphics[width=8cm]{spheres-inc27-8-a25-g25-t2.pdf}
\caption{Сравнение напряжений на линии $AB$ для периодической и тетрагональной структур при двуосном растяжении
\label{f:11:19}}}
\end{figure}

На рис.~\ref{f:11:18}, \ref{f:11:19} приведено сравнение нормальных напряжений на линии $AB$ для периодической (36 включений) и тетрагональной (8 включений) структур при одноосном и двуосном растяжении упругого пространства. Графики показывают, что напряжения $\sigma_y/T$ и $\sigma_z/T$ практически совпадают. Небольшое отличие наблюдается в значениях напряжений $\sigma_x/T$ при сохранении общего характера в их распределении.


%***************************************************************************************

\section[Упругое пространство с периодической системой вытянутых сфероидальных полостей]{Упругое пространство с периодической системой вытянутых сфероидальных полостей\sectionmark{Упругое пространство с периодической системой сфероидальных полостей}}\sectionmark{Упругое пространство с периодической системой сфероидальных полостей}

Рассмотрим упругое пространство $\Omega$ и бесконечную систему вытянутых сфероидальных полостей $\{\omega_{\alpha\beta\gamma}\}_{\alpha,\beta,\gamma=-\infty}^\infty$, центры которых расположены в узлах $\{O_{\alpha\beta\gamma}\}_{\alpha,\beta,\gamma=-\infty}^\infty$ кубической периодической решетки со стороной $a$. Декартовыми координатами узлов решетки будут упорядоченные наборы чисел $\{(\alpha a,\beta a,\gamma a);\,\alpha,\beta,\gamma\in\mathbb{Z}\}$. Полуоси полостей обозначим через $d_1$ и $d_2$ ($d_1>d_2$). Выполним линейное упорядочение узлов таким же образом, как в параграфе~6.1.\sloppy

В новой нумерации точка $O_{\alpha,\beta,\gamma}$ обозначается через $O_j$ (рис.~\ref{f:11:1a}).

\begin{figure}[h!]
\centering
\includegraphics[width=10cm]{cartesian-spheroids-periodic.pdf}
\caption{Периодическая система сфероидальных полостей в упругом пространстве}
\label{f:11:1a}
\end{figure}

С каждой точкой $O_j$ свяжем локальные декартовую $(x_j,y_j,z_j)$ и сонаправленную с ней вытянутую сфероидальную системы координат $(\xi_j,\eta_j,\varphi_j)$. Считается, что декартовые системы координат с началами в точках $O_j$ сонаправлены.

Будем рассматривать задачу упругого деформирования пространства с вытянутыми сфероидальными полостями $\Omega\backslash\bigg\{\bigcup\limits_{\alpha,\beta,\gamma}\omega_{\alpha\beta\gamma}\bigg\}$ под действием нагрузки, приложенной на бесконечности (одноосное, двуосное или всестороннее растяжения упругого пространства). Сфероидальные полости считаются свободными от нагрузки.

Соотношения между координатами можно описать формулами

\begin{equation*}
{x_i} = c\,\mathrm{sh}\xi_j\sin {\eta _i}\cos {\varphi _i},
\end{equation*}

\begin{equation}
{y_i} = c\,\mathrm{sh}\xi_j\sin {\eta _i}\sin {\varphi _i},
\label{eq:11:22}
\end{equation}

\begin{equation*}
{z_i} = c\,\mathrm{ch}\xi_j\cos {\eta _i},
\end{equation*}

\begin{equation}
\left\{ {\begin{array}{*{20}{l}}
{{x_j} = {x_\alpha } + {x_{j\alpha }},}\\
{{y_j} = {y_\alpha } + {y_{j\alpha }},}\\
{{z_j} = {z_\alpha } + {z_{j\alpha }},}
\end{array}} \right.\qquad {\kern 1pt} j \ne \alpha ,\quad j,\alpha  = \overline {1,N},
\label{eq:11:23a}
\end{equation}

\noindent где $\overrightarrow {{O_j}{O_\alpha }}  = \left( {{x_{j\alpha }},{y_{j\alpha }},{z_{j\alpha }}} \right) = \left( {{r_{j\alpha }},{\theta _{j\alpha }},{\varphi _{j\alpha }}} \right)$.

Для определения НДС в рассматриваемом теле необходимо решить краевую задачу для уравнения Ламе относительно неизвестного вектора перемещения   $\mathbf{U}$ с граничными условиями

\begin{equation}
{\bf{FU}}{|_{{\Gamma _j}}} = 0
\end{equation}

\noindent и условиями на бесконечности одного из трех типов

\begin{equation}
{\bf{FU}}{|_{z =  \pm \infty }} =  \pm T{{\bf{e}}_z}\quad\text{(одноосное растяжение)},
\label{eq:11:1a}
\end{equation}

\begin{equation}
{\bf{FU}}{|_{\rho  = \infty }} = T{{\bf{e}}_\rho }\quad\text{(двуосное растяжение)},
\label{eq:11:2a}
\end{equation}

\begin{equation}
{\bf{FU}}{|_{r  = \infty }} = T{{\bf{e}}_r }\quad\text{(всестороннее растяжение)},
\label{eq:11:3a}
\end{equation}

\noindent где $\mathbf{FU}$~--- отвечающий перемещению $\mathbf{U}$ вектор усилий на соответствующей граничной поверхности; ${\Gamma _j} = \left\{ {\left( {{\xi_j},{\eta _j},{\varphi _j}} \right):\,{\xi_j} = {\xi_{0j}}} \right\}$.

Решение задачи будем искать в виде

\begin{equation}
{\bf{U}} = {\bf{\tilde U}} + {{\bf{U}}_0},
\end{equation}

\begin{equation}
{\bf{\tilde U}} = \sum\limits_{j = 1}^\infty {\sum\limits_{s = 1}^3 {\sum\limits_{n = 0}^\infty  {\sum\limits_{m=-n}^{n} {a_{s,n,m}^{(j)}} } } } {\bf{\tilde U}}_{s,n,m}^{ + (5)}\left( {{\xi_j},{\eta _j},{\varphi _j}} \right),
\label{eq:11:24a}
\end{equation}

\begin{equation}
{{\bf{U}}_0} = \frac{T}{{2G}}\left( { - \frac{\sigma }{{1 + \sigma }}{\rho}{{\bf{e}}_{{\rho}}} + \frac{1}{{1 + \sigma }}{z}{{\bf{e}}_z}} \right)\,\text{(одноосное растяжение)},
\label{eq:11:25}
\end{equation}

\begin{equation}
{{\bf{U}}_0} = \frac{T}{{2G}}\left( {\frac{{1 - \sigma }}{{1 + \sigma }}{\rho}{{\bf{e}}_{{\rho}}} - \frac{{2\sigma }}{{1 + \sigma }}{z}{{\bf{e}}_z}} \right)\,\text{(двуосное растяжение)},
\label{eq:11:26}
\end{equation}

\begin{equation}
{{\bf{U}}_0} = \frac{T}{2G}\frac{1-2\sigma}{1+\sigma}\left(\rho\mathbf{e}_\rho+z\mathbf{e}_z\right)\,\text{(всестороннее растяжение)},
\label{eq:11:27}
\end{equation}

\noindent где $G$, $\sigma$~--- модуль сдвига и коэффициент Пуассона упругого пространства; $a_{s,n,m}^{(j)}$~--- неизвестные коэффициенты; перемещения $\mathbf{\tilde U}_{s,n,m}^{+(5)}$ приведены в формулах~\eqref{eq:1:89a}~--- \eqref{eq:1:94a}.

Относительно перемещения $\mathbf{\tilde U}$ граничные условия запишем следующим образом:

\begin{equation}
{\bf{F\tilde U}}{|_{{\Gamma _j}}} =  - {\bf{F}}{{\bf{U}}_0}{|_{{\Gamma _j}}};
\label{eq:11:28a}
\end{equation}

\begin{equation}
{\bf{F\tilde U}}{|_{z =  \pm \infty }} = 0\quad\text{(одноосное растяжение)};
\label{eq:11:29a}
\end{equation}

\begin{equation}
{\bf{F\tilde U}}{|_{\rho  = \infty }} = 0\quad\text{(двуосное растяжение)};
\label{eq:11:30a}
\end{equation}

\begin{equation}
{\bf{F\tilde U}}{|_{r  = \infty }} = 0\quad\text{(всестороннее растяжение)}.
\label{eq:11:30b}
\end{equation}

Вспомогательным перемещениям $\mathbf{U}_0$ отвечают такие напряжения на поверхностях $\Gamma_j$ ($\mathbf{n}_j=\mathbf{e}_{\xi_j}$~--- вектор нормали на поверхности $\Gamma_j$):

\begin{equation}
{\bf{F}}{{\bf{U}}_0} = TH_j\,\mathrm{sh}\xi_j P_1(\cos\eta_j)\mathbf{e}_z\quad\text{(одноосное растяжение)};
\label{eq:11:g1}
\end{equation}

\begin{equation}
{\bf{F}}{{\bf{U}}_0} = -TH_j\,\mathrm{ch}\xi_j P_1^{(1)}(\cos\eta_j)\mathbf{e}_{\rho_j}\quad\text{(двуосное растяжение)};
\label{eq:11:g2}
\end{equation}

\begin{multline}
\mathbf{FU}_0=TH_j\bigg[-\mathrm{ch}\xi_j P_1^{(1)}(\cos\eta_j)\mathbf{e}_{\rho_j}+ \\
+\mathrm{sh}\xi_jP_1(\cos\eta_j)\mathbf{e}_z\bigg]\,\text{(всестороннее растяжение)}.
\label{eq:11:g3}
\end{multline}

Используя теоремы сложения~\eqref{eq:1:100t}, перемещение $\mathbf{\tilde U}$ можно записать полностью в системе координат с началом в точке $O_j$:

\begin{multline}
{\bf{\tilde U}} = \sum\limits_{s = 1}^3 {\sum\limits_{n = 0}^\infty  {\sum\limits_{m =  - n }^{n } {a_{s,n,m}^{(j)}} } } {\bf{\tilde U}}_{s,n,m}^{ +(5)}\left( {{\xi_j},{\eta _j},{\varphi _j}} \right) + \\
+ \sum\limits_{s=1}^3\sum\limits_{n = 0}^\infty\sum\limits_{m =  - n }^{n }{\bf{\tilde U}}_{s,n,m}^{ - (5)}\left( {{\xi_j},{\eta _j},{\varphi _j}} \right)\sum\limits_{\alpha  \ne j}\sum\limits_{t=1}^3\sum\limits_{k = 0}^\infty\mathop \sum \limits_{l =  - k}^{k}\tilde T_{t,k,l,\alpha}^{s,n,m,j}a_{t,k,l}^{(\alpha)}.
\label{eq:11:31}
\end{multline}

В силу периодичности задачи вклад каждого слагаемого в формуле~\eqref{eq:11:24a} будет одинаковым, поэтому можно считать, что $a_{s,n,m}^{(j)}=a_{s,n,m}$.

После перехода в формуле~\eqref{eq:11:31} к напряжениям и удовлетворения граничным условиям относительно неизвестных $a_{s,n,m}$ получаем бесконечную систему линейных алгебраических уравнений:

\begin{equation}
\sum\limits_{s=1}^3 a_{s,n,m}\tilde F_{s,n,m}^{+(k)}+\tilde F_{s,n,m}^{-(k)}\sum\limits_{t=1}^3\sum\limits_{k=0}^\infty\sum\limits_{l=-k}^k a_{t,k,l}\sum\limits_{\alpha\neq j}\tilde T_{t,k,l,\alpha}^{s,n,m,j}=F_{n,m}^{(k)};
\label{eq:11:sys}
\end{equation}
$$
n,m \in\mathbb{Z}:\quad n \ge 0,{\mkern 1mu} \quad {\kern 1pt} |m| \le n,\quad {\mkern 1mu} k =  - 1,{\mkern 1mu} {\kern 1pt} 0,{\mkern 1mu} {\kern 1pt} 1;\;\;\;{\mkern 1mu} {\kern 1pt} j = \overline {1,\infty},
$$

\noindent где $\tilde F_{s,n,m}^{\pm(k)}$~--- компоненты вектора напряжений на поверхности $\xi_j=\xi_{j0}$, отвечающего вектору перемещения $\tilde U_{s,n,m}^{\pm(5)}$:

$$
\mathbf{F\tilde U}_{s,n,m}^{\pm(5)}=\frac{H_j}{c}\bigg(\tilde F_{s,n,m}^{\pm(-1)}S_n^{m-1}\mathbf{e}_{-1}+\tilde F_{s,n,m}^{\pm(1)}S_n^{m+1}\mathbf{e}_1+\tilde F_{s,n,m}^{\pm(0)}S_n^m\mathbf{e}_0\bigg);
$$

\begin{figure}[h!]
\centering\footnotesize
\parbox[b]{7.5cm}{\centering\includegraphics[width=7.6cm]{periodic-cav27-a-d75-t1-sig_x.pdf}
\caption{Напряжения $\sigma_x/T$ на линии $AB$ в зависимости от расстояния между полостями при одноосном растяжении
\label{f:11:20}}}\hfil\hfil
\parbox[b]{7.5cm}{\centering\includegraphics[width=7.6cm]{periodic-cav27-a-d75-t2-sig_x.pdf}
\caption{Напряжения $\sigma_x/T$ на линии $AB$ в зависимости от расстояния между полостями при двуосном растяжении
\label{f:11:21}}}
\end{figure}

\begin{figure}[h!]
\centering\footnotesize
\parbox[b]{7.5cm}{\centering\includegraphics[width=7.6cm]{periodic-cav27-a-d75-t1-sig_y.pdf}
\caption{Напряжения $\sigma_y/T$ на линии $AB$ в зависимости от расстояния между полостями при одноосном растяжении
\label{f:11:22}}}\hfil\hfil
\parbox[b]{7.5cm}{\centering\includegraphics[width=7.6cm]{periodic-cav27-a-d75-t2-sig_y.pdf}
\caption{Напряжения $\sigma_y/T$ на линии $AB$ в зависимости от расстояния между полостями при двуосном растяжении
\label{f:11:23}}}
\end{figure}

\begin{equation*}
F_{n,m}^{(k)} =  -\frac{Td_2}{2G}{\delta _{n1}}{\delta _{m0}}{\delta _{k0}}\quad\text{(одноосное растяжение)},
\label{eq:11:19}
\end{equation*}

\begin{equation*}
F_{n,m}^{(k)} =  -\frac{Td_1}{2G}{\delta _{n1}}{\delta _{m0\,}}(2{\delta _{k, - 1}} - {\delta _{k1}})\quad\text{(двуосное растяжение)},
\label{eq:11:20}
\end{equation*}

\begin{equation*}
F_{n,m}^{(k)} =  -\frac{T}{2G}{\delta _{n1}}{\delta _{m0\,}}(\delta_{k0}d_2+2{\delta _{k, - 1}}d_1 - {\delta _{k1}}d_1)\quad\text{(всестороннее растяжение)}.
\label{eq:11:21}
\end{equation*}

Явный вид компонент $\tilde F_{s,n,m}^{\pm(k)}$ не приводится ввиду их громоздкости. Они получаются из формул~\eqref{eq:1:89a}~--- \eqref{eq:1:99a}, \eqref{eq:9:f1}~--- \eqref{eq:9:f3}.

\begin{figure}[h!]
\centering\footnotesize
\parbox[b]{7.5cm}{\centering\includegraphics[width=7.6cm]{periodic-cav27-a-d75-t1-sig_z.pdf}
\caption{Напряжения $\sigma_z/T$ на линии $AB$ в зависимости от расстояния между полостями при одноосном растяжении
\label{f:11:24}}}\hfil\hfil
\parbox[b]{7.5cm}{\centering\includegraphics[width=7.8cm]{periodic-cav27-a-d75-t2-sig_z.pdf}
\caption{Напряжения $\sigma_z/T$ на линии $AB$ в зависимости от расстояния между полостями при двуосном растяжении
\label{f:11:25}}}
\end{figure}

\begin{figure}[h!]
\centering\footnotesize
\parbox[b]{7.5cm}{\centering\includegraphics[width=8cm]{periodic-cav27-d-a28-t1.pdf}
\caption{Сравнение напряжений на линии $AB$ в зависимости от формы сфероидальной полости при одноосном растяжении
\label{f:11:31}}}\hfil\hfil
\parbox[b]{7.5cm}{\centering\includegraphics[width=8cm]{periodic-cav27-d-a28-t2.pdf}
\caption{Сравнение напряжений на линии $AB$ в зависимости от формы сфероидальной полости при двуосном растяжении
\label{f:11:32}}}
\end{figure}

На рис.~\ref{f:11:20}~--- \ref{f:11:25} приведены нормальные напряжения на линии $AB$ в зависимости от относительного расстояния $a/d_1$ между полостями при одноосном и двуосном растяжениях упругого пространства при $\sigma=0.38$ и $d_2/d_1=0.75$.

При одноосном растяжении основной вклад в тензор напряжений вносят напряжения $\sigma_z/T$. Областью их концентрации являются границы полостей. С приближением полостей друг к другу эти напряжения возрастают.

При двуосном растяжении подобным поведением характеризуются напряжения $\sigma_y/T$.

На рис.~\ref{f:11:31}, \ref{f:11:32} представлено сравнение нормальных напряжений на линии $AB$ в зависимости от формы сфероидальных полостей при одноосном и двуосном растяжениях упругого пространства.

Заметное отличие наблюдается в значениях напряжений $\sigma_z/T$ при одноосном растяжении и $\sigma_x/T$~--- при двуосном. В случае одноосного растяжения для полости, близкой по форме к сферической, все приведенные напряжения больше, чем для вытянутой сфероидальной полости. В случае двуосного растяжения для напряжения $\sigma_x/T$ наблюдается обратная картина.

\begin{table}[h!]
\centering
\caption{\centering Сравнение напряжений для разного количества полостей~периодической~структуры}
$
\begin{array}{|c|c|c|c|}
\hline
\text{Кол-во полостей} & \sigma_x/T & \sigma_y/T & \sigma_z/T \\
\hline
27 & 0.1751 & 0.04153 & 1.1248 \\
\hline
125 & 0.1752 & 0.04261 & 1.1261 \\
\hline
\end{array}
$
\label{t:11:2}
\end{table}

В табл.~\ref{t:11:2} приведено сравнение нормальных напряжений в средней точке отрезка $AB$ для разного количества (27 и 125) сфероидальных полостей периодической структуры. Из таблицы видно, что увеличение числа полостей практически не меняет значения напряжений.

\section[Упругое пространство с периодической системой вытянутых сфероидальных включений]{Упругое пространство с периодической системой вытянутых сфероидальных включений\sectionmark{Упругое пространство с периодической системой сфероидальных включений}}\sectionmark{Упругое пространство с периодической системой сфероидальных включений}

Рассмотрим упругое пространство $\Omega$ и бесконечную систему вытянутых сфероидальных включений $\{\omega_{\alpha\beta\gamma}\}_{\alpha,\beta,\gamma=-\infty}^\infty$, центры которых расположены в узлах $\{O_{\alpha\beta\gamma}\}_{\alpha,\beta,\gamma=-\infty}^\infty$ кубической периодической решетки со стороной $a$. Декартовыми координатами узлов решетки будут упорядоченные наборы чисел $\{(\alpha a,\beta a,\gamma a);\,\alpha,\beta,\gamma\in\mathbb{Z}\}$. Полуоси включений обозначим через $d_1$ и $d_2$ ($d_1>d_2$). Выполним линейное упорядочение узлов таким же образом, как в параграфе 6.1.\sloppy

В новой нумерации точка $O_{\alpha,\beta,\gamma}$ обозначается через $O_j$ (см.~рис.~\ref{f:11:1a}). С каждой точкой $O_j$ свяжем локальные декартовую $(x_j,y_j,z_j)$ и сонаправленную с ней вытянутую сфероидальную системы координат $(\xi_j,\eta_j,\varphi_j)$. Считается, что декартовые системы координат с началами в точках $O_j$ сонаправлены.

Будем рассматривать задачу упругого деформирования пространства с вытянутыми сфероидальными включениями под действием нагрузки, приложенной на бесконечности (одноосное, двуосное или всестороннее растяжения упругого пространства).

Соотношения между координатами можно описать формулами~\eqref{eq:11:22}, \eqref{eq:11:23a}.

Предполагается, что упругие постоянные включений равны $(\sigma_j,G_j)$. Упругие постоянные матрицы будем считать равными $(\sigma,G)$.

Граничные условия в рассматриваемой задаче представляют собой условия сопряжения полей перемещений и напряжений на поверхностях $\Gamma_j$. Для того, чтобы их записать, представим вектор перемещений в упругом пространстве в виде

\begin{equation}
{\bf{U}} = \left\{ {\begin{array}{*{20}{l}}
{{\bf{\tilde U}}_j^ - ,\quad \left( {x,y,z} \right) \in \omega_j,}\\
{{{{\bf{\tilde U}}}^ + } + {{\bf{U}}_0},\quad \left( {x,y,z} \right) \in\Omega\backslash {\bigcup\limits_j\omega_j},}
\end{array}} \right.
\label{eq:11:23i}
\end{equation}

\noindent где $\omega_j = \left\{ {\left( {{\xi_j},{\eta _j},{\varphi _j}} \right):\, {\xi_j} < {\xi_{0j}}} \right\}$. Тогда условия сопряжения принимают следующий вид:

\begin{equation}
\left( {{{{\bf{\tilde U}}}^ + } + {{\bf{U}}_0}} \right){|_{{\Gamma _j}}} = {\bf{\tilde U}}_j^ - {|_{{\Gamma _j}}},
\label{eq:11:24i}
\end{equation}

\begin{equation}
\left( {{\bf{F\tilde U^+}} + {\bf{F}}{{\bf{U}}_0}} \right){|_{{\Gamma _j}}} = {\bf{F}}{{\bf{\tilde U^-}}_j}{|_{{\Gamma _j}}},\qquad {\kern 1pt} j = 1,{\mkern 1mu} {\kern 1pt} 2,\dots.
\label{eq:11:25i}
\end{equation}

\noindent Условия на бесконечности задаются формулами~\eqref{eq:11:g1}~--- \eqref{eq:11:g3}.

Решение задачи будем искать в виде

\begin{equation}
{\bf{\tilde U^+}} = \sum\limits_{j = 1}^\infty {\sum\limits_{s = 1}^3 {\sum\limits_{n = 0}^\infty  {\sum\limits_{m=-n}^{n} {a_{s,n,m}^{(j)}} } } } {\bf{\tilde U}}_{s,n,m}^{ + (5)}\left( {{\xi_j},{\eta _j},{\varphi _j}} \right),
\label{eq:11:26i}
\end{equation}

\begin{equation}
\mathbf{\tilde U}_j^- = {\sum\limits_{s = 1}^3 {\sum\limits_{n = 0}^\infty  {\sum\limits_{m=-n}^{n} {b_{s,n,m}^{(j)}} } } } {\bf{\tilde U}}_{s,n,m}^{ - (5)}\left( {{\xi_j},{\eta _j},{\varphi _j}} \right),
\label{eq:11:27i}
\end{equation}

\noindent где $G$, $\sigma$~--- модуль сдвига и коэффициент Пуассона упругого пространства; $a_{s,n,m}^{(j)}$, $b_{s,n,m}^{(j)}$~--- неизвестные коэффициенты; перемещения $\mathbf{\tilde U}_{s,n,m}^{\pm(5)}$ приведены в формулах~\eqref{eq:1:89a}~--- \eqref{eq:1:99a}.

Для представления вектора перемещений $\mathbf{\tilde U}^+$ в системах координат с началами $O_j$ можем использовать формулу~\eqref{eq:11:31}.

В силу периодичности задачи вклад каждого слагаемого в формулах~\eqref{eq:11:26i}, \eqref{eq:11:27i} будет одинаковым, поэтому можно считать, что $a_{s,n,m}^{(j)}=\\=a_{s,n,m}$, $b_{s,n,m}^{(j)}=b_{s,n,m}$.

После перехода к напряжениям в формуле~\eqref{eq:11:31} и удовлетворения условиям~\eqref{eq:11:24i}, \eqref{eq:11:25i} получаем бесконечную систему линейных алгебраических уравнений относительно неизвестных $a_{s,n,m}$, $b_{s,n,m}$:

\begin{multline}
\sum\limits_{s=1}^3 a_{s,n,m}\tilde E_{s,n,m}^{+(k)}(\sigma)+\tilde E_{s,n,m}^{-(k)}(\sigma)\sum\limits_{t=1}^3\sum\limits_{k=0}^\infty\sum\limits_{l=-k}^k a_{t,k,l}\sum\limits_{\alpha\neq j}\tilde T_{t,k,l,\alpha}^{s,n,m,j}= \\
=-E_{n,m}^{(k)}+\sum\limits_{s=1}^3 b_{s,n,m}\tilde E_{s,n,m}^{-(k)}(\sigma_j);
\label{eq:11:28i}
\end{multline}

\begin{multline}
\sum\limits_{s=1}^3 a_{s,n,m}\tilde F_{s,n,m}^{+(k)}(\sigma)+\tilde F_{s,n,m}^{-(k)}(\sigma)\sum\limits_{t=1}^3\sum\limits_{k=0}^\infty\sum\limits_{l=-k}^k a_{t,k,l}\sum\limits_{\alpha\neq j}\tilde T_{t,k,l,\alpha}^{s,n,m,j}= \\
=F_{n,m}^{(k)}+\frac{G_j}{G}\sum\limits_{s=1}^3 b_{s,n,m}\tilde F_{s,n,m}^{-(k)}(\sigma_j);
\label{eq:11:29i}
\end{multline}

\begin{equation}
n,m \in\mathbb{Z} :\quad n \ge 0,\quad |m| \le n,\quad k =  - 1,{\mkern 1mu} {\kern 1pt} 0,{\mkern 1mu} {\kern 1pt} 1,
\end{equation}

\noindent где $\tilde E_{s,n,m}^{\pm(k)}$~--- компоненты вектора перемещений на поверхности $\xi_j=\xi_{0j}$; $\tilde F_{s,n,m}^{\pm(k)}$~--- компоненты вектора напряжений на поверхности $\xi_j=\xi_{0j}$, отвечающего вектору перемещения $\tilde U_{s,n,m}^{\pm(5)}$:

$$
\mathbf{\tilde U}_{s,n,m}^{\pm(5)}=\tilde E_{s,n,m}^{\pm(-1)}S_n^{m-1}\mathbf{e}_{-1}+\tilde E_{s,n,m}^{\pm(1)}S_n^{m+1}\mathbf{e}_1+\tilde E_{s,n,m}^{\pm(0)}S_n^m\mathbf{e}_0;
$$

$$
\mathbf{F\tilde U}_{s,n,m}^{\pm(5)}=\frac{H_j}{c}\bigg(\tilde F_{s,n,m}^{\pm(-1)}S_n^{m-1}\mathbf{e}_{-1}+\tilde F_{s,n,m}^{\pm(1)}S_n^{m+1}\mathbf{e}_1+\tilde F_{s,n,m}^{\pm(0)}S_n^m\mathbf{e}_0\bigg);
$$

\begin{equation*}
E_{n,m}^{(k)} =\frac{Tc}{2G}\delta_{n1}\delta_{m0}\bigg[-\frac{2\sigma}{1+\sigma}\,\mathrm{sh}\xi_{0j}\delta_{k,-1}+\frac{\sigma}{1+\sigma}\,\mathrm{sh}\xi_{0j}\delta_{k1}+\frac{1}{1+\sigma}\,\mathrm{ch}\xi_{0j}\delta_{k0}\bigg],
\end{equation*}

\begin{equation*}
F_{n,m}^{(k)} =  -\frac{Td_2}{2G}{\delta _{n1}}{\delta _{m0}}{\delta _{k0}}\quad\text{(одноосное растяжение)};
\end{equation*}

\begin{equation*}
E_{n,m}^{(k)} =\frac{Tc}{2G}\delta_{n1}\delta_{m0}\bigg[\frac{2-2\sigma}{1+\sigma}\,\mathrm{sh}\xi_{0j}\delta_{k,-1}-\frac{1-\sigma}{1+\sigma}\,\mathrm{sh}\xi_{0j}\delta_{k1}-\frac{2\sigma}{1+\sigma}\,\mathrm{ch}\xi_{0j}\delta_{k0}\bigg],
\end{equation*}

\begin{figure}[h!]
\centering\footnotesize
\parbox[b]{7.5cm}{\centering\includegraphics[width=7.6cm]{periodic-inc27-a-d75-g25-t1-sig_x.pdf}
\caption{Напряжения $\sigma_x/T$ на линии $AB$ в зависимости от расстояния между включениями при одноосном растяжении
\label{f:11:25}}}\hfil\hfil
\parbox[b]{7.5cm}{\centering\includegraphics[width=7.6cm]{periodic-inc27-a-d75-g25-t2-sig_x.pdf}
\caption{Напряжения $\sigma_x/T$ на линии $AB$ в зависимости от расстояния между включениями при двуосном растяжении
\label{f:11:26}}}
\end{figure}

\begin{figure}[h!]
\centering\footnotesize
\parbox[b]{7.5cm}{\centering\includegraphics[width=7.6cm]{periodic-inc27-a-d75-g25-t1-sig_y.pdf}
\caption{Напряжения $\sigma_y/T$ на линии $AB$ в зависимости от расстояния между включениями при одноосном растяжении
\label{f:11:27}}}\hfil\hfil
\parbox[b]{7.5cm}{\centering\includegraphics[width=7.6cm]{periodic-inc27-a-d75-g25-t2-sig_y.pdf}
\caption{Напряжения $\sigma_y/T$ на линии $AB$ в зависимости от расстояния между включениями при двуосном растяжении
\label{f:11:28}}}
\end{figure}

\begin{equation*}
F_{n,m}^{(k)} =  -\frac{Td_1}{2G}{\delta _{n1}}{\delta _{m0\,}}(2{\delta _{k, - 1}} - {\delta _{k1}})\quad\text{(двуосное растяжение)};
\end{equation*}

\begin{equation*}
E_{n,m}^{(k)} =\frac{Tc}{2G}\delta_{n1}\delta_{m0}\frac{1-2\sigma}{1+\sigma}\bigg[2\,\mathrm{sh}\xi_{0j}\delta_{k,-1}-\mathrm{sh}\xi_{0j}\delta_{k1}+\mathrm{ch}\xi_{0j}\delta_{k0}\bigg],
\end{equation*}

\begin{equation*}
F_{n,m}^{(k)} =  -\frac{T}{2G}{\delta _{n1}}{\delta _{m0\,}}(d_2\delta_{k0}+2d_1{\delta _{k, - 1}} - d_1{\delta _{k1}})\quad\text{(всестороннее растяжение)}.
\end{equation*}

Явный вид компонент $\tilde E_{s,n,m}^{\pm(k)}$ и $\tilde F_{s,n,m}^{\pm(k)}$ не приводится ввиду их громоздкости. Они получаются из формул~\eqref{eq:1:42}~--- \eqref{eq:1:44}, \eqref{eq:1:89a}~--- \eqref{eq:1:99a}, \eqref{eq:9:f1}~--- \eqref{eq:9:f3}.

На рис.~\ref{f:11:25}~--- \ref{f:11:30} приведены нормальные напряжения на линии $AB$ в зависимости от относительного расстояния между включениями при одноосном и двуосном растяжениях упругого пространства при $\sigma=0.38$, $\sigma_j=0.21$, $G_j/G=25$, $d_2/d_1=0.75$.

\begin{figure}[h!]
\centering\footnotesize
\parbox[b]{7.5cm}{\centering\includegraphics[width=7.6cm]{periodic-inc27-a-d75-g25-t1-sig_z.pdf}
\caption{Напряжения $\sigma_z/T$ на линии $AB$ в зависимости от расстояния между включениями при одноосном растяжении
\label{f:11:29}}}\hfil\hfil
\parbox[b]{7.5cm}{\centering\includegraphics[width=7.6cm]{periodic-inc27-a-d75-g25-t2-sig_z.pdf}
\caption{Напряжения $\sigma_z/T$ на линии $AB$ в зависимости от расстояния между включениями при двуосном растяжении
\label{f:11:30}}}
\end{figure}

\begin{figure}[h!]
\centering\footnotesize
\parbox[b]{7.5cm}{\centering\includegraphics[width=7.8cm]{periodic-inc27-d-a25-g25-t1.pdf}
\caption{Сравнение напряжений на линии $AB$ в зависимости от формы сфероидального включения при одноосном растяжении
\label{f:11:33}}}\hfil\hfil
\parbox[b]{7.5cm}{\centering\includegraphics[width=8cm]{periodic-inc27-d-a25-g25-t2.pdf}
\caption{Сравнение напряжений на линии $AB$ в зависимости от формы сфероидального включения при двуосном растяжении
\label{f:11:34}}}
\end{figure}

При одноосном растяжении основной вклад в тензор напряжений вносят напряжения $\sigma_z/T$. Напряжения $\sigma_x/T$ тоже значимы. Характерно, что они являются сжимающими. Область концентрации напряжений $\sigma_z/T$ совпадает с серединой отрезка $AB$. С приближением включений друг к другу эти напряжения убывают.

При двуосном растяжении основной вклад в тензор напряжений вносят напряжения $\sigma_x/T$, однако напряжения $\sigma_y/T$ и $\sigma_z/T$ тоже значимы. Напряжения $\sigma_x/T$ и $\sigma_y/T$ незначительно меняются в пределах отрезка $AB$ и возрастают с приближением полостей друг к другу. Наблюдается сильная концентрация напряжений $\sigma_z/T$ на границе включений.

На рис.~\ref{f:11:33}, \ref{f:11:34} представлено сравнение нормальных напряжений на линии $AB$ в зависимости от формы сфероидальных включений при одноосном и двуосном растяжениях упругого пространства.

Графики показывают, что в случае и одноосного и двуосного растяжений упругого пространства при изменении формы сфероидального включения сохраняется общий характер в распределении напряжений, однако численные значения напряжений существенно отличаются.

%*************************************************************************************

\section[Упругое пространство с периодической системой сжатых сфероидальных полостей]{Упругое пространство с периодической системой сжатых сфероидальных полостей\sectionmark{Упругое пространство с периодической системой сфероидальных полостей}}\sectionmark{Упругое пространство с периодической системой сфероидальных полостей}

Рассмотрим упругое пространство $\Omega$ и бесконечную систему сжатых сфероидальных полостей $\{\omega_{\alpha\beta\gamma}\}_{\alpha,\beta,\gamma=-\infty}^\infty$, центры которых расположены в узлах $\{O_{\alpha\beta\gamma}\}_{\alpha,\beta,\gamma=-\infty}^\infty$ кубической периодической решетки со стороной $a$. Декартовыми координатами узлов решетки будут упорядоченные наборы чисел $\{(\alpha a,\beta a,\gamma a);\,\alpha,\beta,\gamma\in\mathbb{Z}\}$. Полуоси полостей обозначим через $d_1$ и $d_2$ ($d_2>d_1$). Выполним линейное упорядочение узлов таким же образом, как в параграфе~6.1.\sloppy

В новой нумерации точка $O_{\alpha,\beta,\gamma}$ обозначается через $O_j$ (рис.~\ref{f:11:1b}).

\begin{figure}[h!]
\centering
\includegraphics[width=12cm]{oblate-spheroids-periodic.pdf}
\caption{Периодическая система сжатых сфероидальных полостей в упругом пространстве}
\label{f:11:1b}
\end{figure}

С каждой точкой $O_j$ свяжем локальные декартовую $(x_j,y_j,z_j)$ и сонаправленную с ней сжатую сфероидальную системы координат $(\tilde\xi_j,\tilde\eta_j,\varphi_j)$. Считается, что декартовые системы координат с началами в точках $O_j$ сонаправлены.

Будем рассматривать задачу упругого деформирования пространства со сжатыми сфероидальными полостями $\Omega\backslash\bigg\{\bigcup\limits_{\alpha,\beta,\gamma}\omega_{\alpha\beta\gamma}\bigg\}$ под действием нагрузки, приложенной на бесконечности (одноосное, двуосное или всестороннее растяжение упругого пространства). Сфероидальные полости считаются свободными от нагрузки.

Соотношения между координатами можно описать формулами

\begin{equation*}
{x_i} = \tilde c\,\mathrm{ch}\tilde\xi_j\sin {\tilde\eta _i}\cos {\varphi _i},
\end{equation*}

\begin{equation}
{y_i} = \tilde c\,\mathrm{ch}\tilde\xi_j\sin {\tilde\eta _i}\sin {\varphi _i},
\label{eq:11:22k}
\end{equation}

\begin{equation*}
{z_i} = \tilde c\,\mathrm{sh}\tilde\xi_j\cos {\tilde\eta _i},
\end{equation*}

\begin{equation}
\left\{ {\begin{array}{*{20}{l}}
{{x_j} = {x_\alpha } + {x_{j\alpha }},}\\
{{y_j} = {y_\alpha } + {y_{j\alpha }},}\\
{{z_j} = {z_\alpha } + {z_{j\alpha }},}
\end{array}} \right.\qquad {\kern 1pt} j \ne \alpha ,\quad j,\alpha  = \overline {1,N},
\label{eq:11:23k}
\end{equation}

\noindent где $\overrightarrow {{O_j}{O_\alpha }}  = \left( {{x_{j\alpha }},{y_{j\alpha }},{z_{j\alpha }}} \right) = \left( {{r_{j\alpha }},{\theta _{j\alpha }},{\varphi _{j\alpha }}} \right)$.

Для определения НДС в рассматриваемом теле необходимо решить краевую задачу для уравнения Ламе относительно неизвестного вектора перемещения   $\mathbf{U}$ с граничными условиями

\begin{equation}
{\bf{FU}}{|_{{\Gamma _j}}} = 0
\end{equation}

\noindent и условиями на бесконечности одного из трех типов

\begin{equation}
{\bf{FU}}{|_{z =  \pm \infty }} =  \pm T{{\bf{e}}_z}\quad\text{(одноосное растяжение)},
\label{eq:11:1k}
\end{equation}

\begin{equation}
{\bf{FU}}{|_{\rho  = \infty }} = T{{\bf{e}}_\rho }\quad\text{(двуосное растяжение)},
\label{eq:11:2k}
\end{equation}

\begin{equation}
{\bf{FU}}{|_{r  = \infty }} = T{{\bf{e}}_r }\quad\text{(всестороннее растяжение)},
\label{eq:11:3k}
\end{equation}

\noindent где $\mathbf{FU}$~--- отвечающий перемещению $\mathbf{U}$ вектор усилий на соответствующей граничной поверхности; ${\Gamma _j} = \left\{ {\left( {{\tilde\xi_j},{\tilde\eta _j},{\varphi _j}} \right):\,{\tilde\xi_j} = {\tilde\xi_{0j}}} \right\}$.

Решение задачи будем искать в виде

\begin{equation}
{\bf{U}} = {\bf{\tilde U}} + {{\bf{U}}_0},
\end{equation}

\begin{equation}
{\bf{\tilde U}} = \sum\limits_{j = 1}^\infty {\sum\limits_{s = 1}^3 {\sum\limits_{n = 0}^\infty  {\sum\limits_{m=-n}^{n} {a_{s,n,m}^{(j)}} } } } {\bf{\tilde U}}_{s,n,m}^{ + (6)}\left( {{\tilde\xi_j},{\tilde\eta _j},{\varphi _j}} \right),
\label{eq:11:24k}
\end{equation}

\begin{equation}
{{\bf{U}}_0} = \frac{T}{{2G}}\left( { - \frac{\sigma }{{1 + \sigma }}{\rho}{{\bf{e}}_{{\rho}}} + \frac{1}{{1 + \sigma }}{z}{{\bf{e}}_z}} \right)\,\text{(одноосное растяжение)},
\label{eq:11:25k}
\end{equation}

\begin{equation}
{{\bf{U}}_0} = \frac{T}{{2G}}\left( {\frac{{1 - \sigma }}{{1 + \sigma }}{\rho}{{\bf{e}}_{{\rho}}} - \frac{{2\sigma }}{{1 + \sigma }}{z}{{\bf{e}}_z}} \right)\,\text{(двуосное растяжение)},
\label{eq:11:26k}
\end{equation}

\begin{equation}
{{\bf{U}}_0} = \frac{T}{2G}\frac{1-2\sigma}{1+\sigma}\left(\rho\mathbf{e}_\rho+z\mathbf{e}_z\right)\,\text{(всестороннее растяжение)},
\label{eq:11:27k}
\end{equation}

\noindent где $G$, $\sigma$~--- модуль сдвига и коэффициент Пуассона упругого пространства; $a_{s,n,m}^{(j)}$~--- неизвестные коэффициенты; перемещения $\mathbf{\tilde U}_{s,n,m}^{+(6)}$ приведены в формулах~\eqref{eq:1:89o}~--- \eqref{eq:1:94o}.

Относительно перемещения $\mathbf{\tilde U}$ граничные условия записывают следующим образом:

\begin{equation}
{\bf{F\tilde U}}{|_{{\Gamma _j}}} =  - {\bf{F}}{{\bf{U}}_0}{|_{{\Gamma _j}}};
\label{eq:11:28k}
\end{equation}

\begin{equation}
{\bf{F\tilde U}}{|_{z =  \pm \infty }} = 0\quad\text{(одноосное растяжение)};
\label{eq:11:29k}
\end{equation}

\begin{equation}
{\bf{F\tilde U}}{|_{\rho  = \infty }} = 0\quad\text{(двуосное растяжение)};
\label{eq:11:30k}
\end{equation}

\begin{equation}
{\bf{F\tilde U}}{|_{r  = \infty }} = 0\quad\text{(всестороннее растяжение)},
\label{eq:11:30ak}
\end{equation}

Вспомогательным перемещениям $\mathbf{U}_0$ отвечают следующие напряжения на поверхностях $\Gamma_j$ ($\mathbf{n}_j=\mathbf{e}_{\tilde\xi_j}$~--- вектор нормали на поверхности $\Gamma_j$):

\begin{equation}
{\bf{F}}{{\bf{U}}_0} = T\tilde H_j\,\mathrm{ch}\tilde\xi_j P_1(\cos\tilde\eta_j)\mathbf{e}_z\quad\text{(одноосное растяжение)};
\label{eq:11:k1}
\end{equation}

\begin{equation}
{\bf{F}}{{\bf{U}}_0} = -T\tilde H_j\,\mathrm{sh}\tilde\xi_j P_1^{(1)}(\cos\tilde\eta_j)\mathbf{e}_{\rho_j}\quad\text{(двуосное растяжение)};
\label{eq:11:k2}
\end{equation}

\begin{multline}
\mathbf{FU}_0=T\tilde H_j\bigg[-\mathrm{sh}\tilde\xi_j P_1^{(1)}(\cos\tilde\eta_j)\mathbf{e}_{\rho_j}+ \\
+\mathrm{ch}\tilde\xi_jP_1(\cos\tilde\eta_j)\mathbf{e}_z\bigg]\,\text{(всестороннее растяжение)}.
\label{eq:11:k3}
\end{multline}

Используя теоремы сложения~\eqref{eq:1:100o}, перемещение $\mathbf{\tilde U}$ можно записать полностью в системе координат с началом в точке $O_j$:

\begin{multline}
{\bf{\tilde U}} = \sum\limits_{s = 1}^3 {\sum\limits_{n = 0}^\infty  {\sum\limits_{m =  - n }^{n } {a_{s,n,m}^{(j)}} } } {\bf{\tilde U}}_{s,n,m}^{ +(6)}\left( {{\tilde\xi_j},{\tilde\eta _j},{\varphi _j}} \right) + \\
+ \sum\limits_{s=1}^3\sum\limits_{n = 0}^\infty\sum\limits_{m =  - n }^{n }{\bf{\tilde U}}_{s,n,m}^{ - (6)}\left( {{\tilde\xi_j},{\tilde\eta _j},{\varphi _j}} \right)\sum\limits_{\alpha  \ne j}\sum\limits_{t=1}^3\sum\limits_{k = 0}^\infty\mathop \sum \limits_{l =  - k}^{k}\tilde T_{t,k,l,\alpha}^{s,n,m,j}a_{t,k,l}^{(\alpha)}.
\label{eq:11:31k}
\end{multline}

В силу периодичности задачи вклад каждого слагаемого в формуле~\eqref{eq:11:24k} будет одинаковым, поэтому можно считать, что $a_{s,n,m}^{(j)}=a_{s,n,m}$.

После перехода в формуле~\eqref{eq:11:31k} к напряжениям и удовлетворения граничным условиям относительно неизвестных $a_{s,n,m}$ получаем бесконечную систему линейных алгебраических уравнений:

\begin{equation}
\sum\limits_{s=1}^3 a_{s,n,m}\tilde F_{s,n,m}^{+(k)}+\tilde F_{s,n,m}^{-(k)}\sum\limits_{t=1}^3\sum\limits_{k=0}^\infty\sum\limits_{l=-k}^k a_{t,k,l}\sum\limits_{\alpha\neq j}\tilde T_{t,k,l,\alpha}^{s,n,m,j}=F_{n,m}^{(k)};
\label{eq:11:32k}
\end{equation}
$$
n,m \in\mathbb{Z}:\quad n \ge 0,{\mkern 1mu} \quad {\kern 1pt} |m| \le n,\quad {\mkern 1mu} k =  - 1,{\mkern 1mu} {\kern 1pt} 0,{\mkern 1mu} {\kern 1pt} 1;\;\;\;{\mkern 1mu} {\kern 1pt} j = \overline {1,\infty},
$$

\noindent где $\tilde F_{s,n,m}^{\pm(k)}$~--- компоненты вектора напряжений на поверхности $\tilde\xi_j=\tilde\xi_{j0}$, отвечающего вектору перемещения $\tilde U_{s,n,m}^{\pm(6)}$:

$$
\mathbf{F\tilde U}_{s,n,m}^{\pm(6)}=\frac{\tilde H_j}{\tilde c}\bigg(\tilde F_{s,n,m}^{\pm(-1)}S_n^{m-1}\mathbf{e}_{-1}+\tilde F_{s,n,m}^{\pm(1)}S_n^{m+1}\mathbf{e}_1+\tilde F_{s,n,m}^{\pm(0)}S_n^m\mathbf{e}_0\bigg);
$$

\begin{equation*}
F_{n,m}^{(k)} =  -\frac{Td_2}{2G}{\delta _{n1}}{\delta _{m0}}{\delta _{k0}}\quad\text{(одноосное растяжение)},
\end{equation*}

\begin{equation*}
F_{n,m}^{(k)} =  -\frac{Td_1}{2G}{\delta _{n1}}{\delta _{m0\,}}(2{\delta _{k, - 1}} - {\delta _{k1}})\quad\text{(двуосное растяжение)},
\end{equation*}

\begin{equation*}
F_{n,m}^{(k)} =  -\frac{T}{2G}{\delta _{n1}}{\delta _{m0\,}}(\delta_{k0}d_2+2{\delta _{k, - 1}}d_1 - {\delta _{k1}}d_1)\quad\text{(всестороннее растяжение)}.
\end{equation*}

Явный вид компонент $\tilde F_{s,n,m}^{\pm(k)}$ не приводится ввиду их громоздкости. Они получаются из формул~\eqref{eq:1:89o}~--- \eqref{eq:1:99o}, \eqref{eq:10:25o}~--- \eqref{eq:10:27o}.

На рис.~\ref{f:11:35}~--- \ref{f:11:40} представлены нормальные напряжения на линии $AB$ в зависимости от расстояния между включениями при одноосном и двуосном растяжениях упругого пространства при $\sigma=0.38$, $d_1/d_2=0.5$.

При одноосном растяжении основной вклад в тензор напряжений вносят напряжения $\sigma_z/T$. Областью их концентрации являются границы полостей, и с приближением полостей друг к другу эти напряжения растут. Подобным свойством обладают напряжения $\sigma_x/T$ и $\sigma_y/T$.

При двуосном растяжении основной вклад в тензор напряжений вносят напряжения $\sigma_y/T$. Эти напряжения изменяются несущественно на отрезке $AB$ и с приближением полостей друг к другу возрастают. Напряжения $\sigma_z/T$ меняют знак на отрезке $AB$, оставаясь сжимающими вблизи его концов, и растягивающими~--- вблизи середины отрезка.

\begin{figure}[h!]
\centering\footnotesize
\parbox[b]{7.5cm}{\centering\includegraphics[width=7.8cm]{periodic-oblate-cav27-a-d50-t1-sig_x.pdf}
\caption{Напряжения $\sigma_x/T$ на линии $AB$ в зависимости от расстояния между полостями при одноосном растяжении
\label{f:11:35}}}\hfil\hfil
\parbox[b]{7.5cm}{\centering\includegraphics[width=7.6cm]{periodic-oblate-cav27-a-d50-t2-sig_x.pdf}
\caption{Напряжения $\sigma_x/T$ на линии $AB$ в зависимости от расстояния между полостями при двуосном растяжении
\label{f:11:36}}}
\end{figure}

\begin{figure}[h!]
\centering\footnotesize
\parbox[b]{7.5cm}{\centering\includegraphics[width=7.6cm]{periodic-oblate-cav27-a-d50-t1-sig_y.pdf}
\caption{Напряжения $\sigma_y/T$ на линии $AB$ в зависимости от расстояния между полостями при одноосном растяжении
\label{f:11:37}}}\hfil\hfil
\parbox[b]{7.5cm}{\centering\includegraphics[width=7.6cm]{periodic-oblate-cav27-a-d50-t2-sig_y.pdf}
\caption{Напряжения $\sigma_y/T$ на линии $AB$ в зависимости от расстояния между полостями при двуосном растяжении
\label{f:11:38}}}
\end{figure}

\begin{figure}[h!]
\centering\footnotesize
\parbox[b]{7.5cm}{\centering\includegraphics[width=7.6cm]{periodic-oblate-cav27-a-d50-t1-sig_z.pdf}
\caption{Напряжения $\sigma_z/T$ на линии $AB$ в зависимости от расстояния между полостями при одноосном растяжении
\label{f:11:39}}}\hfil\hfil
\parbox[b]{7.5cm}{\centering\includegraphics[width=7.8cm]{periodic-oblate-cav27-a-d50-t2-sig_z.pdf}
\caption{Напряжения $\sigma_z/T$ на линии $AB$ в зависимости от расстояния между полостями при двуосном растяжении
\label{f:11:40}}}
\end{figure}

\begin{figure}[h!]
\centering\footnotesize
\parbox[b]{7.5cm}{\centering\includegraphics[width=7.8cm]{periodic-oblate-cav27-d-a25-t1.pdf}
\caption{Сравнение напряжений на линии $AB$ в зависимости от формы сфероидальной полости при одноосном растяжении
\label{f:11:41}}}\hfil\hfil
\parbox[b]{7.5cm}{\centering\includegraphics[width=8cm]{periodic-oblate-cav27-d-a25-t2.pdf}
\caption{Сравнение напряжений на линии $AB$ в зависимости от формы сфероидальной полости при двуосном растяжении
\label{f:11:42}}}
\end{figure}

На рис.~\ref{f:11:41}, \ref{f:11:42} представлено сравнение нормальных напряжений на линии $AB$ в зависимости от формы сжатой сфероидальной полости при одноосном и двуосном растяжениях упругого пространства при $a/d_2=2.5$. При одноосном растяжении в наибольшей степени зависят от формы полости напряжения $\sigma_z/T$ и $\sigma_y/T$, причем различия в значениях напряжений особенно заметны на границах полостей. С уменьшением отношения $d_1/d_2$ наблюдается резкая концентрация этих напряжений.

При двуосном растяжении зависимость от формы полости не столь существенна, как при одноосном растяжении. Уменьшение отношения $d_1/d_2$ не приводит к резкой концентрации напряжений.

\begin{table}[h!]
\centering
\caption{\centering Сравнение напряжений для разного количества полостей~периодической~структуры}
$
\begin{array}{|c|c|c|c|}
\hline
\text{Кол-во полостей} & \sigma_x/T & \sigma_y/T & \sigma_z/T \\
\hline
27 & 0.47211 & 0.17632 & 1.2920 \\
\hline
125 & 0.47208 & 0.17857 & 1.2947 \\
\hline
\end{array}
$
\label{t:11:3}
\end{table}

В табл.~\ref{t:11:3} приведено сравнение нормальных напряжений в средней точке отрезка $AB$ для разного количества (27 и 125) сжатых сфероидальных полостей периодической структуры при $a/d_2=2.8$, $d_1/d_2=0.5$ и при одноосном растяжении упругого пространства. Из таблицы видно, что увеличение числа полостей практически не меняет значения напряжений.


\section[Упругое пространство с периодической системой сжатых сфероидальных включений]{Упругое пространство с периодической системой сжатых сфероидальных включений\sectionmark{Упругое пространство с периодической системой сфероидальных включений}}\sectionmark{Упругое пространство с периодической системой сфероидальных включений}

Рассмотрим упругое пространство $\Omega$ и бесконечную систему сжатых сфероидальных включений $\{\omega_{\alpha\beta\gamma}\}_{\alpha,\beta,\gamma=-\infty}^\infty$, центры которых расположены в узлах $\{O_{\alpha\beta\gamma}\}_{\alpha,\beta,\gamma=-\infty}^\infty$ кубической периодической решетки со стороной $a$. Декартовыми координатами узлов решетки будут упорядоченные наборы чисел $\{(\alpha a,\beta a,\gamma a);\,\alpha,\beta,\gamma\in\mathbb{Z}\}$. Полуоси включений обозначим через $d_1$ и $d_2$ ($d_2>d_1$). Выполним линейное упорядочение узлов таким же образом, как в параграфе 6.1.\sloppy

В новой нумерации точка $O_{\alpha,\beta,\gamma}$ обозначается через $O_j$ (см.~рис.~\ref{f:11:1b}). С каждой точкой $O_j$ свяжем локальные декартовую $(x_j,y_j,z_j)$ и сонаправленную с ней сжатую сфероидальную системы координат $(\tilde\xi_j,\tilde\eta_j,\varphi_j)$. Считается, что декартовые системы координат с началами в точках $O_j$ сонаправлены.

Будем рассматривать задачу упругого деформирования пространства со сжатыми сфероидальными включениями под действием нагрузки, приложенной на бесконечности (одноосное, двуосное или всестороннее растяжения упругого пространства).

Соотношения между координатами можно описать формулами~\eqref{eq:11:22k}, \eqref{eq:11:23k}.

Предполагается, что упругие постоянные включений равны $(\sigma_j,G_j)$. Упругие постоянные матрицы будем считать равными $(\sigma,G)$.

Граничные условия в рассматриваемой задаче представляют собой условия сопряжения полей перемещений и напряжений на поверхностях $\Gamma_j$. Для того, чтобы их записать, представим вектор перемещений в упругом пространстве в виде

\begin{equation}
{\bf{U}} = \left\{ {\begin{array}{*{20}{l}}
{{\bf{\tilde U}}_j^ - ,\quad \left( {x,y,z} \right) \in \omega_j,}\\
{{{{\bf{\tilde U}}}^ + } + {{\bf{U}}_0},\quad \left( {x,y,z} \right) \in\Omega\backslash {\bigcup\limits_j\omega_j},}
\end{array}} \right.
\label{eq:11:23l}
\end{equation}

\noindent где $\omega_j = \left\{ {\left( {{\tilde\xi_j},{\tilde\eta _j},{\varphi _j}} \right):\, {\tilde\xi_j} < {\tilde\xi_{0j}}} \right\}$. Тогда условия сопряжения принимают следующий вид:

\begin{equation}
\left( {{{{\bf{\tilde U}}}^ + } + {{\bf{U}}_0}} \right){|_{{\Gamma _j}}} = {\bf{\tilde U}}_j^ - {|_{{\Gamma _j}}},
\label{eq:11:24l}
\end{equation}

\begin{equation}
\left( {{\bf{F\tilde U^+}} + {\bf{F}}{{\bf{U}}_0}} \right){|_{{\Gamma _j}}} = {\bf{F}}{{\bf{\tilde U^-}}_j}{|_{{\Gamma _j}}},\qquad {\kern 1pt} j = 1,{\mkern 1mu} {\kern 1pt} 2,\dots.
\label{eq:11:25l}
\end{equation}

\noindent Условия на бесконечности задаются формулами~\eqref{eq:11:1k}~--- \eqref{eq:11:3k}.

Решение задачи будем искать в виде

\begin{equation}
{\bf{\tilde U^+}} = \sum\limits_{j = 1}^\infty {\sum\limits_{s = 1}^3 {\sum\limits_{n = 0}^\infty  {\sum\limits_{m=-n}^{n} {a_{s,n,m}^{(j)}} } } } {\bf{\tilde U}}_{s,n,m}^{ + (6)}\left( {{\tilde\xi_j},{\tilde\eta _j},{\varphi _j}} \right),
\label{eq:11:26l}
\end{equation}

\begin{equation}
\mathbf{\tilde U}_j^- = {\sum\limits_{s = 1}^3 {\sum\limits_{n = 0}^\infty  {\sum\limits_{m=-n}^{n} {b_{s,n,m}^{(j)}} } } } {\bf{\tilde U}}_{s,n,m}^{ - (6)}\left( {{\tilde\xi_j},{\tilde\eta _j},{\varphi _j}} \right),
\label{eq:11:27l}
\end{equation}

\noindent где $G$, $\sigma$~--- модуль сдвига и коэффициент Пуассона упругого пространства; $a_{s,n,m}^{(j)}$, $b_{s,n,m}^{(j)}$~--- неизвестные коэффициенты; перемещения $\mathbf{\tilde U}_{s,n,m}^{\pm(6)}$ приведены в формулах~\eqref{eq:1:89o}~--- \eqref{eq:1:99o}.

Для представления вектора перемещений $\mathbf{\tilde U}^+$ в системах координат с началами $O_j$ можем использовать формулу~\eqref{eq:11:31k}.

В силу периодичности задачи вклад каждого слагаемого в формулах~\eqref{eq:11:26l}, \eqref{eq:11:27l} будет одинаковым, поэтому можно считать, что $a_{s,n,m}^{(j)}=\\=a_{s,n,m}$, $b_{s,n,m}^{(j)}=b_{s,n,m}$.

После перехода к напряжениям в формуле~\eqref{eq:11:31k} и удовлетворения условиям~\eqref{eq:11:24l}, \eqref{eq:11:25l} получаем бесконечную систему линейных алгебраических уравнений относительно неизвестных $a_{s,n,m}$, $b_{s,n,m}$:

\begin{multline}
\sum\limits_{s=1}^3 a_{s,n,m}\tilde E_{s,n,m}^{+(k)}(\sigma)+\tilde E_{s,n,m}^{-(k)}(\sigma)\sum\limits_{t=1}^3\sum\limits_{k=0}^\infty\sum\limits_{l=-k}^k a_{t,k,l}\sum\limits_{\alpha\neq j}\tilde T_{t,k,l,\alpha}^{s,n,m,j}= \\
=-E_{n,m}^{(k)}+\sum\limits_{s=1}^3 b_{s,n,m}\tilde E_{s,n,m}^{-(k)}(\sigma_j);
\label{eq:11:28l}
\end{multline}

\begin{multline}
\sum\limits_{s=1}^3 a_{s,n,m}\tilde F_{s,n,m}^{+(k)}(\sigma)+\tilde F_{s,n,m}^{-(k)}(\sigma)\sum\limits_{t=1}^3\sum\limits_{k=0}^\infty\sum\limits_{l=-k}^k a_{t,k,l}\sum\limits_{\alpha\neq j}\tilde T_{t,k,l,\alpha}^{s,n,m,j}= \\
=F_{n,m}^{(k)}+\frac{G_j}{G}\sum\limits_{s=1}^3 b_{s,n,m}\tilde F_{s,n,m}^{-(k)}(\sigma_j);
\label{eq:11:29l}
\end{multline}

\begin{equation}
n,m \in\mathbb{Z} :\quad n \ge 0,\quad |m| \le n,\quad k =  - 1,{\mkern 1mu} {\kern 1pt} 0,{\mkern 1mu} {\kern 1pt} 1,
\end{equation}

\noindent где $\tilde E_{s,n,m}^{\pm(k)}$~--- компоненты вектора перемещений $\mathbf{\tilde U}_{s,n,m}^{\pm(6)}$ на поверхности $\tilde\xi_j=\\=\tilde\xi_{0j}$; $\tilde F_{s,n,m}^{\pm(k)}$~--- компоненты вектора напряжений на поверхности $\tilde\xi_j=\tilde\xi_{0j}$, отвечающего вектору перемещения $\tilde U_{s,n,m}^{\pm(6)}$:
$$
\mathbf{\tilde U}_{s,n,m}^{\pm(6)}=\tilde E_{s,n,m}^{\pm(-1)}S_n^{m-1}\mathbf{e}_{-1}+\tilde E_{s,n,m}^{\pm(1)}S_n^{m+1}\mathbf{e}_1+\tilde E_{s,n,m}^{\pm(0)}S_n^m\mathbf{e}_0;
$$
$$
\mathbf{F\tilde U}_{s,n,m}^{\pm(6)}=\frac{\tilde H_j}{\tilde c}\bigg(\tilde F_{s,n,m}^{\pm(-1)}S_n^{m-1}\mathbf{e}_{-1}+\tilde F_{s,n,m}^{\pm(1)}S_n^{m+1}\mathbf{e}_1+\tilde F_{s,n,m}^{\pm(0)}S_n^m\mathbf{e}_0\bigg);
$$

\begin{equation*}
E_{n,m}^{(k)} =\frac{T\tilde c}{2G}\delta_{n1}\delta_{m0}\bigg[-\frac{2\sigma}{1+\sigma}\,\mathrm{ch}\tilde\xi_{0j}\delta_{k,-1}+\frac{\sigma}{1+\sigma}\,\mathrm{ch}\tilde\xi_{0j}\delta_{k1}+\frac{1}{1+\sigma}\,\mathrm{sh}\tilde\xi_{0j}\delta_{k0}\bigg],
\end{equation*}

\begin{equation*}
F_{n,m}^{(k)} =  -\frac{Td_2}{2G}{\delta _{n1}}{\delta _{m0}}{\delta _{k0}}\quad\text{(одноосное растяжение)};
\end{equation*}

\begin{equation*}
E_{n,m}^{(k)} =\frac{T\tilde c}{2G}\delta_{n1}\delta_{m0}\bigg[\frac{2-2\sigma}{1+\sigma}\,\mathrm{ch}\tilde\xi_{0j}\delta_{k,-1}-\frac{1-\sigma}{1+\sigma}\,\mathrm{ch}\tilde\xi_{0j}\delta_{k1}-\frac{2\sigma}{1+\sigma}\,\mathrm{sh}\tilde\xi_{0j}\delta_{k0}\bigg],
\end{equation*}

\begin{equation*}
F_{n,m}^{(k)} =  -\frac{Td_1}{2G}{\delta _{n1}}{\delta _{m0\,}}(2{\delta _{k, - 1}} - {\delta _{k1}})\quad\text{(двуосное растяжение)};
\end{equation*}

\begin{equation*}
E_{n,m}^{(k)} =\frac{T\tilde c}{2G}\delta_{n1}\delta_{m0}\frac{1-2\sigma}{1+\sigma}\bigg[2\,\mathrm{ch}\tilde\xi_{0j}\delta_{k,-1}-\mathrm{ch}\tilde\xi_{0j}\delta_{k1}+\mathrm{sh}
\tilde\xi_{0j}\delta_{k0}\bigg],
\end{equation*}

\begin{equation*}
F_{n,m}^{(k)} =  -\frac{T}{2G}{\delta _{n1}}{\delta _{m0\,}}(d_2\delta_{k0}+2d_1{\delta _{k, - 1}} - d_1{\delta _{k1}})\quad\text{(всестороннее растяжение)}.
\end{equation*}

\begin{figure}[h!]
\centering\footnotesize
\parbox[b]{7.5cm}{\centering\includegraphics[width=8.1cm]{periodic-oblate-inc27-a-d50-g25-t1-sig_x.pdf}
\caption{Напряжения $\sigma_x/T$ на линии $AB$ в зависимости от расстояния между включениями при одноосном растяжении
\label{f:11:43}}}\hfil\hfil
\parbox[b]{7.5cm}{\centering\includegraphics[width=7.6cm]{periodic-oblate-inc27-a-d50-g25-t2-sig_x.pdf}
\caption{Напряжения $\sigma_x/T$ на линии $AB$ в зависимости от расстояния между включениями при двуосном растяжении
\label{f:11:44}}}
\end{figure}

\begin{figure}[h!]
\centering\footnotesize
\parbox[b]{7.5cm}{\centering\includegraphics[width=7.8cm]{periodic-oblate-inc27-a-d50-g25-t1-sig_y.pdf}
\caption{Напряжения $\sigma_y/T$ на линии $AB$ в зависимости от расстояния между включениями при одноосном растяжении
\label{f:11:45}}}\hfil\hfil
\parbox[b]{7.5cm}{\centering\includegraphics[width=7.8cm]{periodic-oblate-inc27-a-d50-g25-t2-sig_y.pdf}
\caption{Напряжения $\sigma_y/T$ на линии $AB$ в зависимости от расстояния между включениями при двуосном растяжении
\label{f:11:46}}}
\end{figure}

Явный вид компонент $\tilde E_{s,n,m}^{\pm(k)}$ и $\tilde F_{s,n,m}^{\pm(k)}$ не приводится ввиду их громоздкости. Они получаются из формул~\eqref{eq:1:73}~--- \eqref{eq:1:75}, \eqref{eq:1:89o}~--- \eqref{eq:1:99o}, \eqref{eq:10:25o}~--- \eqref{eq:10:27o}.

На рис.~\ref{f:11:43}~--- \ref{f:11:48} приведены нормальные напряжения на линии $AB$ в зависимости от расстояния между включениями при одноосном и двуосном растяжениях упругого пространства при $d_1/d_2=0.5$, $\sigma=0.38$, $\sigma_j=0.21$, $G_j/G=25$.
При одноосном растяжении основной вклад в тензор напряжений вносят напряжения $\sigma_x/T$ и $\sigma_z/T$. Напряжения $\sigma_x/T$ и $\sigma_y/T$ являются сжимающими на отрезке $AB$. Их модуль растет с приближением включений друг к другу. Напряжения $\sigma_z/T$ меняют знак на отрезке $AB$.

\begin{figure}[h!]
\centering\footnotesize
\parbox[b]{7.5cm}{\centering\includegraphics[width=7.8cm]{periodic-oblate-inc27-a-d50-g25-t1-sig_z.pdf}
\caption{Напряжения $\sigma_z/T$ на линии $AB$ в зависимости от расстояния между включениями при одноосном растяжении
\label{f:11:47}}}\hfil\hfil
\parbox[b]{7.5cm}{\centering\includegraphics[width=7.8cm]{periodic-oblate-inc27-a-d50-g25-t2-sig_z.pdf}
\caption{Напряжения $\sigma_z/T$ на линии $AB$ в зависимости от расстояния между включениями при двуосном растяжении
\label{f:11:48}}}
\end{figure}

\begin{figure}[ht!]
\centering\footnotesize
\parbox[b]{7.5cm}{\centering\includegraphics[width=8cm]{oblate-inc27-8-a25-g25-t1.pdf}
\caption{Сравнение напряжения на линии $AB$ для периодической и тетрагональной структур при одноосном растяжении
\label{f:11:49}}}\hfil\hfil
\parbox[b]{7.5cm}{\centering\includegraphics[width=7.8cm]{oblate-inc27-8-a25-g25-t2.pdf}
\caption{Сравнение напряжения на линии $AB$ для периодической и тетрагональной структур при двуосном растяжении
\label{f:11:50}}}
\end{figure}

При двуосном растяжении основной вклад в тензор напряжений вносят напряжения $\sigma_x/T$, однако напряжения $\sigma_y/T$ и $\sigma_z/T$ являются значимыми. Для напряжений $\sigma_y/T$ и $\sigma_z/T$ область концентрации находится вблизи границы включений. При приближении включений друг к другу все напряжения возрастают.

На рис.~\ref{f:11:49}, \ref{f:11:50} представлено сравнение нормальных напряжений на линии $AB$ для периодической (27 включений) и тетрагональной (8 включений) структур при одноосном и двуосном растяжениях упругого пространства. Графики показывают, что напряжения $\sigma_y/T$ и $\sigma_z/T$ практически совпадают.

\end{russian}
%!TeX root = book.tex
%!TEX TS-program = lualatex

\begin{russian}
\chapter[Эффективные упругие модули пористых и композиционных материалов зернистой структуры]{Эффективные упругие модули пористых и композиционных материалов зернистой структуры}\chaptermark{Эффективные упругие модули композитов}

\section[Эффективные упругие модули материалов со сферическими порами]{Эффективные упругие модули материалов со сферическими порами\sectionmark{Эффективные упругие модули материалов со сферическими порами}}\sectionmark{Эффективные упругие модули материалов со сферическими порами}

Будем рассматривать пористый материал как упругое пространство $\Omega$ с бесконечной системой сферических полостей $\{\omega_{\alpha\beta\gamma}\}_{\alpha,\beta,\gamma=-\infty}^\infty$, центры которых расположены в узлах $\{O_{\alpha\beta\gamma}\}_{\alpha,\beta,\gamma=-\infty}^\infty$ кубической периодической решетки со стороной $a$. Декартовыми координатами узлов решетки будут упорядоченные наборы чисел $\{(\alpha a,\beta a,\gamma a);\,\alpha,\beta,\gamma\in\mathbb{Z}\}$. Радиусы полостей обозначим через $R$. Выделим отдельную представительскую ячейку пористого материала (рис.~\ref{f:12:1})
$$
\Omega_{\alpha\beta\gamma}=\bigg\{(x_\alpha,y_\beta,z_\gamma): |x_\alpha|\le\dfrac{a}{2},|y_\beta|\le\dfrac{a}{2},|z_\gamma|\le\dfrac{a}{2}\bigg\}.
$$
Здесь ($x_\alpha$, $y_\beta$, $z_\gamma$)~--- локальная декартовая система координат с началом в точке $O_{\alpha\beta\gamma}$.

\begin{figure}[h!]
\centering
\includegraphics[width=8cm]{cell-spheres.pdf}
\caption{Представительская ячейка пористого материала со сферическими порами}
\label{f:12:1}
\end{figure}

%\begin{equation}
%\begin{cases}
%\dfrac{E_0}{G}\langle\varepsilon_x\rangle+\sigma_0\bigg(\langle\sigma_x\rangle+
%\langle\sigma_z\rangle\bigg)=\langle\sigma_x\rangle, \\
%\dfrac{E_0}{G}\langle\varepsilon_z\rangle+2\sigma_0\langle\sigma_x\rangle=
%\langle\sigma_z\rangle.
%\end{cases}
%\end{equation}

Удельную упругую энергию представительской ячейки можно записать в виде

\begin{equation}
W=\frac{1}{2V}\int\limits_{\Omega_k}\sigma_{ij}\varepsilon_{ij}dv,
\label{eq:12:1}
\end{equation}
где $\Omega_k$~--- представительская ячейка; $V$~--- объем этой ячейки; $\sigma_{ij}$, $\varepsilon_{ij}$~--- компоненты тензоров напряжений и деформаций соответственно. Здесь и далее принято правило суммирования по повторяющимся индексам. Используя формулу Гаусса~--- Остроградского, объемный интеграл в формуле~\eqref{eq:12:1} можно преобразовать к поверхностному

\begin{equation}
W=\frac{1}{2V}\int\limits_S\sigma_{ij}u_i n_j ds,
\label{eq:12:2}
\end{equation}
где $S$~--- поверхность ячейки $\Omega_k$; $(n_j)_{j=1}^3$~--- направляющие косинусы вектора нормали к поверхности. Заметим, что интеграл по поверхности полости обращается в ноль, так как нормальные напряжения на границе полости равны нулю.
 
В монографии~\cite{Vanin1985} в предположении, что троякопериодичное двухфазное упругое тело находится в однородном поле напряжений, получены два представления удельной энергии деформации через компоненты усредненных напряжений и деформаций в виде

\begin{equation}
W=\frac{1}{2}\langle\sigma_{ij}\rangle\langle\varepsilon_{ij}\rangle,
\label{eq:12:3}
\end{equation}

\begin{equation}
W=\frac{1}{2}c_{ijkl}\langle\sigma_{kl}\rangle\langle\sigma_{ij}\rangle,
\label{eq:12:4}
\end{equation}
где $c_{ijkl}$~--- эффективные упругие постоянные.

Будем использовать результаты задачи упругого деформирования пространства с периодической системой сферических полостей $\Omega\backslash\bigg\{\bigcup\limits_{\alpha,\beta,\gamma}\omega_{\alpha\beta\gamma}\bigg\}$ под действием нагрузки, приложенной на бесконечности (одноосное, двуосное или всестороннее растяжения упругого пространства), приведенные в параграфе 6.1, для вычисления эффективных упругих модулей пористого материала со сферическими порами. В силу того, что на\-пряжен\-но-де\-фор\-ми\-ро\-ван\-ное состояние упругого пространства с периодической системой полостей является однородным, можно считать, что материал с периодической системой пор является изотропным. Как известно, изотропный материал характеризуется двумя независимыми упругими модулями. Для эффективных упругих модулей пористого материала будем использовать следующие обозначения: $K_0$~--- объемный модуль, $E_0$~--- модуль Юнга, $G_0$~--- модуль сдвига, $\sigma_0$~--- коэффициент Пуассона.

При вычислении эффективных упругих модулей используем подход, предложенный в работе~\cite{Vanin1985}.

Пусть упругое пространство с периодической системой полостей подвергается всестороннему растяжению $\sigma_r^\infty=T$, приложенному на бесконечности. При всестороннем растяжении упругого пространства в силу однородности напряженного состояния при вычислении удельной энергии деформации фактически остается одно слагаемое в формуле~\eqref{eq:12:2}

\begin{equation}
W=\frac{1}{2V}\int\limits_S\sigma_{r}u_r ds.
\label{eq:12:5}
\end{equation}
Аналогичная ситуация имеет место в формуле~\eqref{eq:12:3}~\cite{Vanin1985}

\begin{equation}
W=\frac{3}{2}\langle\sigma\rangle\langle\varepsilon\rangle,
\label{eq:12:6}
\end{equation}
где $\langle\sigma\rangle$~--- среднее всестороннее растяжение; $\langle\varepsilon\rangle$~--- средняя деформация рассматриваемого материала при всестороннем растяжении. Приведенные величины связаны соотношением $\langle\sigma\rangle=3K_0\langle\varepsilon\rangle$.

Деформируем представительскую ячейку в равновеликий шар. В данном случае $S$~--- поверхность этого шара. Предполагается, что поверхность $S$ не пересекает границ полостей $\omega_{\alpha\beta\gamma}$. Заменяя под интегралом в формуле~\eqref{eq:12:5} напряжения $\sigma_r$ на средние напряжения
$$
\langle\sigma\rangle=\frac{1}{S}\int\limits_S\sigma_r ds,
$$
с учетом соотношения между средними напряжениями и деформациями получаем формулу для вычисления эффективного объемного модуля

\begin{equation}
K_0=\frac{\langle\sigma\rangle}{\dfrac{1}{V}\int\limits_S u_r ds}.
\label{eq:12:7}
\end{equation}

Вычислим значение интеграла

\begin{equation}
\int\limits_S u_r ds=\int\limits_0^{2\pi}\int\limits_0^\pi u_r R_1^2\sin\theta d\theta d\varphi,
\label{eq:12:8}
\end{equation}
где $R_1$~--- радиус сферы $S$. Для вычисления интеграла~\eqref{eq:12:8} используем результат решения краевой задачи для периодической системы сферических полостей в упругом пространстве~\eqref{eq:11:8s}:

\begin{equation}
u_r=\mathbf{U}\cdot\mathbf{e}_r=\mathbf{U}_0\cdot\mathbf{e}_r+\sum\limits_{j=1}^\infty
\sum\limits_{s=1}^3\sum\limits_{n=0}^\infty\sum\limits_{m=-n}^n a_{s,n,m}^{(j)}\mathbf{\tilde U}_{s,n,m}^{+(4)}(r_j,\theta_j,\varphi_j)\cdot\mathbf{e}_r.
\label{eq:12:9}
\end{equation}

Представим перемещения $\mathbf{\tilde U}_{s,n,m}^{+(4)}(r_j,\theta_j,\varphi_j)$ через $\mathbf{\tilde U}_{s,n,m}^{-(4)}(r_1,\theta_1,\varphi_1)$ с помощью теоремы сложения~\eqref{eq:1:99t}:

\begin{multline}
\sum\limits_{j=1}^\infty
\sum\limits_{s=1}^3\sum\limits_{n=0}^\infty\sum\limits_{m=-n}^n a_{s,n,m}\mathbf{\tilde U}_{s,n,m}^{+(4)}(r_j,\theta_j,\varphi_j)= \\
=\sum\limits_{s=1}^3\sum\limits_{n=0}^\infty\sum\limits_{m=-n}^n\bigg\lbrack a_{s,n,m}\mathbf{\tilde U}_{s,n,m}^{+(4)}(r_1,\theta_1,\varphi_1)+ \\
+\mathbf{\tilde U}_{s,n,m}^{-(4)}(r_1,\theta_1,\varphi_1)\sum\limits_{j=2}^\infty\sum\limits_{t=1}^3\sum\limits_{k=0}^\infty\sum\limits_{l=-k}^k a_{t,k,l}\tilde T_{t,k,l,j}^{s,n,m,1}\bigg\rbrack.
\label{eq:12:10}
\end{multline}

Для вычисления $\mathbf{\tilde U}_{s,n,m}^{\pm(4)}\cdot\mathbf{e}_r$ заметим, что
$$
\mathbf{e}_{-1}\cdot\mathbf{e}_r=\frac{1}{2}(\mathbf{e}_x+i\mathbf{e}_y)\cdot(\mathbf{e}_x\sin\theta\cos\varphi+
\mathbf{e}_y\sin\theta\sin\varphi+\mathbf{e}_z\cos\theta)=\frac{1}{2}\sin\theta e^{i\varphi},
$$
$$
\mathbf{e}_1\cdot\mathbf{e}_r=\frac{1}{2}(\mathbf{e}_x-i\mathbf{e}_y)\cdot(\mathbf{e}_x\sin\theta\cos\varphi+
\mathbf{e}_y\sin\theta\sin\varphi+\mathbf{e}_z\cos\theta)=\frac{1}{2}\sin\theta e^{-i\varphi},
$$
$$
\mathbf{e}_0\cdot\mathbf{e}_r=\cos\theta.
$$

Справедливы следующие формулы:

\begin{equation}
\mathbf{U}_{1,n,m}^{\pm(4)}\cdot\mathbf{e}_r=-u_{n,m-1}^{\pm(4)}\frac{1}{2}\sin\theta e^{i\varphi}+u_{n,m+1}^{\pm(4)}\frac{1}{2}\sin\theta e^{-i\varphi}\mp u_{n,m}^{\pm(4)}\cos\theta;
\label{eq:12:11}
\end{equation}

\begin{multline}
\mathbf{U}_{2,n,m}^{+(4)}\cdot\mathbf{e}_r=\frac{1}{2n+3}\bigg\{-(n-m+2)(n+m)u_{n,m-1}^{+(4)}\frac{1}{2}\sin\theta e^{i\varphi}+ \\
+(n-m)(n+m+2)u_{n,m+1}^{+(4)}\frac{1}{2}\sin\theta e^{-i\varphi}- \\
-\bigg\lbrack(n-m+1)(n+m+1)+\chi(2n+3)\bigg\rbrack u_{n,m}^{+(4)}\cos\theta+ \\
+(r^2-R^2)\bigg(-u_{n+2,m-1}^{+(4)}\frac{1}{2}\sin\theta e^{i\varphi}+ \\
+u_{n+2,m+1}^{+(4)}\frac{1}{2}\sin\theta e^{-i\varphi}-u_{n+2,m}^{+(4)}\cos\theta\bigg)\bigg\};
\label{eq:12:12}
\end{multline}

\begin{multline}
\mathbf{U}_{2,n,m}^{-(4)}\cdot\mathbf{e}_r=\frac{1}{2n-1}\bigg\{-(n-m+1)(n+m-1)u_{n,m-1}^{-(4)}\frac{1}{2}\sin\theta e^{i\varphi}+ \\
+(n-m-1)(n+m+1)u_{n,m+1}^{-(4)}\frac{1}{2}\sin\theta e^{-i\varphi}+ \\
+\bigg\lbrack(n-m)(n+m)-\chi(2n-1)\bigg\rbrack u_{n,m}^{-4)}\cos\theta+ \\
+(r^2-R^2)\bigg(-u_{n-2,m-1}^{-(4)}\frac{1}{2}\sin\theta e^{i\varphi}+ \\
+u_{n-2,m+1}^{-(4)}\frac{1}{2}\sin\theta e^{-i\varphi}+u_{n-2,m}^{-(4)}\cos\theta\bigg)\bigg\};
\label{eq:12:13}
\end{multline}

\begin{equation}
\mathbf{U}_{3,n,m}^{\pm(4)}\cdot\mathbf{e}_r=u_{n,m-1}^{\pm(4)}\frac{1}{2}\sin\theta e^{i\varphi}+u_{n,m+1}^{\pm(4)}\frac{1}{2}\sin\theta e^{-i\varphi}.
\label{eq:12:14}
\end{equation}

Из приведенных формул вытекает:

\begin{equation}
\frac{1}{V}\int\limits_S \mathbf{U}_{1,n,m}^{+(4)}\cdot\mathbf{e}_r ds=-\frac{4\pi}{V}\delta_{n1}\delta_{m0};
\label{eq:12:15}
\end{equation}

\begin{equation}
\frac{1}{V}\int\limits_S \mathbf{U}_{2,n,m}^{+(4)}\cdot\mathbf{e}_r ds=-\frac{4\pi(2+\chi)}{3V}\delta_{n1}\delta_{m0};
\label{eq:12:16}
\end{equation}

\begin{equation}
\frac{1}{V}\int\limits_S \mathbf{U}_{3,n,m}^{+(4)}\cdot\mathbf{e}_r ds=0;
\label{eq:12:17}
\end{equation}

\begin{equation}
\frac{1}{V}\int\limits_S \mathbf{U}_{1,n,m}^{-(4)}\cdot\mathbf{e}_r ds=0;
\label{eq:12:18}
\end{equation}

\begin{equation}
\frac{1}{V}\int\limits_S \mathbf{U}_{2,n,m}^{-(4)}\cdot\mathbf{e}_r ds=-\frac{4\pi(-1+\chi)R_1^3}{3V}\delta_{n1}\delta_{m0};
\label{eq:12:19}
\end{equation}

\begin{equation}
\frac{1}{V}\int\limits_S \mathbf{U}_{3,n,m}^{-(4)}\cdot\mathbf{e}_r ds=0;
\label{eq:12:20}
\end{equation}

\begin{equation}
\frac{1}{S}\int\limits_S \mathbf{FU}_{1,n,m}^{+(4)}\cdot\mathbf{e}_r ds=2G\frac{8\pi}{R_1 S}\delta_{n1}\delta_{m0};
\label{eq:12:21}
\end{equation}

\begin{equation}
\frac{1}{S}\int\limits_S \mathbf{FU}_{2,n,m}^{+(4)}\cdot\mathbf{e}_r ds=2G\frac{8\pi(2+\chi)}{3R_1 S}\delta_{n1}\delta_{m0};
\label{eq:12:22}
\end{equation}

\begin{equation}
\frac{1}{S}\int\limits_S \mathbf{FU}_{3,n,m}^{+(4)}\cdot\mathbf{e}_r ds=0;
\label{eq:12:23}
\end{equation}

\begin{equation}
\frac{1}{S}\int\limits_S \mathbf{FU}_{1,n,m}^{-(4)}\cdot\mathbf{e}_r ds=0;
\label{eq:12:24}
\end{equation}

\begin{equation}
\frac{1}{S}\int\limits_S \mathbf{FU}_{2,n,m}^{-(4)}\cdot\mathbf{e}_r ds=-2G\frac{8\pi(1+\sigma)}{3R_1 S}\delta_{n1}\delta_{m0};
\label{eq:12:25}
\end{equation}

\begin{equation}
\frac{1}{S}\int\limits_S \mathbf{FU}_{3,n,m}^{-(4)}\cdot\mathbf{e}_r ds=0,
\label{eq:12:26}
\end{equation}
где $\mathbf{FU}_{s,n,m}^{\pm(4)}$~--- напряжения, отвечающие перемещениям $\mathbf{U}_{s,n,m}^{\pm(4)}$ на поверхности $r_1=R_1$.

Теперь из соотношения~\eqref{eq:12:7} получаем формулу для эффективного объемного модуля:

\begin{equation}
\frac{K_0}{K}=\frac{1+\zeta\bigg(2 \tilde a_{1,1,0}+\dfrac{2(5-4\sigma)}{3} \tilde a_{2,1,0}-\dfrac{2(1+\sigma)}{3}\Gamma\bigg)}{1-(1+\sigma)\zeta\bigg(\dfrac{1}{1-2\sigma} \tilde a_{1,1,0}+\dfrac{(5-4\sigma)}{3(1-2\sigma)} \tilde a_{2,1,0}+\dfrac{2}{3}\Gamma\bigg)};
\label{eq:12:27}
\end{equation}
$$
\Gamma=\sum\limits_{t=1}^3\sum\limits_{k=0}^\infty\sum\limits_{l=-k}^k \tilde a_{t,k,l}\sum\limits_{j=2}^\infty\tilde T_{t,k,l,j}^{2,1,0,1},
$$
где $\tilde a_{t,k,l}=\dfrac{2G}{TR^3}a_{t,k,l}$, $a_{t,k,l}$~--- решения линейной системы~\eqref{eq:11:sys}; $\zeta$~--- объемное содержание полостей в материале.

Если решаемую периодическую задачу заменить задачей об определении напряжений и деформаций в упругом пространстве с одной сферической полостью, находящемся под действием всестороннего растяжения, то нужно принять $\Gamma=0$, $\tilde a_{1,1,0}=-1/2$, $\tilde a_{2,1,0}=0$. Тогда формула~\eqref{eq:12:27} превращается в соотношение

\begin{equation}
\frac{K_0}{K}=\frac{(1-\zeta)(2-4\sigma)}{2-4\sigma+(1+\sigma)\zeta},
\label{eq:12:28}
\end{equation}  
которое совпадает с известным результатом, приведенным в монографии~\cite{Vanin1985}.

Таким образом, формула~\eqref{eq:12:27} дает обобщение соотношения~\eqref{eq:12:28}, учитывающее периодическую структуру полостей в материале.

\begin{figure}[h!]
\centering
\includegraphics[width=7.5cm]{porous-spheres-k.pdf}
\caption{Зависимость объемного модуля от объемного содержания полостей в материале}
\label{f:12:2}
\end{figure}

На рис.~\ref{f:12:2} приведены зависимости объемного упругого модуля от объемного содержания полостей в материале. Представлены графики в случаях одной полости в материале и периодической системы полостей в материале. Отличия значений объемных модулей для этих случаев становятся заметными с увеличением объемного содержания полостей. Максимальное отличие достигает 5~\% при объемном содержании полостей $\zeta=0.45$. Таким же образом могут быть вычислены остальные эффективные упругие модули.

%***************************************************************************************

\section[Эффективные упругие модули материалов со сферическими включениями]{Эффективные упругие модули материалов со сферическими включениями\sectionmark{Эффективные упругие модули материалов со сферическими включениями}}\sectionmark{Эффективные упругие модули материалов со сферическими включениями}

Будем рассматривать композиционный материал как упругое пространство $\Omega$ с бесконечной системой сферических включений $\{\omega_{\alpha\beta\gamma}\}_{\alpha,\beta,\gamma=-\infty}^\infty$, центры которых расположены в узлах $\{O_{\alpha\beta\gamma}\}_{\alpha,\beta,\gamma=-\infty}^\infty$ кубической периодической решетки со стороной $a$. Декартовыми координатами узлов решетки будут упорядоченные наборы чисел $\{(\alpha a,\beta a,\gamma a);\,\alpha,\beta,\gamma\in\mathbb{Z}\}$. Радиусы включений обозначим через $R$. Считаем, что материалы матрицы и включений имеют упругие постоянные $(\sigma, G)$ и $(\sigma_1, G_1)$. Выделим отдельную представительскую ячейку зернистого композита (см.~рис.~\ref{f:12:1})
$$
\Omega_{\alpha\beta\gamma}=\bigg\{(x_\alpha,y_\beta,z_\gamma): |x_\alpha|\le\dfrac{a}{2},|y_\beta|\le\dfrac{a}{2},|z_\gamma|\le\dfrac{a}{2}\bigg\}.
$$
Здесь ($x_\alpha$, $y_\beta$, $z_\gamma$)~--- локальная декартовая система координат с началом в точке $O_{\alpha\beta\gamma}$. Предполагается, что включения находятся в условиях идеального контакта с матрицей.

%\begin{figure}[h!]
%\centering
%\includegraphics[width=8cm]{cell-spheres.pdf}
%\caption{Представительская ячейка пористого материала со сферическими порами}
%\label{f:12:1}
%\end{figure}

%\begin{equation}
%\begin{cases}
%\dfrac{E_0}{G}\langle\varepsilon_x\rangle+\sigma_0\bigg(\langle\sigma_x\rangle+
%\langle\sigma_z\rangle\bigg)=\langle\sigma_x\rangle, \\
%\dfrac{E_0}{G}\langle\varepsilon_z\rangle+2\sigma_0\langle\sigma_x\rangle=
%\langle\sigma_z\rangle.
%\end{cases}
%\end{equation}

Удельную упругую энергию представительской ячейки можно записать в виде

\begin{equation}
W=\frac{1}{2V}\int\limits_{\Omega_k}\sigma_{ij}\varepsilon_{ij}dv,
\label{eq:12:29}
\end{equation}
где $\Omega_k$~--- представительская ячейка; $V$~--- объем этой ячейки; $\sigma_{ij}$, $\varepsilon_{ij}$~--- компоненты тензоров напряжений и деформаций соответственно. Здесь и далее принято правило суммирования по повторяющимся индексам. Используя формулу Гаусса~--- Остроградского, объемный интеграл в формуле~\eqref{eq:12:29} можно преобразовать к поверхностному

\begin{equation}
W=\frac{1}{2V}\int\limits_S\sigma_{ij}u_i n_j ds,
\label{eq:12:30}
\end{equation}
где $S$~--- поверхность ячейки $\Omega_k$; $(n_j)_{j=1}^3$~--- направляющие косинусы вектора нормали к поверхности. Заметим, что интеграл по поверхности включения отсутствует в формуле~\eqref{eq:12:30}, так как два таких интеграла для включения и для матрицы взаимно уничтожаются.

При вычислении упругих модулей применим подход, предложенный в монографии~\cite{Vanin1985}, и формулы для удельной упругой энергии~\eqref{eq:12:3}, \eqref{eq:12:4}. 

%\begin{equation}
%W=\frac{1}{2}\langle\sigma_{ij}\rangle\langle\varepsilon_{ij}\rangle,
%\label{eq:12:31}
%\end{equation}
%
%\begin{equation}
%W=\frac{1}{2}c_{ijkl}\langle\sigma_{kl}\rangle\langle\sigma_{ij}\rangle,
%\label{eq:12:32}
%\end{equation}
%где $c_{ijkl}$~--- эффективные упругие постоянные.

Будем использовать результаты задачи упругого деформирования пространства с периодической системой сферических включений под действием нагрузки, приложенной на бесконечности (одноосное, двуосное или всестороннее растяжения упругого пространства), приведенные в параграфе 6.2, для вычисления эффективных упругих модулей зернистого композита со сферическими зернами. В силу того, что на\-пряжен\-но-де\-фор\-ми\-ро\-ван\-ное состояние упругого пространства с периодической системой включений является однородным, можно считать, что материал с периодической системой включений является изотропным. Для эффективных упругих модулей композиционного материала будем использовать следующие обозначения: $K_0$~--- объемный модуль, $E_0$~--- модуль Юнга, $G_0$~--- модуль сдвига, $\sigma_0$~--- коэффициент Пуассона.

Пусть упругое пространство с периодической системой включений подвергается всестороннему растяжению $\sigma_r^\infty=T$, приложенному на бесконечности. Повторяя рассуждения предыдущего параграфа, получаем для эффективного объемного модуля формулу

\begin{equation}
\frac{K_0}{K}=\frac{1+\zeta\bigg(2 \tilde a_{1,1,0}+\dfrac{2(5-4\sigma)}{3} \tilde a_{2,1,0}-\dfrac{2(1+\sigma)}{3}\Gamma\bigg)}{1-(1+\sigma)\zeta\bigg(\dfrac{1}{1-2\sigma} \tilde a_{1,1,0}+\dfrac{(5-4\sigma)}{3(1-2\sigma)} \tilde a_{2,1,0}+\dfrac{2}{3}\Gamma\bigg)},
\label{eq:12:31}
\end{equation}
$$
\Gamma=\sum\limits_{t=1}^3\sum\limits_{k=0}^\infty\sum\limits_{l=-k}^k \tilde a_{t,k,l}\sum\limits_{j=2}^\infty\tilde T_{t,k,l,j}^{2,1,0,1},
$$
где $\tilde a_{t,k,l}=\dfrac{2G}{TR^3}a_{t,k,l}$, $a_{t,k,l}$~--- решения линейной системы~\eqref{eq:11:28}, \eqref{eq:11:29}; $\zeta$~--- объемное содержание полостей в материале.

Если решаемую периодическую задачу заменить задачей об определении напряжений и деформаций в упругом пространстве с одним сферическим включением, находящимся под действием всестороннего растяжения, то нужно принять $\Gamma=0$, $\tilde a_{2,1,0}=0$,
$$
\tilde a_{1,1,0}=\frac{G_1(1+\sigma_1)(1-2\sigma)-G(1+\sigma)(1-2\sigma_1)}{(1+\sigma)\bigg\lbrack G_1(1+\sigma_1)+G(2-4\sigma_1)\bigg\rbrack}.
$$
Тогда формула~\eqref{eq:12:31} превращается в соотношение

\begin{multline}
\frac{K_0}{K}=\frac{1-2\sigma}{1+\sigma}\times \\
\times\frac{G(1-2\sigma_1)(1-\zeta)(2+2\sigma)+G_1(1+\sigma_1)(\zeta(2-4\sigma)+1+\sigma)}{G(1-2\sigma_1)(\zeta(1+\sigma)+2-4\sigma)+G_1(1+\sigma_1)(1-\zeta)(1-2\sigma)},
\label{eq:12:32}
\end{multline}  
которое совпадает с известным результатом, приведенным в монографии~\cite{Vanin1985}.

Таким образом, формула~\eqref{eq:12:31} дает обобщение соотношения~\eqref{eq:12:32}, учитывающее периодическую структуру включений в материале.

\begin{figure}[h!]
\centering\footnotesize
\parbox[b]{7.5cm}{\centering\includegraphics[width=7.3cm]{composite-spheres-g25-k.pdf}
\caption{Зависимость объемного модуля от объемного содержания включений в композите при $G_j/G=25$
\label{f:12:3}}}\hfil\hfil
\parbox[b]{7.5cm}{\centering\includegraphics[width=7.4cm]{composite-spheres-g100-k.pdf}
\caption{Зависимость объемного модуля от объемного содержания включений в композите при $G_j/G=100$
\label{f:12:4}}}
\end{figure}

На рис.~\ref{f:12:3}, \ref{f:12:4} приведены зависимости объемного упругого модуля от объемного содержания включений в композиционном материале при $G_j/G=25$ и $G_j/G=100$. Представлены графики в случаях одного включения в материале и периодической системы включений в материале. Отличия значений объемных модулей для этих случаев становятся заметными с увеличением объемного содержания включений. Максимальное отличие достигает 3.5~\% при $G_j/G=25$ и 4~\% при $G_j/G=100$ в случае объемного содержания включений $\zeta=0.45$. Таким же образом могут быть вычислены остальные эффективные упругие модули.

\end{russian}
%%!TeX root = book.tex
%!TEX TS-program = lualatex

\begin{russian}
\setcounter{secnumdepth}{-1}
\chapter{Заключение}
\markright{Заключение}
\setcounter{secnumdepth}{2}

В отчете приведены результаты исследований второго этапа бюджетной науч\-но-ис\-сле\-до\-ва\-тель\-ской работы, посвященной моделированию на\-пря\-же\-н\-но-де\-фор\-ми\-ро\-ва\-н\-но\-го состояния упругого материала, содержащего полости и включения. Рассмотрены глобальные модели, которые описывают поля напряжений и деформаций в реальных упругих пористых и композиционных материалах в областях между конечным числом  концентраторов напряжений. В качестве таковых рассмотрены: цилиндры, шары, вытянутые или сжатые сфероиды. Поля описываются аналитически точно при помощи базисных решений уравнения Ламе в канонических односвязных областях. Для определения параметров моделей при помощи обобщенного метода Фурье получены операторные уравнения с оптимальными свойствами, которые допускают эффективные численные решения. Приведен строгий аналитический анализ предложенных моделей, в результате которого определена область их эффективного применения. Создано программное обеспечение для численной реализации построенных моделей. На его основе проведен численный анализ и дана визуализация распределения напряжений в некоторых телах в зонах их максимальной концентрации. Исследована скорость сходимости приближенных методов решения операторных уравнений для определения параметров моделей. Проведено сравнение полученных результатов с результатами локальных моделей, исследованных на первом этапе работы. По результатам исследований можно сделать следующие выводы:

%\enlargethispage{4\baselineskip}

\begin{enumerate}
\item Особенностями полученных моделей являются:
\begin{itemize}
\item[а)] 	все построенные модели существенно неосесимметричны и неодносвязны;
\item[б)] модели аналитически определяют поля перемещений, напряжений и деформаций в теле;
\item[в)] модели позволяют точно учесть произвольную нагрузку, прикладываемую к телу;
\item[г)] предложенная структура моделей обуславливает оптимальность операторных уравнений для определения параметров моделей;
\item[д)] оптимальность связана с экспоненциальным убыванием матричных коэффициентов этих уравнений;
\item[е)] последнее свойство обеспечивает эффективную численную реализацию моделей, а также приближенные аналитические (в замкнутой форме) описания моделей.
\end{itemize}
\item Для проверки адекватности глобальные модели сравнивались с локальными моделями. Исследования показали, что первые можно заменять вторыми только в определенном диапазоне изменения геометрических и механических параметров.
\end{enumerate}
\end{russian}
\include{bibliography}
\begin{russian}
\tableofcontents
\end{russian}
\makelastpage
\end{document}
