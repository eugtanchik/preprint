%!TeX root = book.tex
%!TEX TS-program = lualatex

\begin{russian}
\setcounter{secnumdepth}{-1}
\chapter{Предисловие}
\markright{Предисловие}
\setcounter{secnumdepth}{2}

В последние десятилетия пористые и композиционные материалы получили широкое распространение не только в высокотехнологичных областях современной техники, но и в разных сферах окружающей нас жизни, таких, как микроэлектроника, транспортное машиностроение, строительство, дизайн, упаковка и др. Такой интерес к использованию этих материалов связан с их уникальными свойствами, в которых сочетаются высокие прочностные качества с малой удельной плотностью. Широта использования многокомпонентных материалов выдвигает на один из передних планов науки проблемы моделирования в них физико-механических полей, анализа их прочностных характеристик, создания оптимальных материалов с заранее заданными свойствами. Решение этих задач невозможно без переосмысления методов и подходов, сложившихся в механике композиционных материалов в последние годы, и создания новых методов исследования пространственных задач теории упругости для многосвязных тел с большим числом компонент связности. Надо заметить, что известные аналитико-численные методы недостаточно эффективны в подобных задачах. В конечном итоге все они не учитывают в полной мере структуру материала, что приводит зачастую к результатам с малой точностью.\sloppy

В данной монографии впервые в мировой научной литературе ставится проблема определения с любой практической степенью точности полей напряжений и деформаций в многосвязных пространственных телах, имеющих большое число неоднородных фаз. Для ее решения используется обобщенный метод Фурье (ОМФ), который был создан в работах одного из авторов монографии. В настоящей книге он получил дальнейшее развитие, которое позволило сделать его более адаптированным к задачам, рассматриваемым в монографии.

В книге исследованы пространственные краевые задачи теории упругости для многокомпонентных тел с неоднородностями в виде одинаково ориентированных цилиндрических, сферических, вытянутых и сжатых сфероидальных полостей или включений. Центры неоднородностей расположены в узлах плоской (цилиндры) или пространственной (сферы, сфероиды) решеток, которые обладают определенной трансляционной симметрией. Обычно в этом случае говорят о некоторой упаковке неоднородностей в материале. Рассматриваются случаи тетрагональной и гексагональной упаковок при наличии или отсутствии объемного центрирования.

Основное внимание в монографии уделено количественному и качественному анализу напряженно-деформированного состояния в зонах наибольшей концентрации напряжений в указанных выше многокомпонентных телах. Изучены локальные модели, в которых распределения напряжений определяются двумя или несколькими ближайшими неоднородностями, образующими ячейку рассматриваемой упаковки. Исследовано влияние дальних неоднородностей на значения напряжений в ячейке. Приводится анализ поведения напряжений в зависимости от изменения геометрических и механических параметров дисперсной фазы материала. Дано сравнение распределений напряжений для разных типов упаковок неоднородностей.

Кроме локальных моделей рассмотрены также модели напряженно-деформированного состояния многокомпонентного материала с периодически расположенными полостями или включениями.

Полученные результаты позволили исследовать в настоящей книге проблему определения эффективных упругих модулей для некоторых типов пористых и композиционных материалов.

Материал монографии полностью основан на исследованиях авторов книги, которые проводились в течение нескольких последних лет.

Авторы считают своим приятным долгом выразить благодарность рецензентам книги члену-корреспонденту НАН Украины, профессору А.~Н.~Довбне и профессору В.~А.~Дорошенко. Особую признательность авторы выражают Л.~А.~Кузьменко за квалифицированное литературное редактирование рукописи книги.

%\begin{flushright}
%\vskip1cm
%А.~Г.~Николаев, \\
%Е.~А.~Танчик. \\
%\vskip1cm
%Харьков, \\
%декабрь 2014~г.
%\end{flushright}

\end{russian}