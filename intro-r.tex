%!TeX root = report.tex
%!TEX TS-program = pdflatex

\addcontentsline{toc}{chapter}{Введение}
\begin{center}
{\normalsize\textbf{\centering ВВЕДЕНИЕ}}\vspace{14pt}
\end{center}

%\setcounter{secnumdepth}{-1}
%\chapter{Введение}
%\markright{Введение}
%\setcounter{secnumdepth}{2}

Современный уровень развития техники и технологии в высокотехнологических областях накладывает повышенные требования на точность и эффективность моделей материалов, которые широко используются в авиации и ракетостроении. Одной из наиболее важных характеристик материалов, которые здесь применяются, является такая комплексная характеристика, как малая удельная масса и одновременно высокая прочность материала. Такой характеристикой обладают материалы типа композитов, в которых присутствуют конструктивно заложенные неоднородности. При современном уровне моделирования появляется возможность конструирования материалов с заранее заданными свойствами вначале на уровне модели, определяя оптимальную структуру, геометрические размеры и механические характеристики неоднородностей. И только после этого полученные в результате моделирования данные можно воплощать в реальном материале.

По самому названию композиционный материал~--- это составной материал, обладающий гетерогенной структурой. Разнородные компоненты композита имеют различные физико-механические свойства. Особое сочетание этих свойств и геометрии неоднородностей приводит к качественно новым характеристикам композита, отличным от характеристик составляющих его фаз. Двухфазный композит~--- это однородный материал, армированный волокнами или зернами из другого материала. В зависимости от технологии изготовления армирующие элементы тем или иным способом пропитываются связующим веществом (матрицей), которое после застывания обеспечивает сплошность композиционной среды и идеальный механический и тепловой контакт между разнородными фазами. В качестве армирующих элементов обычно применяют материалы с кристаллической или аморфной микроструктурой, такие, как полимеры, стекла, металлы и др. К материалам заполнителя относятся полимеры, металлы, керамика. Свойства композита существенно зависят от физико-механических характеристик арматуры и матрицы, геометрии арматуры, структуры армирующих элементов, характера их упаковки, объемного содержания элементов, углов армирования и др. Укажем на те основные физико-механические характеристики композиционного материала, которые приводятся в нормативных документах и должны контролироваться в процессе его изготовления. Это модули упругости и пределы прочности на растяжение, сжатие и поперечный сдвиг, коэффициенты Пуассона и линейного температурного расширения вдоль и поперек волокон, удельная теплоемкость и др.{\sloppy\par}

В настоящей работе будут рассмотрены композиционные и пористые материалы, обладающие регулярной структурой. Под регулярной структурой понимается периодическое расположение слоев, зерен или волокон в материале. Считается, что фи\-зи\-ко-\-ме\-ха\-ни\-че\-ские характеристики включений одинаковые, но отличаются от характеристик матрицы. Далее будут рассмотрены только упругие модели деформирования этих материалов.

До настоящего времени многие важные задачи механики композиционных материалов остаются неизученными или недостаточно изученными. К ним относятся задачи определения напряженно-деформированного состояния образца из пористого и композиционного материалов в зависимости от приложенной внешней нагрузки, задачи выявления зон концентрации упругих напряжений как областей, в первую очередь подверженных разрушению, задачи анализа напряжений и деформаций в композиционном материале с отслоившимися включениями, задачи с межфазными и внутрифазными трещинами и др. Актуальной остается задача теоретического определения эффективных упругих модулей композиционных и пористых материалов. Все эти задачи в общей постановке относятся к классу весьма сложных задач механики деформированного твердого тела с многосвязной неоднородной структурой. До последнего времени эффективных методов решения подобных задач не существовало. 

{\bf Обзор литературы.} 
Приведем обзор научных исследований, связанных с тематикой данной работы. Исследованию пространственных задач теории упругости для односвязных цилиндра, шара и сфероида посвящены работы~\cite{Abramian, Andreev, Valov, Volpert1977, Volpert1967, Gomilko, Grinchenko1965, Grinchenko1967, Grinchenko1985, Grinchenko1978, Lur'e, Podilchuk1967, Podilchuk1984, Prokopov, Tokovyy, Ulitko, Chen1978-1, Chen1978-2, Edwards, Zhong, Zureick1989, Zureick1988, Ambartsumian, Arutynian, Kaufman, Kolesov, Meleshko}. В них краевые задачи для указанных областей решены методом Фурье.

В работах А.~Я.~Александрова~\cite{Aleksandrov1978, Aleksandrov1973} получил развитие метод интегральных наложений, позволивший связать пространственное напряженное состояние с плоским и, как следствие, выразить решения ряда пространственных задач для односвязных и многосвязных тел через аналитические функции комплексного переменного.

Г.~Н.~Положий на основе построенной им теории р–аналитических функций предложил метод решения осесимметричных задач теории упругости, аналогичный методу Колосова~--- Мусхелишвили для плоских задач. В дальнейшем этот метод был распространен на двусвязные тела в работах А.~А.~Капшивого, Л.~Н.~Ломоноса~\cite{Kapshiviy, Lomonos}.

Важный подход к решению пространственных задач теории упругости основан на применении методов теории потенциала. В работах В.~Д.~Купрадзе, Т.~Г.~Гегели, О.~М.~Башелейшвили, Т.~В.~Бурчуладзе~\cite{Kupradze, Burchuladze} путём исследования сингулярных интегральных уравнений различных классов краевых задач теории упругости были установлены условия существования и единственности их решений. Параллельно были созданы алгоритмы численного решения задач методами теории потенциала~\cite{Aleksandrov1973}. Численные возможности трех последних методов в многосвязных неосесимметричных телах весьма ограничены.
 
Методы теории потенциала в простанственных задачах теории трещин были развиты в монографии~\cite{Kit}.
  
В монографии~\cite{Guz1984} описан метод решения краевых задач теории упругости для областей, близких к каноническим, путем возмущения формы границы.

Приведем анализ методов решения пространственных задач для многосвязных тел. Прежде всего, заметим, что в этой области отсутствуют методы, близкие по общности и эффективности к методам решения плоских задач. Используемые здесь методы либо носят частный характер и предназначены для областей специального вида, не допуская распространения на более сложные области, либо, в виду своей общности, недостаточно эффективны при решении конкретных задач~\cite{Chen1978-2, Miyamoto, Sheikh, Strenberg}.

Многие задачи решаются численными методами~\cite{Method, Erzhanov}. Однако в телах с большим количеством неоднородностей эти методы неэффективны.

Остановимся на методах исследования механики композиционных материалов.

Определение интегральных упругих характеристик композиционных материалов проводилось различными методами в работах: Н.~С.~Бахвалова~\cite{Bahvalov}, Л.~Браутмана~\cite{Brautman}, А.~С.~Вавакина~\cite{Vavakin}, Ван Фо Фы~\cite{VanFoFiKo1971, VanFoFi1971}, Г.~А.~Ванина~\cite{Vanin1985, Vanin1994, Vanin1976, Vanin1980, Vanin1977}, А.~Н.~Власова, О.~К.~Гаришина~\cite{Garishin}, А.~В.~Головина, В.~Т.~Головчана~\cite{Golovchan1974, Golovchan1993, Golovchan1987, Golovchan2000}, В.~Т.~Гринченко~\cite{Grinchenko1965, Grinchenko1967, Grinchenko1985, Grinchenko1978}, А.~Н.~Гузя~\cite{Guz1968}, А.~В.~Дыскина~\cite{Diskin}, А.~В.~Ефименко, С.~П.~Киселёва, С.~П.~Копысова, Р.~Кристенсен~\cite{Kristensen}, Г.~Н.~Кувыркина, В.~И.~Куща~\cite{Kusch1995}, В.~М.~Левина~\cite{Levin}, Б.~П.~Маслова, В.~В.~Мошева, А.~А.~Панькова, Б.~Е.~Победри~\cite{Pobedrya}, Ю.~Н.~Подильчука~\cite{Podilchuk1967, Podilchuk1984}, В.~В.~Полякова, Я.~Я.~Рущицкого, Р.~Л.~Салганик~\cite{Salganik}, К.~Б.~Устинова~\cite{Ustinov}, А.~Ф.~Федотова~\cite{Fedotov}, В.~М.~Фомина, Л.~П.~Хорошуна~\cite{Khoroshun, Khoroshun2000-1, Khoroshun2000-2}, Т.~Д.~Шермергора~\cite{Schermergor}, Дж.~Эшелби~\cite{Eshelbi, Eshelby}, B.~Budiansky~\cite{Budiansky}, S.~Boucher~\cite{Boucher}, R.~M.~Christensen~\cite{Christensen, Christensen1990}, Z.~Hashin~\cite{Hashin, Hashin1983, Hashin1988, Hashin1963}, M.~Kachanov~\cite{Kachanov}, S.~Nemat-Nasser, M.~Taya~\cite{Nemat-Nasser}, A.~S.~Sangini, W.~Lu~\cite{Sangini}, S.~Torquato~\cite{Torquato}, R.~W.~Zimmerman~\cite{Zimmerman} и др.{\sloppy\par}

Один из первых подходов к определению эффективных упругих модулей был предложен в работах Дж.~Эшелби~\cite{Eshelbi, Eshelby}. Он основан на решении задачи об одиночном включении в форме эллипсоида. Метод предполагает пренебрежение взаимодействием между включениями.

В работе~\cite{Kachanov} при анализе эффективных упругих свойств тела с трещинами использовался метод эффективного поля, в котором предполагается, что включения находятся в поле напряжений, соответствующему среднему полю напряжений в матрице.

В ряде работ~\cite{Christensen1990} использовался метод самосогласования, в котором каждое включение находится в эквивалентной упругой среде, соответствующей матрице и остальным включениям, а также дифференциальный вариант метода самосогласования~\cite{Salganik, Vavakin, Zimmerman}.

В работе~\cite{Fedotov} предложен метод расчета эффективных упругих модулей зернистых композитов, основанный на модели упругого деформирования пористых материалов. Отличительная особенность метода заключается в осреднении микроскопических напряжений и деформаций не по полному, а по эффективному объему фаз. Получены расчетные зависимости эффективных объемов осреднения от упругих модулей и объемного содержания фаз. Проведено сопоставление результатов расчета с экспериментальными данными при различном сочетании упругих модулей и произвольной объемной концентрации фаз.

Модели многослойных материалов и конструкций исследовались в монографии~\cite{Bolotin}.

В книге~\cite{Gerega} рассматривается моделирование перколяционных кластеров фаз внутренних границ композиционных материалов.

В монографиях~\cite{Grigoliuk, Savin} решаются плоские задачи для многосвязных тел с отверстиями.

В книге~\cite{Erzhanov} метод конечных элементов применен при анализе напряжений в двухсвязных телах.

В статье~\cite{Zharkov} исследовано напряженно-деформированное состояние дисперсно наполненного полимерного композита с использованием объемных моделей.

В работе~\cite{Nazarenko} эффективные свойства трансверсально-изотропных композитных материалов определяются из стохастических дифференциальных уравнений нелинейной теории упругости применением метода условных моментов. В работе~\cite{Ilinikh} вычислительным экспериментом получены масштабные эффекты, определяющие зависимость прочностных характеристик материала от его структуры. В монографии~\cite{Bolshakov} упругие модули среды со сферическими включениями асимптотическим подходом получены из приближенных полидисперсной и трехфазной моделей. В статье~\cite{Michel} метод конечных элементов применяется для определения эффективных характеристик материала. В статье~\cite{Duan} метод самосогласованного поля применяется к анализу упругих характеристик материала с микроструктурой. В работе~\cite{Drugan} вариационный подход использован для получения материального уравнения, связывающего средние напряжения и деформации для класса случайных упругих композитных материалов.

В статьях~\cite{Khoroshun2000-1, Khoroshun2000-2, Scalon, Trias} обобщены базовые подходы, применяемые в математических моделях, и общие методы решения уравнений механики стохастических композитов. Они могут быть сведены к стохастическим уравнениям теории упругости структурно неоднородного тела, к уравнениям теории эффективных упругих модулей, к уравнениям теории упругих смесей или к более общим уравнениям четвертого порядка. Решение стохастических уравнений теории упругости для произвольной области вызывает значительные математические трудности и может быть реализовано только приближенно. Построение уравнений теории эффективных упругих модулей связано с задачей определения интегральных модулей стохастически неоднородной среды, которая может быть решена методом возмущений, методом моментов или методом условных моментов. Однако поскольку уравнения состояния не были строго обоснованы, эта теория не может использоваться для систематического моделирования композитных структур.

В работе~\cite{Golovchan2000} проанализировано термонапряженное состояние в окрестности сферических включений в   керамическом композите. Была строго решена краевая задача, которая соответствует приближенному моделированию механики композита парой включений. Численные результаты качественно согласуются с известными экспериментальными зависимостями.

В статье~\cite{Zureick1988} получено явное аналитическое решение задачи с жестким сфероидальным включением в трансверсально изотропном упругом пространстве, где включению заданы контактные перемещения в направлении, перпендикулярном к оси симметрии материала. Для решения задачи использовано представление потенциала перемещений для равновесия трехмерного трансверсально изотропного тела.

В работе~\cite{Zhong} исследован трансверсально изотропный стержень с цилиндрическим включением с осесимметричными собственными деформациями. Получено аналитическое упругое решение для перемещений, напряжений и энергии упругой деформации стержня.

В монографии~\cite{Vanin1985} разработаны методы микромеханики композиционных сред с дискретной структурой и трещинами и некоторые их приложения к конкретным материалам. Рассмотрено влияние свойств компонентов и вида структуры неоднородных сред на их эффективные (интегральные) параметры и распределение внутренних полей. Наряду с задачами теории упругости исследуются другие физические свойства материалов.

В статье~\cite{Garishin} предложена структурная модель зернистого эластомерного композита, позволившая связать его деформационное и прочностное поведение с размерами частиц дисперсной фазы, т.~е. учесть масштабный фактор прочности. На основе теоретических исследований на\-пря\-жен\-но-де\-фор\-ми\-ро\-ван\-но\-го состояния вокруг двух жестких сферических включений в упругой несжимаемой матрице установлены зависимости математического ожидания разрывного усилия от физико-механических характеристик связующего, размеров частиц и расстояния между ними. В результате предложен новый вероятностный критерий появления микроразрушения в композитной структуре в виде отслоений матрицы от частиц. С его помощью проведены модельные исследования процессов развития внутренней поврежденности в композитной системе в зависимости от степени наполнения и величины включений. Построены соответствующие кривые растяжения, определены предельные разрывные макронапряжения и макродеформации.

В работе~\cite{Shailiev} построена математическая модель расчета эффективного модуля упругости полиминеральных горных пород. Суть исследований заключается в расчете эффективного упругого модуля характерного объема материала путем осреднения по всем реализациям случайного поля неоднородностей с учетом их концентрации и пространственной ориентации. В работе были использованы методы: теории обобщенных функций, тензорного исчисления, теории уравнений математической физики и ин\-тег\-ро-\-диф\-фе\-рен\-ци\-аль\-ных уравнений.

В работе~\cite{Dudchenko} дано построение модели межфазного слоя материала, содержащего жесткую частицу под действием растягивающей нагрузки. Предлагается вариант чисто конструктивной расчетной модели. Основанием для построения такой модели являются результаты исследований, где показано, что в окрестности границ включения возникает дополнительная межфазная зона, механические свойства которой являются переменными, изменяясь по экспоненциальному закону от жесткости включения до жесткости матрицы. В рамках полученной расчетной модели учитываются размеры включения и протяженность межфазной зоны. Учитываются также изменение модулей упругости при повороте частицы по отношению к действующей нагрузке и влияние соседних частиц на свойства межфазного слоя. Приведены примеры расчета.

В статье~\cite{Chernous} представлены исследования по моделированию структуры пористых материалов с малой объемной долей содержания твердой фазы. В качестве моделей рассматриваются главным образом стержневые и оболочечные конструкции. Предложена классификация моделей по степени упорядоченности структурных единиц. Представляется, что наиболее адекватной с точки зрения морфологии и деформационных свойств является модель, состоящая из хаотически ориентированных 14-гранных ячеек. Для анализа представленной модели предлагается метод выделения структурного элемента.

В работе~\cite{Akbarov} методом конечных элементов исследовано влияние начального растяжения на концентрацию напряжений вокруг круговых отверстий в пластине-полосе, подверженной изгибу. Математическая формулировка соответствующей краевой задачи дается в рамках трехмерной линеаризованной теории упругости при плоском деформированном состоянии. Материал пластины-полосы~--- линейно-упругий, однородный и ортотропный. Представлены численные результаты, оценивающие влияние предварительного растяжения и взаимного расположения отверстий на концентрацию напряжений. Установлено, что начальное растяжение существенно уменьшает концентрацию напряжений в некоторых характерных точках на контуре отверстий.

В статье~\cite{Yankovskiy} разработана численно-аналитическая методика моделирования нелинейно-наследственного поведения композитов, имеющих про\-стран\-ст\-ве\-н\-но-ори\-ен\-ти\-ро\-ва\-н\-ную структуру (пространственно-армированных композитов), позволяющая в дискретные моменты времени рассматривать такую композицию как не\-ли\-ней\-но-уп\-ру\-гую. Применение итерационного процесса типа метода переменных параметров упругости позволило линеаризовать определяющие соотношения и свести задачу расчета механического поведения рассматриваемого композита в дискретные моменты времени к серии ли\-ней\-но-уп\-ру\-гих задач механики композитов. 

В статье~\cite{Golotina} исследованы зернистые композитные материалы с эластомерной матрицей, наполненной твердыми частицами диаметром $10-1000~\text{мкм}$. Рассмотрен один из возможных механизмов реологического поведения наполненных систем, связанный с возникновением и развитием вакуолей около жестких включений в вязкоупругой матрице. Для моделирования такого механизма формирования реологических свойств наполненного эластомера использована структурная ячейка в виде эластомерного цилиндра, высота которого равна диаметру, с жестким сферическим включением в центре цилиндра. Деформирование ячеек исследовали при соблюдении граничных условий, обеспечивающих сохранение их плотной упаковки при деформировании. Принято, что включение является жестким, а свойства матрицы описываются уравнениями линейной наследственной теории вязкоупругости. Для описания процесса роста вакуоли использован подход, согласно которому изначальное отслоение начинает распространяться, когда энергия, накопленная в матрице при растяжении, достигает величины, достаточной для создания новой поверхности раздела. Структурную неоднородность композита моделировали путем учета непостоянства локальной концентрации наполнителя. Рассчитаны кривые ползучести для композитов с разным содержанием твердой фазы. Проведено сравнение с экспериментальными данными, показавшее удовлетворительное согласование результатов.{\sloppy\par}

В работе~\cite{Gordeev} предлагается алгоритм оценки свойств волокнистого композита при растяжении, основанный на процедуре В.~З.~Власова; дается оценка эффективных свойств эквивалентного гомогенного материала; приводится сравнение результатов расчета эффективного модуля Юнга с результатами других авторов.

В статье~\cite{Belov} предлагается корректная модель сред с микроструктурой (по определению Миндлина), которая определяется наличием свободных деформаций и обобщает известные модели Миндлина, Коссера и Аэро-Кувшинского. Корректность формулировки модели определяется использованием ``кинематического'' вариационного принципа, основанного на последовательном формальном описании кинематики сред, формулировке кинематических связей для сред разной сложности и построении соответствующей потенциальной энергии деформации с использованием процедуры множителей Лагранжа. Ус\-та\-нав\-ли\-ва\-ет\-ся система определяющих соотношений и формулируется согласованная постановка краевой задачи. Показывается, что исследуемая модель среды не только отражает масштабные эффекты, аналогичные когезионным взаимодействиям, но и является основой для описания широкого спектра адгезионных взаимодействий. В связи с анализом физической стороны модели предлагается трактовка физических характеристик, ответственных за неклассические эффекты, дается описание спектра адгезионных механических параметров.

В работе~\cite{Garishin} разработаны модели структур многоуровневых волокнистых композитов с упрочняющими частицами в матрице и методы решения задач о плоском напряженном состоянии и разрушении. Предложен новый критерий хрупкого разрушения на основе интегралов, не зависящих от пути интегрирования. Найдены обобщенные равновесные термодинамические потенциалы неоднородных сред. Предложены методы определения поправок в состоянии материалов, возникающих от градиентных эффектов вблизи свободных и межфазных поверхностей, кончиков трещин, при высокочастотных волновых процессах. Указанные результаты оригинальны, и пока не известны их аналоги в литературе. Соотношения теории описывают состояние тел на выбранном масштабном уровне и связывают его с состоянием и процессами, протекающими на других масштабных уровнях неоднородных сред. Отмеченное представляет интерес в теории разрушения материалов, в задачах нанотехнологии и др.

Исследование напряженного состояния тела с однопериодической системой сферических полостей (осесимметричная задача) проводилось в работе~\cite{Olegin}.

Двусвязные осесимметричные задачи рассмотрены в работах~\cite{Smirnov, Atsumi, Mura, Tsuchida1980, Tsuchida1979}. Напряженно-деформированное состояние кусочно-однородных тел исследовано с помощью обобщенного метода Фурье в работах~\cite{Nikolaev2003, Nikolaev2004, Nikolaev2011, Nikolaev2007, Nikolaev2010, Nikolaev2009, Nikolaev2013, Nikolaev2012, Nikolaev-Tanchik2012, Nikolaev2013-1, Nikolaev2013-2, Nikolaev2013-19, Nikolaev2013-4, Nikolaev2014-1, Nikolaev2014-9, Nikolaev2013-3, Nikolaev2014-2, Nikolaev2014-3, Nikolaev2014-4, Nikolaev2014-5, Nikolaev2014-6, Nikolaev2014-7, Nikolaev2015-1}.
%Аппарат обобщенного метода Фурье развит в работах~\cite{Nikolaev1984, Nikolaev2011, Nikolaev1993, Nikolaev1998, Nikolaev1998-1}.

Приведенный обзор литературы показывает, что одной из актуальных проблем современной механики деформированного твердого тела является построение и анализ моделей напряженного состояния пористых и композиционных материалов, которые находят широкое применение в высокотехнологических областях техники, в частности в авиации и ракетостроении. Известные пространственные модели напряженного состояния указанных выше материалов не являются достаточно точными. Они учитывают неоднородную структуру материала или усредненно, или локально в окрестности только одного концентратора напряжений. При моделировании подобных материалов также используются разные типы стохастических моделей, которые, как правило, не являются полностью адекватными. Однако аэрокосмическая отрасль накладывает повышенные требования на прочностные характеристики материалов, управление которыми зависит от точности определения напряжений в теле. Создание более точных моделей указанных материалов позволит не только уточнить расчеты на прочность (коэффициенты концентрации напряжений на границе раздела фаз могут отличаться в 1.5--2 раза), но и на основе этих моделей проводить оптимальное проектирование материалов с заданными прочностью и массой. Обзор литературы показывает, что напряженное состояние в пористом и композиционном материалах традиционно моделируется напряженным состоянием представительской ячейки, содержащей одну неоднородность. Подобные модели используются при определении эффективных упругих модулей материалов с порами и включениями. Очевидно, эти модели заведомо не обладают высокой точностью. Основным их недостатком является тот факт, что они не вполне адекватны в локальных зонах концентрации напряжений в телах из пористых и композиционных материалов.

В монографии приводятся результаты исследований по моделированию на\-пря\-же\-н\-но-де\-фор\-ми\-ро\-ва\-н\-но\-го состояния упругого пористого и композиционного материалов с порами или включениями цилиндрической, сферической, вытянутой или сжатой сфероидальной формы. Пористый или композиционный материал моделируется упругой средой с конечным числом полостей или включений указанной выше формы. Число неоднородностей варьируется в широком диапазоне (от 2 до 30). Также рассматриваются задачи для упругого пространства с бесконечным числом полостей или включений, образующих периодическую структуру.
%Объектом исследования настоящей работы является многосвязный кусочно-однородный упругий материал. Предметом исследования являются модели НДС многосвязного кусочно-однородного упругого материала. Целью и задачами данного исследования является построение ана\-ли\-ти\-ко-чис\-ле\-н\-ных моделей НДС упругих тел с полостями и включениями, указанной выше формы.
%Важной составной частью исследования является строгий анализ полученных моделей, определение областей их эффективности, численный анализ полей напряжений и деформаций на основе построенных моделей, проверка адекватности предлагаемых моделей.
Модели НДС указанных тел строятся на основе точных базисных решений уравнения Ламе в канонических пространственных областях~\cite{Nikolaev1993, Nikolaev1998, Nikolaev1984}. Для определения параметров модели используется обобщенный метод Фурье. Аппарат обобщенного метода Фурье был развит в работах~\cite{Nikolaev1998, Nikolaev2011, Nikolaev1998-1, Nikolaev1993, Nikolaev1984}.

Остановимся кратко на содержании отчета.

В разделе 1 приведен аппарат обобщенного метода Фурье. Кроме известных теорем сложения базисных решений уравнения Ламе в цилиндрических, сферических, вытянутых и сжатых сфероидальных системах координат, начала которых произвольно сдвинуты друг относительно друга, впервые получены теоремы сложения для модифицированных базисных решений уравнения Ламе в указанных системах координат. Эти теоремы наиболее приспособлены для класса задач, который был исследован на третьем этапе научно-исследовательской работы, и впервые были получены в работах~\cite{Nikolaev2014-1, Nikolaev2014-9}.

%В главе 2 исследована механика упругого деформирования волокнистых пористых и композиционных материалов. Волокна моделируются параллельно расположенными цилиндрическими полостями или включениями. Рассмотрены локальные модели, в которых напряженно-деформированное состояние определяется цилиндрической неоднородностью и ее ближайшими соседями. Изучены также глобальные модели, которые учитывают всю структуру материала. Основное внимание уделяется количественному и качественному анализу в распределении напряжений в представительской ячейке в зависимости от механических и геометрических характеристик матрицы и включений, а также от типа упаковки полостей или включений. Приведены сравнения разных типов моделей. Результаты этой главы получены в работах~\cite{Nikolaev2013-1, Nikolaev2013-2, Nikolaev2013-4, Nikolaev2013-3, Nikolaev2015-1}.
% 
%Глава 3 посвящена механике упругого деформирования зернистых композиционных и пористых материалов со сферическими зернами. Рассмотрены локальные модели, в которых напряженно-деформированное состояние определяется сферической неоднородностью и ее ближайшими соседями. Изучены также глобальные модели, которые учитывают всю структуру материала. Основное внимание уделяется количественному и качественному анализу в распределении напряжений в представительской ячейке в зависимости от механических и геометрических характеристик матрицы и включений, а также от типа упаковки полостей или включений. Рассмотрены тетрагональная и гексагональная типы упаковок неоднородностей в случаях отсутствия или наличия объемного центрирования.
%Приведены сравнения этих типов моделей. Результаты этой главы получены в работах~\cite{Nikolaev2012, Nikolaev2014-2, Nikolaev2014-5, Nikolaev2014-7}.
%
%В главах 4 и 5 рассмотрены модели упругого деформирования пористых и композиционных материалов с вытянутыми или сжатыми сфероидальными полостями или включениями. Указанные материалы моделируются упругим пространством с конечным или бесконечным числом неоднородностей, центры которых расположены в узлах периодической решетки, обладающей определенной трансляционной симметрией. Предполагается, что оси симметрии включений одинаково направлены. Рассмотрены локальные модели, в которых напряженно-деформированное состояние определяется неоднородностью и ее ближайшими соседями. Изучены также глобальные модели, которые учитывают всю структуру материала. Приведен количественный и качественный анализ распределения напряжений в представительской ячейке в зависимости от механических и геометрических характеристик матрицы и включений, а также от типа упаковки неоднородностей. Рассмотрены тетрагональная и гексагональная типы упаковок неоднородностей в случаях отсутствия или наличия объемного центрирования.
%Приведены сравнения этих типов моделей. Результаты этой главы получены в работах~\cite{Nikolaev2010, Nikolaev-Tanchik2012, Nikolaev2013, Nikolaev2014-1, Nikolaev2014-9, Nikolaev2014-3, Nikolaev2014-4, Nikolaev2014-6}.
%
В разделе 2 изучены краевые задачи для уравнения Ламе в областях с бесконечным числом полостей или включений, которые образуют периодическую структуру. Рассмотрено пространство с полостями или включениями в форме сфер, вытянутых или сжатых сфероидов под действием одноосного, двуосного или всестороннего растяжений, приложенных на бесконечности. Приведен анализ распределения напряжений в представительской ячейке в зависимости от механических и геометрических характеристик матрицы и включений. Дано сравнение напряжений для разных типов моделей (локальные, глобальные и периодические).

Раздел 3 посвящен определению эффективных упругих модулей некоторых типов пористых и композиционных материалов. Построенные в разделе 2 модели позволяют с любой степенью точности восстанавливать напряженно-деформированное состояние в представительской ячейке пористого или композиционного материала, который находится под действием однородного поля напряжений, приложенных на бесконечности. Последнее обстоятельство дало возможность предложить метод определения эффективных упругих характеристик пористых и композиционных материалов, основанный на осреднении компонент тензоров напряжений и деформаций по объему представительской ячейки или по ее поверхности. Для примера получены формулы для расчета эффективных объемных модулей пористых и композиционных материалов с периодической системой сферических полостей или включений. Проведен численный анализ зависимости объемных модулей от объемного содержания полостей или включений в представительской ячейке материала. Приводится сравнение полученных результатов с известными, которые были вычислены на основании упрощенных моделей. 

В разделе 4 исследованы стохастические модели зернистых и композиционных материалов, в которых радиусы неоднородностей представляют собой  независимые одинаково распределенные случайные величины. Исследована величина разброса нормальных компонент тензора напряжений в зависимости от среднеквадратичного отклонения радиусов неоднородностей. Получены аналитические формулы для приближенных решений краевых задач пространственной теории упругости с произвольным конечным числом полостей или включений, центры которых расположены в узлах кубической решетки, а радиусы являются одинаково распределенными независимыми случайными величинами. Получены формулы для величин первых двух моментов компонент тензора напряжений. Приводится сравнение полученных результатов с результатами для точной детерминированной модели и приближенной детерминированной модели.

Создано программное обеспечение для численной реализации построенных моделей. На его основе проведен численный анализ и дана визуализация распределения напряжений в некоторых телах в зонах их максимальной концентрации. Исследована скорость сходимости приближенных методов решения операторных уравнений для определения параметров моделей. Проведено сравнение полученных результатов с результатами локальных и глобальных моделей, исследованных на первом и втором этапах работы.

Результаты исследований по первому и второму этапам данной науч\-но-ис\-сле\-до\-ва\-тель\-ской работы были приведены в отчетах:

Новые методы исследования линейных и нелинейных деформируемых тел из композиционных материалов. Построение локальных линейных моделей материалов с порами и композиционных материалов. ИН НТП: 0713U002822 (2012~г.);

Новые методы исследования линейных и нелинейных деформируемых тел из композиционных материалов. Построение глобальных линейных моделей материалов с порами и композиционных материалов. ИН НТП: 07140005344 (2013~г.).
  
%В рамках второго этапа бюджетной темы были построены и исследованы глобальные модели НДС в телах с неоднородностями, которые описывают поля напряжений и деформаций реального пористого или композиционного материала в зависимости от типа упаковки неоднородностей и их числа в упругом теле. Проведен строгий аналитический анализ построенных моделей и определены области их эффективности. Созданы пакеты прикладных программ для численной реализации построенных моделей. Проведен численный и качественный анализ распределения напряжений в некоторых рассматриваемых телах. Дано сравнение изученных глобальных моделей с локальными моделями, исследованными на первом этапе бюджетной темы. Результаты исследований приведены в работах~\cite{Nikolaev2013-1, Nikolaev2013-2, Nikolaev2013-3, Nikolaev2013-4, Nikolaev2013-5, Nikolaev2013-6, Nikolaev2013-7, Nikolaev2013-8, Nikolaev2013-9, Nikolaev2013-19}.

%Они докладывались на международных конференциях~\cite{Tanchik, Nikolaev2013-10, Nikolaev2013-11, Nikolaev2013-12, Nikolaev2013-13, Nikolaev2013-14, Nikolaev2013-15, Nikolaev2013-16, Nikolaev2013-17, Nikolaev2013-18}.

